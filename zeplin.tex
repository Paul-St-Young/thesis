\section{Basic Assumptions}
Suppose a particle can be described by a probability distribution over space $x$ and time $t$, then the ``uniform'' particle is described by a plane wave
\begin{align}
\psi \propto \exp\left(
i~\dfrac{px-Et}{\hbar}
\right),
\end{align}
with constant momentum $p$ and energy $E$. The imaginary $i$ is needed to keep the amplitude of this wave $\psi^*\psi$ constant, whereas $\hbar$ is needed to remove units from the exponent. From derivatives of this plane wave, it is natural to conjecture the relations of conjugate variables to space and time, i.e. momentum and energy
\begin{align}
\left\{
\begin{array}{l}
\frac{d\psi}{dx} =  i\frac{p}{\hbar}\psi \\ [8 pt]
\frac{d\psi}{dt} = -i\frac{E}{\hbar}\psi
\end{array}
\right. \Rightarrow
\left\{
\begin{array}{l}
p\psi = -i\hbar\frac{d}{dx} \psi \\ [8 pt]
E\psi =  i\hbar\frac{d}{dt} \psi
\end{array}
\right..
\end{align}
Suppose total energy is a sum of non-relativist kinetic energy and local potential energy
\begin{align}
E = \frac{p^2}{2m} + V(x),
\end{align}
then the Hamiltonian operator
\begin{align}
\hat{\mathcal{H}} = \frac{\hat{p^2}}{2m} + V(x) = -\frac{\hbar^2}{2m}\nabla^2 + V(x).
\end{align}

\section{Introduction}
%Exact simulation of electron-ion systems is the grand challenge of this century.
Exact simulation of the homogeneous electron gas was the holy grail of the last century. It has led to quantitative understanding of metals and semiconductors as well as the popular local density approximation of the density functional which opened the flood gates on countless materials simulations. However, the grand challenge of this century is the exact simulation of electron-ion systems.
While the jellium model mimic many essential features of electrons in real materials, valence electrons in real materials interact with point charges (nuclei) and the surrounding inner electrons that screen them.
To complicate matters, these nuclei form a crystalline arrangement only on average. They move around and can change the electronic structure significantly. Even at absolute zero, the zero-point motion of the ions can still be the deciding factor of the stability between two candidate crystal structures.
The electron-ion problem is far from being solved.
Along with excited states.

The quantum liquids helium and jellium are at the forefront of the quantum era ushered in the last century.
Having stable closed-shells, helium atoms interact mostly via an isotropic pair potential. It consists of a $r^{-6}$ van der Waals attraction at long range due to correlated dipole fluctuation and a hard-core repulsion upon electron density overlap~\cite{Aziz1979}.
Thus, the electronic problem is removed and only the ionic problem remains
\begin{align}
\hat{H}_{He} = \sum\limits_{I=1}^{N_I} -\dfrac{\hbar^2}{2m_I}\nabla_I^2
+\frac{1}{2}\sum\limits_{I=1}^{N_I}\sum\limits_{J=1,J\neq I}^{N_I} v(\vert\ri-\rj\vert).
\end{align}
In the opposite extreme, 
\begin{align}
\dfrac{1}{n} \equiv \dfrac{\Omega}{N} = \left\{\begin{array}{lr}
2 r_s a_B & 1D\\
\pi(r_sa_B)^2 & 2D\\
\frac{4\pi}{3}(r_sa_B)^3 & 3D
\end{array}\right.,
\end{align}

%\subsection{Exact Electron-Ion Schr\"odinger equations towards Born-Oppenheimer}


\begin{comment}
\begin{table}[h]
\centering
\begin{tabular}{cccccc}
\toprule
     & Temperature & Classical & Quantum & Quantization & Sign Problem \\
\midrule
PIMC &     any     &   yes     &   yes   & first & $\braket{\sigma}\propto \exp\left[ -\beta N (F_f-F_b) \right]$ \\
DMC  &    zero     &    no     &   yes   & first & $\braket{\sigma}\propto \exp\left[ -\beta N (F_f-F_T) \right]$ \\
FP-DMC & zero & no & yes & first & no \\
VMC & low & no & yes & first & no \\
AFQMC & low & no & yes & second & ? \\
\bottomrule
\end{tabular}
\end{table}
\end{comment}

The electron-ion pair correlation functions are shown in Fig.~\ref{fig:hsolid-epgr}.
They are shown separately for each candidate structure with pressure encoded in color.
For all candidate structures, the small-distance electron-proton correlation decrease as pressure increases.
As the kinetic energy becomes more important, the electrons delocalize more to lower kinetic energy, thus localize less around the protons.

\begin{figure}[h]
\centering
\begin{minipage}{0.49\textwidth}
\centering
\includegraphics[width=\linewidth]{h117ga_dynamic-c2c-s2-ep}\\
(a) C2/c-24
\end{minipage}
\begin{minipage}{0.49\textwidth}
\centering
\includegraphics[width=\linewidth]{h117ga_dynamic-cmca12-s2-ep}\\
(b) Cmca-12
\end{minipage}
\begin{minipage}{0.49\textwidth}
\centering
\includegraphics[width=\linewidth]{h117ga_dynamic-cmca4-s2-ep}\\
(c) Cmca-4
\end{minipage}
\begin{minipage}{0.49\textwidth}
\centering
\includegraphics[width=\linewidth]{h117ga_dynamic-i41amd-s2-ep}\\
(d) I4$_1$/amd
\end{minipage}
\caption{Dynamic-lattice electron-proton pair correlation function ep $g(r)$.}
\label{fig:hsolid-epgr}
\end{figure}