\section{Basic Assumptions}
Suppose a particle can be described by a probability distribution over space $x$ and time $t$, then the ``uniform'' particle is described by a plane wave
\begin{align}
\psi \propto \exp\left(
i~\dfrac{px-Et}{\hbar}
\right),
\end{align}
with constant momentum $p$ and energy $E$. The imaginary $i$ is needed to keep the amplitude of this wave $\psi^*\psi$ constant, whereas $\hbar$ is needed to remove units from the exponent. From derivatives of this plane wave, it is natural to conjecture the relations of conjugate variables to space and time, i.e. momentum and energy
\begin{align}
\left\{
\begin{array}{l}
\frac{d\psi}{dx} =  i\frac{p}{\hbar}\psi \\ [8 pt]
\frac{d\psi}{dt} = -i\frac{E}{\hbar}\psi
\end{array}
\right. \Rightarrow
\left\{
\begin{array}{l}
p\psi = -i\hbar\frac{d}{dx} \psi \\ [8 pt]
E\psi =  i\hbar\frac{d}{dt} \psi
\end{array}
\right..
\end{align}
Suppose total energy is a sum of non-relativist kinetic energy and local potential energy
\begin{align}
E = \frac{p^2}{2m} + V(x),
\end{align}
then the Hamiltonian operator
\begin{align}
\hat{\mathcal{H}} = \frac{\hat{p^2}}{2m} + V(x) = -\frac{\hbar^2}{2m}\nabla^2 + V(x).
\end{align}

The quantum liquids helium and jellium are at the forefront of the quantum era ushered in the last century.
Having stable closed-shells, helium atoms interact mostly via an isotropic pair potential. It consists of a $r^{-6}$ van der Waals attraction at long range due to correlated dipole fluctuation and a hard-core repulsion upon electron density overlap~\cite{Aziz1979}.
Thus, the electronic problem is removed and only the ionic problem remains
\begin{align}
\hat{H}_{He} = \sum\limits_{I=1}^{N_I} -\dfrac{\hbar^2}{2m_I}\nabla_I^2
+\frac{1}{2}\sum\limits_{I=1}^{N_I}\sum\limits_{J=1,J\neq I}^{N_I} v(\vert\ri-\rj\vert).
\end{align}
In the opposite extreme, 
\begin{align}
\dfrac{1}{n} \equiv \dfrac{\Omega}{N} = \left\{\begin{array}{lr}
2 r_s a_B & 1D\\
\pi(r_sa_B)^2 & 2D\\
\frac{4\pi}{3}(r_sa_B)^3 & 3D
\end{array}\right.,
\end{align}

%\subsection{Exact Electron-Ion Schr\"odinger equations towards Born-Oppenheimer}


\begin{comment}
\begin{table}[h]
\centering
\begin{tabular}{cccccc}
\toprule
     & Temperature & Classical & Quantum & Quantization & Sign Problem \\
\midrule
PIMC &     any     &   yes     &   yes   & first & $\braket{\sigma}\propto \exp\left[ -\beta N (F_f-F_b) \right]$ \\
DMC  &    zero     &    no     &   yes   & first & $\braket{\sigma}\propto \exp\left[ -\beta N (F_f-F_T) \right]$ \\
FP-DMC & zero & no & yes & first & no \\
VMC & low & no & yes & first & no \\
AFQMC & low & no & yes & second & ? \\
\bottomrule
\end{tabular}
\end{table}
\end{comment}