\section{Gaskell RPA Jastrow}
\label{sec:gaskell-rpa-jas}
% 2018-04-27 li-nofk-iso
The RPA Jastrow potential electron gas in 3D given by T. Gaskell~\cite{Gaskell1961, Ceperley1978, Holzmann2009, Holzmann2016} is
\begin{align}
2\rho u^{RPA}_k = \left[\left( S_0(k)^{-2} + 2\rho v_k/\epsilon_k \right)^{-1/2} \right] ^{-1} - S_0(k)^{-1}, \label{eq:wf-gaskell-rpa-uk}
\end{align}
where the $\nu_k$ is the Coulomb potential in reciprocal space and $\epsilon_k$ is the energy-momentum dispersion relation. For non-relativistic electrons in 3D, $\nu_k=\frac{4\pi}{k^2}$ and $\epsilon_k=\frac{k^2}{2}$ using Hatree atomic units. $S_0(k)$ is the static structure factor of the free Fermi gas ~\cite{Gori-Giorgi2000}
\begin{align}
S_0(k) = \left\{
\frac{3}{4}\left( \dfrac{k}{k_F} \right) - \frac{1}{16}\left(\dfrac{k}{k_F}\right)^3\right\} \Theta(2k_F-k) + \Theta(k-2k_F).
\end{align}
Gaskell~\cite{Gaskell1961} used an integral identity to obtain a general relation between the Jastrow potential and the static structure factor%~\cite{Holzmann2011} %~\cite{Gaskell1961, Holzmann2011, Holzmann2016} [Interacting Electrons (6.22)]
\begin{align}
2\rho u_{\bs{k}} = S^{-1}(\bs{k}) - S^{-1}_0(\bs{k}). \label{eq:uk-sk}
\end{align}
Therefore, the RPA structure factor can be read off of $u^{RPA}_k$ in eq.~(\ref{eq:wf-gaskell-rpa-uk}) via eq.~(\ref{eq:uk-sk})
\begin{align}
S^{RPA}(k) = \left( S_0^{-2}(k) + 2\rho v_k/\epsilon_k \right)^{-1/2}, \label{eq:wf-gaskell-rpa-sk}
\end{align}
where $k_F$ is the Fermi k-vector. For unpolarized electrons $k_F=3\pi^2\rho=\left(\frac{9\pi}{4r_s^3}\right)^{1/3}$.

Equation~(\ref{eq:wf-gaskell-rpa-sk}) is exact in the long wavelength $k\rightarrow0$ limit. Taylor expanding eq.~(\ref{eq:wf-gaskell-rpa-sk}) around $k=0$
\begin{align} \label{eq:wf-gaskell-rpa-sk-taylor}
S(k) = \frac{k^2}{2\omega_p}-\frac{k_F^2k^4}{9 \omega_p ^3}+O\left(k^6\right),
\end{align}
where the plasmon frequency $\omega_p=\sqrt{4\pi\rho}=\sqrt{3/r_s^3}$.

%The static structure factor of the valence electrons in lithium is quite close to that in jellium. In Fig.~\ref{fig:rpask}, the DMC pure-estimator structure factor of b.c.c. lithium is compared with the RPA result eq.~(\ref{eq:wf-gaskell-rpa-sk}). The agreement as $k\rightarrow0$ is quite good.

%\begin{figure}[h]
%%\includegraphics[scale=1.0]{figures/008b_rpask}
%\caption{Static structure factor of b.c.c. lithium.\label{fig:rpask}}
%\end{figure}

% 2018-04-24_gaskell-rpa_thesis
While Gaskell originally derived eq.~(\ref{eq:wf-gaskell-rpa-uk}) using perturbation theory, one can derive the same form by minimizing the variance of the local energy in the long wavelength limit, as shown in the next Sec.~\ref{sec:mc-rpa-jas}.