\section{Gaskell Random Phase Approximation (RPA) Jastrow}
% 2018-04-27 li-nofk-iso
The RPA Jastrow potential given by Gaskell~\cite{Gaskell1961, Ceperley1978, Holzmann2009, Holzmann2016} is
\begin{align}
2\rho u^{RPA}_k = \left[ S_0(k) \left( 1 + 2\rho S_0^2(k) \nu_k/\epsilon_k \right)^{-1/2} \right] ^{-1} - S_0(k)^{-1}, \label{eq:rpauk}
\end{align}
where the $\nu_k$ is the Coulomb potential in reciprocal space and $\epsilon_k$ is the energy-momentum dispersion relation. For non-relativistic electrons in 3D, $\nu_k=\frac{4\pi}{k^2}$ and $\epsilon_k=\frac{k^2}{2}$ using Hatree atomic units. $S_0(k)$ is the static structure factor of the free Fermi gas
\begin{align}
S_0(k) = \left\{\begin{array}{l}
\frac{3}{4}\left( \frac{k}{k_F} \right) - \frac{1}{16}\left(\frac{k}{k_F}\right)^3 ~~~, k<2k_F \\
1.0 ~~~~~~~~~~~~~~~~~~~~~~~~, k \geq 2k_F
\end{array}\right..
\end{align}

Assuming Gaussian statistics for $\rho_{\bs{k}}$, one can obtain a general relation between the Jastrow potential and the static structure factor~\cite{Holzmann2011} %~\cite{Gaskell1961, Holzmann2011, Holzmann2016} [Interacting Electrons (6.22)]
\begin{align}
2\rho u_{\bs{k}} = S^{-1}(\bs{k}) - S^{-1}_0(\bs{k}). \label{eq:uk-sk}
\end{align}
Therefore, the RPA structure factor can be read off of $u^{RPA}_k$ in eq.~(\ref{eq:rpauk}) via eq.~(\ref{eq:uk-sk})
\begin{align}
S^{RPA}(k) = S_0(k)\left( 1 + 2\rho S_0^2(k) \nu_k/\epsilon_k \right)^{-1/2}. \label{eq:rpask}
\end{align}

The static structure factor of the valence electrons in lithium is quite close to that in jellium. In Fig.~\ref{fig:rpask}, the DMC pure-estimator structure factor of b.c.c. lithium is compared with the RPA result eq.~(\ref{eq:rpask}). The agreement as $k\rightarrow0$ is quite good.

\begin{figure}[h]
%\includegraphics[scale=1.0]{figures/008b_rpask}
\caption{Static structure factor of b.c.c. lithium.\label{fig:rpask}}
\end{figure}

% 2018-04-24_gaskell-rpa_thesis
Given one-component 3D homogeneous electron gas, Gaskell RPA Jastrow reads~\cite{Holzmann2009}
\begin{align}
2\rho u(k) = -S_0^{-1}(k) + \left[ S_0^{-2}(k) + \left( \frac{2\rho \nu_k}{\epsilon_k} \right) \right] ^{1/2}, \label{eq:gaskell-rpa-uk}
\end{align}
where $\rho = (4\pi r_s^3/3)^{-1}$, $\nu_k = \frac{4\pi e^2}{k^2}$, $\epsilon_k=\frac{\hbar^2 k^2}{2m_e}$, and $S_0(k)$ is the non-interacting structure factor
\begin{align}
S_0(k) = \frac{3}{4} \left(\frac{k}{k_F}\right) - \frac{1}{16}\left(\frac{k}{k_F}\right)^3,\text{ for } k\in[0,2k_F]; 1.0 \text{ for } k > 2k_F.
\end{align}
$k_F=(3\pi^2\rho)=\left(\frac{9\pi}{4r_s^3}\right)^{1/3}$ is the Fermi k vector. In Hatree atomic units, eq.~(\ref{eq:gaskell-rpa-uk}) simplifies to
\begin{align}
2\rho u(k) = -S_0^{-1}(k) + \left[ S_0^{-2}(k) + \left( \frac{12}{r_s^3 k^4} \right) \right]^{1/2}. \label{eq:gaskell-uk-au}
\end{align}