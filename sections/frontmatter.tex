\frontmatter

% ProQuest truncates their abstracts at 350 words.
\begin{abstract}
This thesis explores properties of a mixture of electrons and ions using the quantum Monte Carlo method. In many electronic structure studies, purely electronic properties are calculated on a static potential energy surface generated by ``clamped'' ions. This can lead to quantitative errors, for example, in the prediction of diamond carbon band gap, as well as qualitatively wrong behavior, especially when light nuclei such as protons are involved. In this thesis, we explore different ways to include effects of dynamic ions and tackle challenges that arise in the process.
We benchmarked the diffusion Monte Carlo (DMC) method on electron-ion simulations consisting of small atoms and molecules. We found the method to be nearly exact once sufficiently accurate trial wave functions have been constructed. The difference between the dynamic-ion and static-ion simulations can mostly be explained by the diagonal Born-Oppenheimer correction.
We applied this method to solid hydrogen at megabar pressures and tackled additional problems involving geometry optimization and finite-size effects. The phase diagram produced by our electron-ion DMC simulations differ from previous DMC studies, showing $50$ GPa higher molecular-molecular transition and $150$ GPa higher molecular-to-atomic transition pressures. Both aforementioned studies forego the Born-Oppenheimer approximation (BOA) at hefty computational cost. Unfortunately, this makes it more difficult to compare our results with previous studies performed within the BOA.
The remainder of the thesis tackle finite-size and ionic effects within the BOA. We calculated the Compton profile of solid and liquid lithium, achieving excellent agreement with experiment. Ionic effects of the liquid were included by averaging over disorder atomic configurations. Finite-size correction was crucial for the Compton profile near the Fermi surface. Finally, we tackled the finite-size error in the calculation of band gaps and devised a higher-order correction, which allowed thermodynamic values of the band gap to be obtained from small simulation cells.
These advances mark important points along the path to the exact solution of the electron-ion problem. We expect that the better understanding of both the electron-ion wave function and its relation to finite-size effects obtained in this thesis to be crucial for future simulations of electron-ion systems.
\end{abstract}

%\begin{dedication}
%To my friends and family.
%\end{dedication}

%\chapter*{Acknowledgments}
%This project would not have been possible without the support of many people.