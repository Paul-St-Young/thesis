\frontmatter

% ProQuest truncates their abstracts at 350 words.
\begin{abstract}
This thesis explores properties of a mixture of electrons and ions using the quantum Monte Carlo method. In many electronic structure studies, purely electronic properties are calculated on a static potential energy surface generated by ``clamped'' ions. This can lead to quantitative errors, for example, in the prediction of diamond carbon band gap, as well as qualitatively wrong behavior, especially when light nuclei such as protons are involved. In this thesis, we explore different ways to include effects of dynamic ions and tackle challenges that arise in the process.
We benchmarked the diffusion Monte Carlo (DMC) method on electron-ion simulations consisting of small atoms and molecules. We found the method to be nearly exact once sufficiently accurate trial wave functions have been constructed. The difference between the dynamic-ion and static-ion simulations can mostly be explained by the diagonal Born-Oppenheimer correction.
We applied this method to solid hydrogen at megabar pressures and tackled additional problems involving geometry optimization and finite-size effects. The phase diagram produced by our electron-ion DMC simulations differ from previous DMC studies, showing $50$ GPa higher molecular-molecular transition and $150$ GPa higher molecular-to-atomic transition pressures. Both aforementioned studies forego the Born-Oppenheimer approximation (BOA) at hefty computational cost. Unfortunately, this makes it more difficult to compare our results with previous studies performed within the BOA.
The remainder of the thesis tackle finite-size and ionic effects within the BOA. We calculated the Compton profile of solid and liquid lithium, achieving excellent agreement with experiment. Ionic effects of the liquid were included by averaging over disorder atomic configurations. Finite-size correction was crucial for the Compton profile near the Fermi surface. Finally, we tackled the finite-size error in the calculation of band gaps and devised a higher-order correction, which allowed thermodynamic values of the band gap to be obtained from small simulation cells.
These advances mark important points along the path to the exact solution of the electron-ion problem. We expect that the better understanding of both the electron-ion wave function and its relation to finite-size effects obtained in this thesis can be crucial for future simulations of electron-ion systems.
\end{abstract}

\begin{dedication}
To my friends and family.
\end{dedication}

\chapter*{Acknowledgments}
First and foremost, I would like to thank my adviser, Prof. David Ceperley, for his unwavering support and guidance over the past six years.
His rigor, honesty, curiosity, and openness deeply influenced the way I view and do science.
I feel especially grateful for his fair treatment of students and collaborators.
I was given much freedom to pursue my own interests and always felt like a valued member of the team.
From these collaborations, I would like to thank Markus Holzmann and Carlo Pierleoni for many insightful discussions.
I would like to thank the QMCPACK team, especially Raymond Clay III, who guided me through some initial hurtles getting to know the code and continue to be responsive and helpful.
I look forward to continued collaborations, where I can learn from everyone and hopefully contribute more to our common interests.

I want to thank the CCMS program at LLNL, especially my mentor Miguel Morales.
His enthusiasm of research is infectious.
It was an invigorating summer interacting with Miguel and my fellow CCMS students Marnik Bercx, Ian Bakst, and Christopher Linder\"alv.

Of course, I would not be here without the loving care and support of my family and the wonderful company of my friends. I will not attempt to name everyone that was important for my life at Illinois, as I will invariably miss one of you even if I go on for pages.
I do have to thank my officemate Brian Busemeyer for not only being a model colleague, who is always willing to answer stupid questions and entertain crazy ideas, but also for being a kind and wonderful friend.
From our shared passion for coffee, badminton, swimming, and biking to our never-ending debates on Windows vs. Linux, Python 2 vs. 3, and functions vs. classes, I know we will remain colleagues and friends for years to come.

Thanks for funding from:
U.S. Department of Energy (DOE) Grant No. DE-FG02-12ER46875 as part of the Scientific Discovery through Advanced Computing (SciDAC). DOE Grant No. NA DE-NA0001789. DOE Grant No. 0002911. The Blue Waters sustained-petascalecomputing project and the Illinois Campus Cluster, supported by the National Science Foundation (Awards No. OCI-0725070 and No. ACI-1238993), the state of Illinois, the University of Illinois at Urbana-Champaign, and its National Center for Supercomputing Applications and resources of the Oak Ridge Leadership Computing Facility (OLCF) at the Oak  Ridge National Laboratory, which is supported by the Office of Science DOE Contract No. DE-AC05-00OR22725.
