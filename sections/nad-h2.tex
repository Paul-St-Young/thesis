\chapter{Nonadiabatic Coupling} \label{app:nad}

The electron-ion problem can be made more tractable at the cost of a physically motivated Born-Oppenheimer approximation (BOA). I define the BOA at the end of this section, but first lay out the exact formulation in eq.~(\ref{eq:intro-boa-coupled}) so that the content of the approximation is clear.
Ions move much slower than electrons due to their heavy mass ($m_I\approx 10^3$ to $10^5 m_i$), so it is sensible to isolate the ionic degrees of freedom and consider the electronic part separately. The coupling between the ionic and electronic problems is presumably weak because of the separation of timescales.
\begin{align} \label{eq:intro-ei-to-e-ham}
\hat{H} = -\sum_I \frac{\hbar^2}{2m_I} \nabla^2_I
+ \ham(\bs{R};\RI),
\end{align}
where $\ham$ is the clamped-ion or electronic hamiltonian, which typically defines the ultimate goal of an electronic structure method. The semicolon in $\ham(\bs{R};\RI)$ indicates that the electronic hamiltonian is only parametrically dependent on the ion positions $\RI$. M. Born and R. Oppenheimer (BO)~\cite{Born1927} first utilized this separation of timescales to study diatomic molecules in 1927. As explained around eq.~(27) and (28) in Ref.~\cite{Born1927}, BO expressed the electronic hamiltonian as a Taylor expansion around the equilibrium positions of the ions. They discussed results using the first four leading order terms in the vibration amplitude of the ions. Thus, what we define as ``the'' BOA can be ambiguous. Here, I follow the interpretation by G. A. Worth and L. S. Cederbaum~\cite{Worth2004}, which is equivalent to assuming a product ansatz eq.~(\ref{eq:intro-boa-product}), but without a ``diagonal correction".
%I will now explain the BOA in more detail.

If one can obtain the eigenstates of the electronic hamiltonian $\{\psi_k\}$ at any ion configuration $\bs{R}_I$
\begin{align}
\ham (\bs{R};\RI) \psi_k(\bs{R};\RI) = E_k(\RI) \psi_k(\bs{R};\RI),
\end{align}
then one can expand an eigenstate of the full hamiltonian $\hat{H}$ in the basis of electronic eigenstates
\begin{align} \label{eq:bo expansion}
\Psi_l(\bs{R},\RI) = \sum_{k=0}^\infty \chi_{lk}(\RI)\psi_k(\bs{R};\RI), 
\end{align}
where the expansion coefficients $\chi_{lk}(\RI)$ will later be identified with the ionic wave function in the Born-Oppenheimer approximation. $l$ runs over the full electron-ion hamiltonian's eigenstates, which can have both ionic (vibrational) and electronic characters. The coefficient for one of these \textit{vibronic} states cannot be determined separately for each electronic level $k$ in general. To see this, substitute the $l=0$ expansion eq.~(\ref{eq:bo expansion}) into the time-dependent Schr\"odinger equation for the full electron-ion hamiltonian (drop $l$ for simplicity)
\begin{align}
\left(\ham - \sum_I\frac{\hbar^2}{2M_I}\nabla_I^2\right)\left(\sum_k\chi_k\psi_k\right) = i\hbar\frac{d}{dt}\left(\sum_k\chi_k\psi_k\right) \nonumber \Rightarrow \text{apply operators} \\
\sum_k E_k\chi_k\psi_k-\sum_I\frac{\hbar^2}{2M_I}\left(\nabla^2_I\chi_k\psi_k+2\bs{\nabla}_I\chi_k\cdot\bs{\nabla}_I\psi_k+\chi_k\nabla_I^2\psi_k\right) = \sum_k i\hbar\dot{\chi_k}\psi_k \nonumber \Rightarrow \text{apply }\int\psi_j^* \\
\sum_k E_k\chi_k\delta_{jk}-\sum_I\frac{\hbar^2}{2M_I}\left(\nabla^2_I\chi_k\delta_{jk}+2\bs{\nabla}_I\chi_k\cdot\bs{F}_{jk}+\chi_kG_{jk}\right) = \sum_k i\hbar\dot{\chi_k}\delta_{jk} \nonumber \Rightarrow \text{perform }\sum_k \\
\left(-\sum_I\frac{\hbar^2}{2M_I}\nabla^2_I+E_j\right)\chi_j - \left(\sum_k\sum_I \frac{\hbar^2}{2M_I}\left(2\bs{F}_I^{jk}\cdot\bs{\nabla}_I+G_I^{jk}\right)\chi_k\right) = i\hbar\dot{\chi}_j, \label{eq:intro-boa-coupled}
\end{align}
where the matrix elements for gradient (derivative-coupling terms) and laplacian (scalar-coupling terms) in the electronic eigenstates basis are
\begin{align}
\left\{\begin{array}{l}
\bs{F}^{jk}_I = \int d\bs{r} \psi_j^*(\bs{r};\bs{R}) \bs{\nabla}_I\psi_k(\bs{r};\bs{R}) \\
G_I^{jk} = \int d\bs{r} \psi_j^*(\bs{r};\bs{R}) \nabla_I^2 \psi_k(\bs{r};\bs{R})
\end{array}\right..
\end{align}
The matrix elements that couple different electronic states in eq.~(\ref{eq:intro-boa-coupled}) are named \emph{nonadibatic coupling operators} by Worth and Cederbaum~\cite{Worth2004}
\begin{align} \label{eq:intro-boa-nona}
\Lambda_{jk} = \sum_I \frac{\hbar^2}{2M_I}\left(2\bs{F}_I^{jk}\cdot\bs{\nabla}_I+G_I^{jk}\right).
\end{align}
Every term in $\Lambda_{jk}$ has an inverse ion mass prefactor $\frac{\hbar^2}{2M_I}$, so they are expected to be small in most cases. There are two common approximations of $\Lambda_{kj}$, the first is to set the entire matrix to zero, the second is to set only the off-diagonal terms to zero. Both approximations decouple (\ref{eq:intro-boa-coupled}), allowing the complete separation of electronic and ionic motions.
%Therefore both approximations fall under the umbrella of adiabatic approximation, where the ion positions are considered to be slowly changing parameters of the electronic hamiltonian. 
Many different and sometimes conflicting names have been given to these two approximations including Born-Huang, Born-Oppenheimer and adiabatic approximation.
To fix nomenclature, I will call the all-zero approximation, $\Lambda_{jk}=0,~\forall j, k$, the Born-Oppenheimer approximation (BOA).
The diagonal terms $\Lambda_{jj}$ are considered diagonal Born-Oppenheimer correction (DBOC).
Non-zero off-diagonal elements are responsible for \textit{nonadiabatic effects}.

%\subsection{The Born-Oppenheimer Approximation}
The ground state in the BOA is a product of an ionic and an electronic component
\begin{align} \label{eq:intro-boa-product}
\Psi_{lk}^{BO}(\bs{R},\RI) = \chi_{lk}(\RI) \psi_k(\bs{R};\RI),
\end{align}
where a set of vibrational states labeled by $l$ can be defined over a particular electronic state $k$. $\chi_l(\RI)$ obeys its own Schr\"odinger equation on an effective potential energy surface provided by an eigenvalue of the electronic hamiltonian $E_k(\RI) = \braket{\psi_k|\ham|\psi_k}$, a.k.a. the Born-Oppenheimer potential energy surface (BO-PES)
\begin{align} \label{eq:boa-ion}
\left(-\sum_I\frac{\hbar^2}{2M_I}\nabla^2_I+E_k\right)\chi_l = i\hbar\dot{\chi}_l.
\end{align}
Once the ionic eigenstates are obtained by diagonalizing eq.~(\ref{eq:boa-ion}), the total energy of the electron-ion system is finally obtained as
\begin{align}
E_{lk}^{BO} \equiv \braket{\chi_l| E_k -\sum_I\frac{\hbar^2}{2M_I}\nabla^2_I |\chi_l}.
\end{align}
$E_{00}^{BO}$ differs in two ways from the electronic ground-state energy
\begin{align}
E_0 \equiv \braket{\psi_0|\ham|\psi_0}(\RI),
\end{align}
which is a function of the positions of the ions $\RI$. First, in $E_{00}^{BO}$ the electronic energy is averaged over a distribution of ion configurations $\vert\chi_0\vert^2(\RI)$ rather than evaluated at one fixed configuration $\RI$. This quantum delocalization effect raises the total energy from the bottom of the BO-PES $\RI^e=\underset{\RI}{\text{argmin}}~E_0(\RI)$, which would have been the electron-ion ground state if the ions were classical. Second, the ions have kinetic energy even at absolute zero, which also contributes a positive term to the total energy.
The difference between the electron-ion ground-state energy and the electronic one is the zero-point energy (ZPE). In the BOA, ZPE contains only two terms from delocalization and kinetic energy of the ions.

Within the BOA framework, solving the electron-ion problem for a particular combination of vibrational $l$ and electronic state $k$ involves finding the $k^{th}$ eigenvalue of the clamped-ion electronic problem $\mathcal{H}(\bs{R}; \bs{R}_I)$ for many ion configuration $\bs{R}_I$. There are established first-principle molecular dynamics and Monte Carlo methods for achieving this, but they are not practical for even moderately large system, e.g., O(1000) atoms, because the computational cost of electronic structure methods generally scale as $N^3$ or worse.

The main short-fall of the BOA is its lack of pathways for the ions to transfer energy to the electrons. This is critical in the study of radiation damage, where a fast moving ion can transfer energy to both the electrons and the ions in a material.
Further, for chemical reactions involving vibration-assisted bond breaking, the BOA reduces the number of pathways dissociation can happen, thereby resulting in an incomplete description.
The BOA can also break down if the electrons interact with a particle much lighter than an atomic nucleus such as a positron or a muon.
Finally, the nonadiabatic coupling terms can diverge when two electronic states cross, e.g., at a conical intersection.
Thus, it is sometimes important to go beyond the BOA.

%\subsection{Beyond the Born-Oppenheimer Approximation}

There are two small parameters that control the scale of nonadiabatic coupling eq.~(\ref{eq:intro-boa-nona}). One is clearly the inverse ionic mass $\frac{1}{M_I}$, while the other is the difference between electronic energy levels $\epsilon_j-\epsilon_k$. This can be seen from an explicit form of $\bs{F}_I^{jk}\equiv\braket{\psi_j|\bs{\nabla}_I|\psi_k}$ in the derivative-coupling term. Consider the effect of ion motion on the electronic problem, i.e., take $\bs{\nabla}_I$ of the time-independent electronic Schr\"odinger equation
\begin{align}
&\bs{\nabla}_I(\ham\psi_k) = \bs{\nabla}_I(\epsilon_k\psi_k ) \Rightarrow \psi_k\bs{\nabla}_I\ham + \ham\bs{\nabla}_I\psi_k = \psi_k\bs{\nabla}_I\epsilon_k + \epsilon_k\bs{\nabla}_I\psi_k \Rightarrow \nonumber \text{apply }\int\psi_j^* \\ &\left(\int \psi_j^*\psi_k\bs{\nabla}_I\ham\right) + \epsilon_j \bs{F}_I^{jk} = \bs{\nabla}_I\epsilon_k \delta_{jk}+\epsilon_k\bs{F}_I^{jk} \Rightarrow \nonumber \text{solve for }\bs{F}_I^{jk} \\
&\bs{F}_I^{jk} = \dfrac{\braket{\psi_j|\bs{\nabla}_I\ham|\psi_k}+\bs{\nabla}_I\epsilon_k\delta_{jk}}{\epsilon_k-\epsilon_j}.
\end{align}
If the electronic eigenstates are defined to be orthonormal, then the real part of the derivative-coupling vectors vanish $(\bs{F}_I^{kj})^*+\bs{F}_I^{jk}=\bs{0} \Rightarrow \text{Re}[\bs{F}_I^{jk}]=0$ as derived in eq.~(\ref{eq:intro-boa-refji}).
\begin{align}
&\left\{\begin{array}{l}
\bs{F}^{jk}_I = \braket{\psi_j|\bs{\nabla}_I\psi_k} \\
\braket{\psi_j|\psi_k} = \delta_{jk}
\end{array}\right. \Rightarrow \bs{\nabla}_I \braket{\psi_j|\psi_k} = \bs{0} \nonumber \Rightarrow \\
& \braket{\bs{\nabla}_I\psi_j|\psi_k} + \braket{\psi_j|\bs{\nabla}_I\psi_k}   = (\bs{F}_I^{kj})^*+\bs{F}_I^{jk} = \bs{0}.
\label{eq:intro-boa-refji}
\end{align}
$\bs{F}_I^{kj}$ can be interpreted as follows: the motion of the ions apply an imaginary ``force'' $\braket{\psi_j|\bs{\nabla}_I|\psi_k}$ that drives an electronic transition from state $k$ to state $j$. This interpretation has led to surface hopping methods for classical and quantum ions~\cite{Hammes-Schiffer1995,Sholl1998}, which have been applied successfully to describe proton transfer and proton-coupled electron transfer reactions~\cite{Hammes-Schiffer2015}. Further, this ``force'' is inversely proportional to the energy separation between the two eigenstates. The derivative-coupling term is considered more interesting than the scalar-coupling term due to its potential divergence as $\epsilon_k\rightarrow\epsilon_j$.

If the wave function is real, then there is no derivative coupling within the same electronic state $\bs{F}_I^{jj}=\braket{\psi_j|\bs{\nabla}_I\psi_j} = \braket{\bs{\nabla}_I\psi_j|\psi_j} = (\bs{F}_I^{jj})^* \Rightarrow \bs{F}_I^{jj}=\bs{0}$. In this case, the DBOC is simply the expectation value of the ion kinetic operator in the electronic state
\begin{align}
\Lambda_{jj} = \sum_I \frac{\hbar^2}{2M_I}G_I^{jj} = \int \psi_j^* \sum_I \frac{\hbar^2}{2M_I} \nabla^2_I\psi_j = \braket{\hat{T}_I}_j.
\end{align}

The diagonal correction for the hydrogen molecule was studied extensively by Kolos and Wolniewicz~\cite{Kolos1964,Kolos1968,Wolniewicz1983}.
For the atomization energy of $H_2$, the DBOC was found to be $4.947$ cm$^{-1}$, which is only a $0.0129$\% correction of its clamped-ion value of $38292.7$ cm$^{-1}$.
The nonadiabatic contribution to ionization and atomization energies of a few atoms and small molecules are explored in Ref.~\cite{Yang2015} and references therein.
%They did not explicitly calculate the diagonal correction for HD and D$_2$ because ``The corrections for HD and D$_2$ are readily obtained from those for H$_2$ by multiplications by the mass ratios''. I think the assumption here is that the geometry of the isotopes of H$_2$ are sufficiently close to that of H$_2$ that the electronic ground-state is unperturbed.