If one has obtained the eigenstates of the electronic hamiltonian $\{\psi_k\}$
\begin{align}
\ham (\bs{R};\RI) \psi_k(\bs{R};\RI) = E_k(\RI) \psi_k(\bs{R};\RI),
\end{align}
then one can expand an eigenstate of the full hamiltonian $\hat{H}$ in the basis of electronic eigenstates
\begin{align} \label{eq:bo expansion}
\Psi(\bs{R},\RI) = \sum_{k=0}^\infty \chi_k(\RI)\psi_k(\bs{R};\RI), 
\end{align}
where the expansion coefficients $\chi_k(\RI)$ will later be identified with the ionic wave function in the Born-Oppenheimer approximation. According to the Schr\"odinger equation, these coefficients cannot be determined for each ``vibration level'' $l$ separately in general. To see this, we substitute the expansion eq.~(\ref{eq:bo expansion}) into the time-dependent Schr\"odinger equation for the full electron-ion hamiltonian
\begin{align}
\left(\ham - \sum_I\frac{\hbar^2}{2M_I}\nabla_I^2\right)\left(\sum_k\chi_k\psi_k\right) = i\hbar\frac{d}{dt}\left(\sum_k\chi_k\psi_k\right) \nonumber \Rightarrow \text{apply operators} \\
\sum_k E_k\chi_k\psi_k-\sum_I\frac{\hbar^2}{2M_I}\left(\nabla^2_I\chi_k\psi_k+2\bs{\nabla}_I\chi_k\cdot\bs{\nabla}_I\psi_k+\chi_k\nabla_I^2\psi_k\right) = \sum_k i\hbar\dot{\chi_k}\psi_k \nonumber \Rightarrow \text{apply }\int\psi_j^* \\
\sum_k E_k\chi_k\delta_{jk}-\sum_I\frac{\hbar^2}{2M_I}\left(\nabla^2_I\chi_k\delta_{jk}+2\bs{\nabla}_I\chi_k\cdot\bs{F}_{jk}+\chi_kG_{jk}\right) = \sum_k i\hbar\dot{\chi_k}\delta_{jk} \nonumber \Rightarrow \text{perform }\sum_k \\
\left(-\sum_I\frac{\hbar^2}{2M_I}\nabla^2_I+E_j\right)\chi_j - \left(\sum_k\sum_I \frac{\hbar^2}{2M_I}\left(2\bs{F}_I^{jk}\cdot\bs{\nabla}_I+G_I^{jk}\right)\chi_k\right) = i\hbar\dot{\chi}_j, \label{eq:coupled}
\end{align}
where the matrix elements for gradient and laplacian in the electronic eigenstates basis
\begin{align}
\left\{\begin{array}{l}
\bs{F}^{jk}_I = \int d\bs{r} \psi_j^*(\bs{r};\bs{R}) \bs{\nabla}_I\psi_k(\bs{r};\bs{R}) \\
G_I^{jk} = \int d\bs{r} \psi_j^*(\bs{r};\bs{R}) \nabla_I^2 \psi_k(\bs{r};\bs{R})
\end{array}\right..
\end{align}
The matrix elements that couple different electronic states in eq.~(\ref{eq:coupled}) are named \emph{nonadibatic coupling operators} by Worth and Cederbaum~\cite{Worth2004}
\begin{align}
\Lambda_{jk} = \sum_I \frac{\hbar^2}{2M_I}\left(2\bs{F}_I^{jk}\cdot\bs{\nabla}_I+G_I^{jk}\right).
\end{align}
Every term in $\Lambda_{jk}$ has an inverse ion mass prefactor $\frac{\hbar^2}{2M_I}$, so they are expected to be small in most cases. There are two common approximations of $\Lambda_{kj}$, the first is to set the entire matrix to zero, the second is to set only the off-diagonal terms to zero. Both approximations decouple (\ref{eq:coupled}), which allows the complete separation of electronic and ionic motions.
%Therefore both approximations fall under the umbrella of adiabatic approximation, where the ion positions are considered to be slowly changing parameters of the electronic hamiltonian. 
Many different and sometimes conflicting names have been given to these two approximations including Born-Huang, Born-Oppenheimer and adiabatic approximation. To avoid confusion, I will call the all-zero approximation $\Lambda_{jk}=0,~\forall j, k$ the Born-Oppenheimer approximation (BOA).
The diagonal terms of $\Lambda_{jk}$ can be added as diagonal Born-Oppenheimer correction (DBOC).
Non-zero off-diagonal elements are responsible for \textit{nonadiabatic effects}.

The ground state in the BOA is a product of an ionic and an electronic component
\begin{align} \label{eq:intro-boa-product}
\Psi_{j 0}^{BO}(\bs{R},\RI) = \chi_j(\RI) \psi_0(\bs{R};\RI),
\end{align}
where the ion wave function for the $j^{\text{th}}$ vibrational state $\chi_j(\RI)$ obeys its own Schr\"odinger equation with an effective potential energy surface provided by the ground-state energy of the electronic hamiltonian $E_0(\RI) = \braket{\psi_0|\ham|\psi_0}$, a.k.a. the Born-Oppenheimer potential energy surface (BO-PES)
\begin{align} \label{eq:boa-ion}
\left(-\sum_I\frac{\hbar^2}{2M_I}\nabla^2_I+E_0\right)\chi_j = i\hbar\dot{\chi}_j.
\end{align}
Once the ionic eigenstates are obtained by diagonalizing eq.~(\ref{eq:boa-ion}), the total energy of the electron-ion system is finally obtained as
\begin{align}
E_{j0}^{BO} \equiv \braket{\chi_j| E_0 -\sum_I\frac{\hbar^2}{2M_I}\nabla^2_I |\chi_j}.
\end{align}
$E_{j0}^{BO}$ differs in two ways from the electronic ground-state energy
\begin{align}
E_0 \equiv \braket{\psi_0|\ham|\psi_0}(\RI),
\end{align}
which is a function of the positions of the ions $\RI$. First, the electronic energy is averaged over a distribution of ion configurations $\vert\chi_j\vert^2(\RI)$ rather than evaluated at one fixed configuration $\RI$. This quantum delocalization effect raises the total energy from the bottom of the BO-PES $\RI^e=\underset{\RI}{\text{argmin}}~E_0(\RI)$, which would be the electron-ion ground-state energy if the ions were classical. Second, the ions have kinetic energy even at absolute zero, which also contributes a positive term to the total energy.
The difference between the electron-ion ground-state energy and the electronic one is the zero-point energy (ZPE) of the electron-ion system. In the BOA, ZPE contains only two terms from delocalization and kinetic energy of the ions.

One short-fall of the BOA is its lack of pathways for the ions to transfer energy to the electrons. This is critical in the study of radiation damage, where a fast moving ion can transfer energy to both the electrons and the ions in a material. Further, bond breaking cannot occur via an electronic transition from binding to anti-binding state. The BOA can also break down if the electrons interact with a particle much lighter than an atomic nucleus, e.g., a positron or a muon. Finally, the nonadiabatic coupling terms can diverge when two electronic states cross, e.g., at a conical intersection. Thus, it is sometimes important to go beyond the BOA.

\subsection{Beyond the BOA}

\subsubsection{Small Parameters in Nonadiabatic Operator}
There are two small parameters that control the scale of nonadiabatic operator. One is clearly $\frac{1}{M_I}$, the other is $\epsilon_j-\epsilon_k$, which can only be seen from an explicit form of $\bs{F}_I^{jk}$. Take a $\bs{\nabla}_I$ of the time-independent electronic Schr\"odinger equation
\begin{align}
&\bs{\nabla}_I(H_{\text{el}}\psi_k) = \bs{\nabla}_I(\epsilon_k\psi_k ) \Rightarrow \psi_k\bs{\nabla}_IH_{\text{el}} + H_{\text{el}}\bs{\nabla}_I\psi_k = \psi_k\bs{\nabla}_I\epsilon_k + \epsilon_k\bs{\nabla}_I\psi_k \Rightarrow \nonumber \text{apply }\int\psi_j^* \\ &\left(\int \psi_j^*\psi_k\bs{\nabla}_IH_{\text{el}}\right) + \epsilon_j \bs{F}_{jk} = \bs{\nabla}_I\epsilon_k \delta_{jk}+\epsilon_k\bs{F}_I^{jk} \Rightarrow \nonumber \text{solve for }\bs{F}_I^{jk} \\
&\bs{F}_I^{jk} = \dfrac{\braket{\psi_j|\bs{\nabla}_IH_{\text{el}}|\psi_k}+\bs{\nabla}_I\epsilon_k\delta_{jk}}{\epsilon_k-\epsilon_j}.
\end{align}
If the electronic eigenstates are defined to be orthonormal, then $(\bs{F}_I^{kj})^*+\bs{F}_I^{jk}=\bs{0} \Rightarrow \text{Re}[\bs{F}_I^{jk}]=0$. 
\begin{align}
&\left\{\begin{array}{l}
\bs{F}^{jk}_I = \braket{\psi_j|\bs{\nabla}_I\psi_k} \\
\braket{\psi_j|\psi_k} = \delta_{jk}
\end{array}\right. \Rightarrow \bs{\nabla}_I \braket{\psi_j|\psi_k} = \bs{0} \nonumber \Rightarrow \\
& \braket{\bs{\nabla}_I\psi_j|\psi_k} + \braket{\psi_j|\bs{\nabla}_I\psi_k}   = (\bs{F}_I^{kj})^*+\bs{F}_I^{jk} = \bs{0}.
\end{align}
In addition, if the wave function is real then $\bs{F}_I^{jj}=\braket{\psi_j|\bs{\nabla}_I\psi_j} = \braket{\bs{\nabla}_I\psi_j|\psi_j} = (\bs{F}_I^{jj})^*$ and $\bs{F}_I^{jj}=\bs{0}$. In this case, the diagonal correction
\begin{align}
\Lambda_{jj} = \sum_I \frac{\hbar^2}{2M_I}G_I^{jj} = \int \psi_j^* \sum_I \frac{\hbar^2}{2M_I} \nabla^2_I\psi_j = \braket{\hat{T}}_j,
\end{align}
where $\hat{T}$ is the ion kinetic operator and $\braket{}_j$ denotes expectation in the $j^{\text{th}}$ electronic eigenstate.

\subsubsection{Diagonal Correction to H$_2$}
The diagonal correction for the hydrogen molecule was studied extensively by Kolos and Wolniewicz~\cite{Koos1964,Kolos1968,Wolniewicz1983}. They did not explicitly calculate the diagonal correction for HD and D$_2$ because ``The corrections for HD and D$_2$ are readily obtained from those for H$_2$ by multiplications by the mass ratios''. I think the assumption here is that the geometry of the isotopes of H$_2$ are sufficiently close to that of H$_2$ that the electronic ground-state is unperturbed.