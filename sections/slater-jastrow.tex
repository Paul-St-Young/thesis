\chapter{Slater-Jastrow wave function} \label{chap:sj-wf}

In the previous chapter, I introduced the FN-DMC method, which calculates ground-state properties of a many-body system starting from a trial wave function $\Psi_T$. The accuracy and efficiency of the method depend on the choice of $\Psi_T$.
Understanding of the many-body wave function and its connection to physical properties of particular systems can help us make educated guesses at high-quality trial wave function and perform accurate simulations.
In this chapter, I will describe the most well-understood many-body wave function for electronic structure, the Slater-Jastrow wave function, and discuss what behavior of electrons we can learn from it.

The many-body wave function is also interesting in its own right. One goal of studying the many-body wave function is to understand electron correlation~\cite{Ceperley1999-wf}.
As P.A.M. Dirac pointed out, knowing the Dirac/Schr\"odinger equation and the hamiltonian of the system does not constitute an understanding, because it ``leads to equations much too complicated to be soluble"~\cite{Dirac1929}.
Even a simple hamiltonian that contains only pair interactions, e.g., the Coulomb hamiltonian, can create complex many-body correlations and phase diagrams.
%Hogenberg-Kohn theorem tells us that the exact density functional encodes everything one needs to know about the ground state. How much physical intuition can be extracted from the exact density functional? Does one need to look at the charge density instead?
%What have we learnt from studying wave functions?
%Cusp condition, plasmon modes, back flow quasiparticle excitations
Thus, direct studies of experimental observables, the many-body wave function, and perhaps properties the exact density functional will be more informative.


\section{Historic Overview}

In condensed matter, the development of many-body wave function took off in the study of homogeneous quantum liquids, e.g., liquid helium and the homogeneous electron gas, a.k.a. jellium. Most studies made use of the variational principle eq.~(\ref{eq:wf-variational}), which states that given a Hamiltonian $\ham$, any normalized trial wave function $\Psi_T$ will have an energy value no less than that of the true ground state $\braket{\Psi_T|\ham|\Psi_T}\ge\braket{\Psi_0|\ham|\Psi_0}$. This principle allowed the pioneers to make educated guesses, then check their quality using the expectation value of $\ham$.
%Of course, the energy value is only one number. More are needed to truly quantify the accuracy of a wave function, but I digress.

As early as 1934, E. Wigner~\cite{Wigner1934} noted that by introducing a “hole” in the correlation function of opposite-spin elections, one can improve the Slater determinant and lower its energy value in the homogeneous electron gas. Although, this hint was not acted on until much later.
In 1940, A. Bijl~\cite{Bijl1940} found that the logarithm of the wave function of many-interacting particles is size-extensive. This logarithm can be expressed as a perturbation expansion involving one- and two-electron terms and is convergent in the thermodynamic limit.
Unfortunately, this work went unnoticed for 30 years while others independently developed similar ideas. Expanding the logarithm of the wave function is a general idea that later became known as the Bijl-Dingle-Jastrow-Feenberg expansion~\cite{Carleo2017} for historical reasons, which I will now describe.

In 1949, R. B. Dingle~\cite{Dingle1949}, while estimating the zero-point energy of hard spheres, came up with a product of exponential functions as a variational wave function by considering symmetries and limits. In 1955, R. Jastrow~\cite{Jastrow1955} generalized the Dingle pair-product wave functions to indistinguishable particles with Bose and Fermi statistics. To generalize to fermions, he multiplied by a Slater determinant to enforce antisymmetric permutation symmetry. Thus, the Slater-Jastrow wave function was born. While this thesis is mostly focused on the Slater-Jastrow wave function for electronic matter, the idea of separating particle statistics from correlation is general. The Slater-Jastrow wave function is the fermion variant of eq.~(\ref{eq:wf-prod-jas})
\begin{align} \label{eq:wf-prod-jas}
\Psi = \left\{\begin{array}{ll}
I(\{\phi_i\}) \exp(-U) & \text{distinguishabe} \\
P(\{\phi_i\}) \exp(-U) & \text{bosons} \\
D(\{\phi_i\}) \exp(-U) & \text{fermions}
\end{array}\right.,
\end{align}
where $I$, $P$, $D$ are identity, permanent, and determinant, respectively.
$\{\phi_i\}$ is a set of single-particle wave functions and $U$ consists of pair terms only
\begin{align} \label{eq:wf-pair-jas}
U = \sum\limits_{i<j} u(r_ij).
\end{align}
The minus sign in the definition of eq.~(\ref{eq:wf-prod-jas}) is intentional.
At high temperature, $\vert\Psi\vert^2$ for distinguishable particles becomes the Bolzmann distribution eq.~(\ref{eq:bolzmann-pi}). When only pair interaction is present, the pair contribution to $U$ has the same sign as the potential (see Chapter 6.6 in Ref.~\cite{Martin2016})
\begin{align}
\lim\limits_{\beta\rightarrow 0} u(r_ij) = \frac{1}{2}\beta v(r_ij).
\end{align}
Some important improvements to eq.~(\ref{eq:wf-prod-jas}) came from the study of bosons rather than fermions.

In 1956, R.P. Feynman and M. Cohen~\cite{Feynman1956} found it crucial to include the effect of back flow to accurately describe rotons in liquid helium. A roton is the quantum analog of a microscopic vortex ring. When an atom moves through the ring, it triggers a returning flow far from the ring, a.k.a. back flow. R.P. Feynman first estimated the energy-momentum curve of liquid helium using a permanent of plane wave orbitals in 1954, but found the roton energies severely over-estimated~\cite{Feynman1954}. It is only after the introduction of back flow into the trial wave function did the roton energy reduce to a reasonable value, in qualitative agreement with the phenomenological theory of Landau and with experiment~\cite{Feynman1956}.
In 1961, F.Y. Wu and E. Feenberg~\cite{Wu1961} related the Jastrow pair function, $u(r)$ in eq.~(\ref{eq:wf-pair-jas}), to the pair correlation function of liquid helium
\begin{align}
u(r) = g(r) - \dfrac{1}{8\pi^3\rho}\int d\bs{k} e^{-i\bs{k}\cdot\bs{r}}
\dfrac{[S(k)-1]^2}{1+\xi[S(k)-1]},
\end{align}
where $\xi\approx0.97$ gave accurate results under the superposition approximation.
Direct relation between experimentally measurable correlation functions and the wave function is an important avenue to glean understanding from the many-body wave function.
Similar ideas arose in the study of fermions in the same year.

%The most prominent success of the Slater-Jastrow wave function is its application to the homogeneous electron gas.
Also in 1961, T. Gaskell~\cite{Gaskell1961} derived a Jastrow pair function for homogeneous electron gas from perturbation calculation of its pair correlation function. He expressed the Jastrow pair function in collective coordinates and derived an analytical formula using the random phase approximation (RPA). By minimizing the total energy in the long wavelength limit, Gaskell found an accurate Jastrow pair function, having correct limits at both short and long wavelengths.
This Gaskell RPA wave function proved particularly useful in the study of the homogeneous electron gas and will be discussed in Sec.~\ref{sec:gaskell-rpa-jas}, then extended in Sec.~\ref{sec:mc-rpa-jas}.
Remarkably, this wave function is accurate in both 2D and 3D.
%In 1978, D. M. Ceperley~\cite{Ceperley1978} extended the Gaskell pair function to Wigner crystal.
Using the Gaskell RPA wave function as trial function, in 1980, D. M. Ceperley and B. J. Alder~\cite{Ceperley1980} found the ground state of the homogeneous electron gas in 3D using exact QMC simulations.

%In 1977, C. C. Chang and C. E. Campbell~\cite{Chang1977} added a three-body term to the Slater-Jastrow wave function and observed better agreement with experiment.
A common theme of these early successes is guessing and checking of correlation. One guesses that a many-body correlation is important, incorporates said correlation into the wave function, then checks if the energy value is lowered and/or correlation functions get closer to experiment. %This approach requires one to think long and hard about the physics of a particular problem, to have the mathematical prowess to connect observed correlation with a wave function form, and to be lucky enough to have some experimental data to compare with. This set of circumstances is less and less likely today
In contrast, recent improvements to the many-body wave function rely heavily on numerical optimization of general wave function forms with many parameters. Examples include orbital rotation in a single determinant, multi-determinant expansion, iterative back flow~\cite{PhysRevB.91.115106}, and neural network.
As we move towards these complicated wave functions, it will likely become increasingly difficult to extract physical understanding directly from the wave function.
This chapter will serve as a summary of some physical insights we have been able to grasp from the Slater-Jastrow wave function. I hope some of these will remain useful for more complex wave function forms. After some definitions, I will first discuss the short-range asymptotic behavior of the two-body contribution, i.e., the cusp condition in Sec.~\ref{sec:wf-cusp}. Second, I discuss the two-body long wavelength behavior by studying the Gaskell RPA Jastrow in Sec.~\ref{sec:gaskell-rpa-jas} for one-component system, then extend it to multi-component system in Sec.~\ref{sec:mc-rpa-jas}. Third, in Sec.~\ref{sec:wf-pw-sdet}, I show observables that can be calculated from the Slater determinant in plane wave basis, namely the momentum distribution from  one-particle reduced density matrix (1RDM) in Sec.~\ref{sec:wf-pw-sdet-nk} and the static structure factor from two-particle reduced density matrix (2RDM) in Sec.~\ref{sec:wf-pw-sdet-sk}.

\section{Definitions}

\begin{definition}
\label{sec:def-ft}
The \textit{Fourier Transform} of a 3D function in coordinate space is defined as
\begin{align}
f(\bs{k}) = \int d^3\bs{r} e^{i\bs{k}\cdot\bs{r}} f(\bs{r}). \label{eq:forward-ft}
\end{align}
The above Fourier transform convention eq.~(\ref{eq:forward-ft}) defines its inverse
\begin{align}
f(\bs{r}) = \int \frac{d^3\bs{k}}{(2\pi)^3} e^{-i\bs{k}\cdot\bs{r}} f(\bs{k}). \label{eq:inverse-ft}
\end{align}
The consistency of eq.~(\ref{eq:forward-ft}) and (\ref{eq:inverse-ft}) can be checked using the Coulomb potential $v(\bs{r})=\frac{1}{r}$ and $v(\bs{k})=\frac{4\pi}{k^2}$. 

In a finite cell of volume $\Omega$, momentum states are discretized. Each state takes up $\frac{(2\pi)^3}{\Omega}$ in reciprocal space. Therefore the inverse Fourier transform (\ref{eq:inverse-ft}) becomes
\begin{align} \label{eq:lattice-ift}
f(\bs{r}) = \frac{1}{\Omega} \sum\limits_{\bs{k}} e^{-i\bs{k}\cdot\bs{r}} f_{\bs{k}}.
\end{align}
\end{definition}
%\subsection{Definition 1: Fourier Transform}

\begin{definition}
The \textit{collective coordinates} of $N$ particles of species $\alpha$ is the Fourier transform of their instantaneous number density
\begin{align}
\rho^\alpha(\bs{k}) = \int d^3\bs{r} e^{i\bs{k}\cdot\bs{r}} \rho^\alpha(\bs{r})
= \int d^3\bs{r} e^{i\bs{k}\cdot\bs{r}} \sum\limits_{j=1}^{N_\alpha} \delta(\bs{r}-\bs{r}^\alpha_j)=\sum\limits_{j=1}^{N_\alpha} e^{i\bs{k}\cdot\bs{r}_j^\alpha}. \label{eq:rhok}
\end{align}
The collective coordinates provide a fixed basis for many-body functions in reciprocal space. Consider $N$ particles in a cell of volume $\Omega$ interacting via an isotropic pair potential $v(r)$. The potential energy
\begin{align}
V =& \sum\limits_{i<j} v(r_{ij}) = \frac{1}{2} \sum\limits_{i\neq j} v(r_{ij}) \nonumber \\
=& \frac{1}{2\Omega} \sum\limits_{\bs{k}}v_{\bs{k}} \sum\limits_{i\neq j}   e^{-i\bs{k}\cdot(\bs{r}_i-\bs{r}_j)} \nonumber \\
=& \frac{1}{2\Omega} \sum\limits_{\bs{k}}v_{\bs{k}} \left( \rho_{\bs{k}}\rho_{-\bs{k}}-N \right). \label{eq:pot-oc}
\end{align}
When generalized to multiple species, eq.~(\ref{eq:pot-oc}) becomes
\begin{align}
V =& \frac{1}{2}\sum\limits_{\alpha, \beta} \sum\limits_{i=1}^{N_\alpha} \sum\limits_{j=1}^{N_\beta, (j,\beta)\neq(i,\alpha)} v^{\alpha\beta}(\vert\bs{r}_i^\alpha - \bs{r}_j^\beta \vert) \nonumber \\
=& \frac{1}{2\Omega} \sum\limits_{\bs{k}}
v^{\alpha\beta}_{\bs{k}} \left( \rho^\beta_{\bs{k}\alpha\beta}\rho^\alpha_{-\bs{k}}-\delta_{\alpha\beta}N_\alpha \right). \label{eq:pot-mc}
\end{align}
For particles interacting via the Coulomb pair potential
\begin{align}
v^{\alpha\beta}(\bs{r}) = \frac{Q_\alpha Q_\beta}{r},
\end{align}
where $Q_\alpha$ and $Q_\beta$ are the charges of species $\alpha$ and $\beta$, respectively.
\end{definition}

\begin{definition}\textit{Jastrow Pair Function}:
The general form of a Jastrow wavefunction containing two-body terms is
\begin{align} \label{eq:wf-def-jas}
\Psi = \exp\left(-U\right),
\end{align}
where
\begin{align} \label{eq:wf-def-j2}
U =& \frac{1}{2}\sum\limits_{\alpha, \beta} \sum\limits_{i=1}^{N_\alpha} \sum\limits_{j=1}^{N_\beta, (j,\beta)\neq(i,\alpha)} 
u^{\alpha\beta}(\vert\bs{r}_i^\alpha - \bs{r}_j^\beta \vert) \nonumber \\
=& \frac{1}{2\Omega} \sum\limits_{\bs{k}\alpha\beta}
u^{\alpha\beta}_{\bs{k}} \left( \rho^\beta_{\bs{k}}\rho^\alpha_{-\bs{k}}-\delta_{\alpha\beta}N_\alpha \right).
\end{align}
$u^{\alpha\beta}(\bs{r})$ is the \emph{Jastrow pair function}. In the high temperature limit, $u^{\alpha\beta}(\bs{r})=\frac{v^{\alpha\beta}(\bs{r})}{2k_BT}$. $u^{\alpha\beta}_{\bs{k}}$ is the Fourier transform of $u^{\alpha\beta}(\bs{r})$ in the unit cell having volume $\Omega$ as defined by eq.~(\ref{eq:lattice-ift}).
\end{definition}

\begin{definition}
A \textit{Slater determinant} is a many-body wavefunction ansatz for the ground state of a collection of same-spin fermions. It is the anti-symmetrized version of a product wavefunction ansatz for distinguishable particles.
\begin{align}
\Psi = \frac{1}{\sqrt{N!}}\sum\limits_{\mathcal{P}} (-1)^{\mathcal{P}} \left( \prod\limits_{i=1}^N \phi_{\mathcal{P}_i}(\bs{r}_i) \right),\label{eq:det}
\end{align}
where $N$ is the number of fermions, $\bs{r}_1, \bs{r}_2, \dots, \bs{r}_N$ are their spatial coordinates. $\mathcal{P}$ is a permutation of the particle indices $1, 2, \dots, N$. $\phi_1, \phi_2, \dots, \phi_N$ are a set of one-body wave functions (a.k.a. orbitals).
\end{definition}

\section{Cusp Conditions}\label{sec:wf-cusp}
\emph{Derivation guided by Ex.~6.6 in Ref.~\cite{Martin2016}}

Consider two non-relativistic distinguishable particles having masses $m_1$ and $m_2$ interacting via a pair potential $v(r)$. The Schr\"odinger equation in the center-of-mass coordinate is
\begin{align}
\left[-\lambda\nabla^2 + v(r)\right] \psi = E\psi,
\end{align}
where $\lambda=\frac{\hbar^2}{2\mu}$, and $\mu=(m_1^{-1}+m_2^{-1})^{-1}$. The ground-state wave function
\begin{align}
\psi = \exp(-u(r))
\end{align}
should have a stationary local energy
\begin{align} \label{eq:wf-cusp}
E_L \equiv \dfrac{\ham\psi}{\psi} = v(r) + \lambda\nabla^2u(r) - (\bs{\nabla}u(r))^2 \nonumber \\
= v(r) + \lambda (u''+\frac{(d-1)u'}{r}) - \lambda {u'}^2 = \text{const.},
\end{align}
where $d$ is the number of spatial dimensions.
We see that the laplacian term in the kinetic energy has a potentially divergent term at $r=0$.
This term can respond to the potential and keep $E_L$ stationary, even if $v(r)$ has a divergence at $r=0$, e.g., the Coulomb potential. Suppose the two particles have charges $q_1$, $q_2$, and $v(r)=q_1q_2/r$, the condition for stationary $E_L$ is
\begin{align}
\lim\limits_{r\rightarrow 0}\frac{1}{r} (q_1q_2+ \lambda (d-1) u')=0 \Rightarrow u'(0) = -\dfrac{q_1q_2}{\lambda (d-1)}
\end{align}
For electron-electron interaction in Hartree atomic units $m_1=m_2=1$, so $\lambda=1$ and $u'(0)=-\frac{1}{2}$ in 3D. This is the cusp condition for unlike-spin electron pair. For same-spin pair, the two particles are indistinguishable and the laplacian for each particle contributes a copy of the divergent term, thus $u'(0)=-\frac{1}{4}$. For an electron-ion pair in the clamped-ion approximation ($m_2\rightarrow\infty$)
\begin{align} \label{eq:wf-ei-cusp}
u'(0) = \dfrac{2Z}{d-1},
\end{align}
where $Z$ is the atomic number of the ion. Imposing the cusp conditions on a trial wave function greatly reduces the variance of the local energy and improves the efficiency of a QMC calculation. The electron-ion cusp eq.~(\ref{eq:wf-ei-cusp}) is the most important one to maintain, because the wave function amplitude around an ion is high and many samples from the MC algorithm will have some electron close to an ion. In contrast, one rarely samples a configuration with two electrons close together due to strong electron-electron repulsion at density relevant for materials science, e.g., bulk silicon. However, at high density, electron-electron correlation is weak relative to kinetic energy, so the electron-electron cusp condition is important to maintain. Nevertheless, the effect of imposing the electron-electron cusp condition is typically less pronounced than that of the electron-ion one.

\begin{comment}
For the gravitational interaction $v(r)=Gm_1m_2/r^2$, eq.~(\ref{eq:wf-cusp}) implies
\begin{align}
\lim\limits_{r\rightarrow 0} u(r) = \alpha\ln r,
\end{align}
where $\alpha$ satisfies
\begin{align}
\alpha^2 + (2-d)\alpha - G(m_1+m_2) = 0.
\end{align}
\end{comment}

\section{Gaskell RPA Jastrow}
\label{sec:gaskell-rpa-jas}
% 2018-04-27 li-nofk-iso
The RPA Jastrow potential given by Gaskell~\cite{Gaskell1961, Ceperley1978, Holzmann2009, Holzmann2016} is
\begin{align}
2\rho u^{RPA}_k = \left[ S_0(k) \left( 1 + 2\rho S_0^2(k) \nu_k/\epsilon_k \right)^{-1/2} \right] ^{-1} - S_0(k)^{-1}, \label{eq:rpauk}
\end{align}
where the $\nu_k$ is the Coulomb potential in reciprocal space and $\epsilon_k$ is the energy-momentum dispersion relation. For non-relativistic electrons in 3D, $\nu_k=\frac{4\pi}{k^2}$ and $\epsilon_k=\frac{k^2}{2}$ using Hatree atomic units. $S_0(k)$ is the static structure factor of the free Fermi gas
\begin{align}
S_0(k) = \left\{\begin{array}{l}
\frac{3}{4}\left( \frac{k}{k_F} \right) - \frac{1}{16}\left(\frac{k}{k_F}\right)^3 ~~~, k<2k_F \\
1.0 ~~~~~~~~~~~~~~~~~~~~~~~~, k \geq 2k_F
\end{array}\right..
\end{align}

Assuming Gaussian statistics for $\rho_{\bs{k}}$, one can obtain a general relation between the Jastrow potential and the static structure factor~\cite{Holzmann2011} %~\cite{Gaskell1961, Holzmann2011, Holzmann2016} [Interacting Electrons (6.22)]
\begin{align}
2\rho u_{\bs{k}} = S^{-1}(\bs{k}) - S^{-1}_0(\bs{k}). \label{eq:uk-sk}
\end{align}
Therefore, the RPA structure factor can be read off of $u^{RPA}_k$ in eq.~(\ref{eq:rpauk}) via eq.~(\ref{eq:uk-sk})
\begin{align}
S^{RPA}(k) = S_0(k)\left( 1 + 2\rho S_0^2(k) \nu_k/\epsilon_k \right)^{-1/2}. \label{eq:rpask}
\end{align}

%The static structure factor of the valence electrons in lithium is quite close to that in jellium. In Fig.~\ref{fig:rpask}, the DMC pure-estimator structure factor of b.c.c. lithium is compared with the RPA result eq.~(\ref{eq:rpask}). The agreement as $k\rightarrow0$ is quite good.

%\begin{figure}[h]
%%\includegraphics[scale=1.0]{figures/008b_rpask}
%\caption{Static structure factor of b.c.c. lithium.\label{fig:rpask}}
%\end{figure}

% 2018-04-24_gaskell-rpa_thesis
Given one-component 3D homogeneous electron gas, Gaskell RPA Jastrow reads~\cite{Holzmann2009}
\begin{align}
2\rho u(k) = -S_0^{-1}(k) + \left[ S_0^{-2}(k) + \left( \frac{2\rho \nu_k}{\epsilon_k} \right) \right] ^{1/2}, \label{eq:gaskell-rpa-uk}
\end{align}
where $\rho = (4\pi r_s^3/3)^{-1}$, $\nu_k = \frac{4\pi e^2}{k^2}$, $\epsilon_k=\frac{\hbar^2 k^2}{2m_e}$, and $S_0(k)$ is the non-interacting structure factor
\begin{align}
S_0(k) = \frac{3}{4} \left(\frac{k}{k_F}\right) - \frac{1}{16}\left(\frac{k}{k_F}\right)^3,\text{ for } k\in[0,2k_F]; 1.0 \text{ for } k > 2k_F.
\end{align}
$k_F=(3\pi^2\rho)=\left(\frac{9\pi}{4r_s^3}\right)^{1/3}$ is the Fermi k vector. In Hatree atomic units, eq.~(\ref{eq:gaskell-rpa-uk}) simplifies to
\begin{align}
2\rho u(k) = -S_0^{-1}(k) + \left[ S_0^{-2}(k) + \left( \frac{12}{r_s^3 k^4} \right) \right]^{1/2}. \label{eq:gaskell-uk-au}
\end{align}

% 2018-08-20_tc-rpa-jas_thesis
\section{Multi-Component RPA Jastrow}
\label{sec:mc-rpa-jas}

\textit{Based on notes from D. M. Ceperley dated Sep. 1980}

Given Jastrow wavefunction $\Psi=\exp(-U)$, where
\begin{align}
U = \sum\limits_{i<j} u(r_{ij}) = \frac{1}{2} \sum\limits_{\alpha, \beta} \sum\limits_{i=1}^{N_\alpha} \sum\limits_{j=1}^{N_\beta, (j,\beta)\neq(i,\alpha)}
u^{\alpha\beta}(\vert\bs{r}_i^\alpha-\bs{r}_j^\beta\vert), \label{eq:Uurij}
\end{align}
and non-relativistic Coulomb hamiltonian
\begin{align}
H = \hat{T}+V = \sum\limits_{\alpha}\sum\limits_{j=1}^{N_\alpha} -\lambda_{\alpha}\nabla^{\alpha2}_j + \frac{1}{2}\sum\limits_{\alpha, \beta} \sum\limits_{i=1}^{N_\alpha} \sum\limits_{j=1}^{N_\beta, (j,\beta)\neq(i,\alpha)} v^{\alpha\beta}(\vert\bs{r}_i^\alpha - \bs{r}_j^\beta \vert),
\end{align}
where $\alpha, \beta$ label particle species, $i, j$ label particle positions. $\lambda_\alpha=\frac{\hbar^2}{2m_\alpha}$, $v^{\alpha\beta}(\bs{r})=\frac{Q_\alpha Q_\beta}{r}$. In terms of pair potentials and collective coordinates (see Fourier convention eq.~(\ref{eq:forward-ft}) and its corollaries eq.~(\ref{eq:inverse-ft}-\ref{eq:rhok}))
\begin{align}
U = \frac{1}{2\Omega} \sum\limits_{\bs{k}\alpha\beta}
u^{\alpha\beta}_{\bs{k}} \left( \rho^\beta_{\bs{k}}\rho^\alpha_{-\bs{k}}-\delta_{\alpha\beta}N_\alpha
\right), \label{eq:Uuk} \\
V = \frac{1}{2\Omega} \sum\limits_{\bs{k}\alpha\beta}
v^{\alpha\beta}_{\bs{k}} \left( \rho^\beta_{\bs{k}}\rho^\alpha_{-\bs{k}}-\delta_{\alpha\beta}N_\alpha \right).
\end{align}

The \textbf{goal} is to obtain good Jastrow pair potentials $u_{\bs{k}}^{\alpha\beta}$. The \textit{strategy} is to minimize the variance the local energy $E_L\equiv \Psi^{-1}H\Psi = T + V$, where
\begin{align}
T =& \sum\limits_{\gamma}-\lambda_{\gamma}
\sum\limits_{l=1}^{N_\gamma}  \left(\bs{\nabla}_l^{\gamma}U\cdot\bs{\nabla}_l^{\gamma}U - \nabla^{\gamma2}_l U \right).
\end{align}

In the following, I will detail the few steps needed to obtain the RPA Jastrow potentials. First, we express the local energy in terms of the collective coordinates eq.~(\ref{eq:rhok}). Second, we find the equations that make the local energy invariant to changes in the collective coordinates. Third and finally, we solve these equations for one and two component systems. Assume $u^{\alpha\beta}=u^{\beta\alpha}$ and $u_{\bs{k}}=u_{-\bs{k}}$.

\subsection{Local Energy of Jastrow Wavefunction}
The gradient, laplacian, and gradient squared of eq.~(\ref{eq:Uuk}) are
\begin{align}
%\left\{\begin{array}{l}
\bs{\nabla}_l^\gamma U =& \frac{1}{2\Omega} \sum\limits_{\bs{k}\alpha} (i\bs{k})u_{\bs{k}}^{\gamma\alpha} \left(
e^{i\bs{k}\cdot\bs{r}_l^\gamma}\rho_{-\bs{k}}^\alpha - \rho_{\bs{k}}^\alpha e^{-ik\cdot\bs{r}_l^\gamma}
\right) \\
\bs{\nabla}_l^\gamma\cdot\bs{\nabla}_l^\gamma U =& -\frac{1}{2\Omega}\sum\limits_{\bs{k}\alpha} k^2u_{\bs{k}}^{\gamma\alpha}\left(
e^{i\bs{k}\cdot\bs{r}_l^\gamma}\rho_{-\bs{k}}^\alpha + \rho_{\bs{k}}^\alpha e^{-\bs{k}\cdot\bs{r}_l^\gamma} - 2\delta_{\alpha\gamma}
\right) \label{eq:lapl}\\
\bs{\nabla}_l^\gamma U\cdot\bs{\nabla}_l^\gamma U =& -\frac{1}{4\Omega^2}\sum\limits_{\bs{k}\bs{q}\alpha\beta}\bs{k}\cdot\bs{q} u_{\bs{k}}^{\gamma\alpha}u_{\bs{q}}^{\gamma\beta} \times\left(\right.\nonumber \\
e^{i(\bs{k}+\bs{q})\cdot\bs{r}}\rho_{-\bs{k}}^\alpha\rho_{-\bs{q}}^\beta -
e^{i(\bs{k}-\bs{q})\cdot\bs{r}}\rho_{-\bs{k}}^\alpha\rho_{\bs{q}}^\beta -&
e^{i(\bs{q}-\bs{k})\cdot\bs{r}}\rho_{\bs{k}}^\alpha\rho_{-\bs{q}}^\beta +
e^{-i(\bs{k}+\bs{q})\cdot\bs{r}}\rho_{\bs{k}}^\alpha\rho_{\bs{q}}^\beta \nonumber \\
&\left.\right) \label{eq:grad2l}.
%\end{array}\right..
\end{align}
Summing over $l$ turns $e^{i\bs{k}\cdot\bs{r}_l^\gamma}$ into $\rho_{\bs{k}}^\gamma$ in eq.~(\ref{eq:lapl}) and (\ref{eq:grad2l}). Thus
\begin{align}
\sum\limits_{l=1}^{N_\gamma}\bs{\nabla}_l^\gamma\cdot\bs{\nabla}_l^\gamma U =& -\frac{1}{2\Omega}\sum\limits_{\bs{k}\alpha} k^2u_{\bs{k}}^{\gamma\alpha}\left(
\rho_{\bs{k}}^\gamma\rho_{-\bs{k}}^\alpha + \rho_{\bs{k}}^\alpha \rho_{-\bs{k}}^\gamma - 2N_\gamma\delta_{\alpha\gamma}
\right) \label{eq:lap}\\
\sum\limits_{l=1}^{N_\gamma}\bs{\nabla}_l^\gamma U\cdot\bs{\nabla}_l^\gamma U =& -\frac{1}{4\Omega^2}\sum\limits_{\bs{k}\bs{q}\alpha\beta}\bs{k}\cdot\bs{q} u_{\bs{k}}^{\gamma\alpha}u_{\bs{q}}^{\gamma\beta} \times\left(\right.\nonumber \\
\rho_{\bs{k}+\bs{q}}^\gamma\rho_{-\bs{k}}^\alpha\rho_{-\bs{q}}^\beta -
\rho_{\bs{k}-\bs{q}}^\gamma\rho_{-\bs{k}}^\alpha\rho_{\bs{q}}^\beta -&
\rho_{\bs{q}-\bs{k}}^\gamma\rho_{\bs{k}}^\alpha\rho_{-\bs{q}}^\beta +
\rho_{-(\bs{k}+\bs{q})}^\gamma\rho_{\bs{k}}^\alpha\rho_{\bs{q}}^\beta \nonumber \\
&\left.\right) \label{eq:grad2}.
\end{align}
Eq. (\ref{eq:grad2}) contains terms that couple three wave vectors, i.e. $O(\rho^3)$. In the spirit of RPA, we will drop all such \emph{mode coupling} terms. Note $\rho_{\bs{0}}^\gamma = N_\gamma$, and use $u_{\bs{k}}=u_{-\bs{k}}$
\begin{align}
\sum\limits_{l=1}^{N_\gamma}\bs{\nabla}_l^\gamma U\cdot\bs{\nabla}_l^\gamma U =&
\frac{N_\gamma}{2\Omega^2}\sum\limits_{\bs{k}\alpha\beta} k^2u_{\bs{k}}^{\gamma\alpha}u_{\bs{k}}^{\gamma\beta} \left( \rho_{-\bs{k}}^\alpha\rho_{\bs{k}}^\beta + \rho_{\bs{k}}^\alpha\rho_{-\bs{k}}^\beta \right). \label{eq:grad2_rpa}
\end{align}
Finally, sum over $\gamma$ with $-\lambda_\gamma$ to obtain terms in the kinetic energy. To simplify later assembly of the local energy, rename dummy variables $\alpha, \beta, \gamma$ such that every $O(\rho^2)$ term contains $\rho_{\bs{k}}^\alpha\rho_{-\bs{k}}^\beta$ (use $u^{\alpha\beta}=u^{\beta\alpha}$)
\begin{align}
\sum\limits_{\gamma}-\lambda_\gamma\sum\limits_{l=1}^{N_\gamma}\bs{\nabla}_l^\gamma U\cdot\bs{\nabla}_l^\gamma U =&-\frac{1}{\Omega}
\sum\limits_{\bs{k}\alpha\beta\gamma} \lambda_\gamma\frac{N_\gamma}{\Omega}
k^2u_{\bs{k}}^{\gamma\alpha}u_{\bs{k}}^{\gamma\beta} \rho_{\bs{k}}^\alpha\rho_{-\bs{k}}^\beta, \\
\sum\limits_{\gamma}-\lambda_\gamma\sum\limits_{l=1}^{N_\gamma}\bs{\nabla}_l^\gamma \cdot\bs{\nabla}_l^\gamma U =& -\frac{1}{\Omega}\left( 
\sum\limits_{\bs{k}\alpha\beta} \frac{\lambda_\alpha+\lambda_\beta}{2} u_{\bs{k}}^{\alpha\beta}\rho_{\bs{k}}^\alpha\rho_{-\bs{k}}^\beta -N_\alpha\lambda_\alpha\delta_{\alpha,\beta} \right).
\end{align}

Finally, the local energy can be assembled
\begin{align}
E_L = \sum\limits_{\bs{k}}\left[\frac{v^{\alpha\beta}_{\bs{k}}}{2\Omega} - 
\frac{\frac{\lambda_\alpha+\lambda_\beta}{2}k^2u_{\bs{k}}^{\alpha\beta}}{\Omega} -
\sum\limits_\gamma \frac{(N_\gamma/\Omega)\lambda_\gamma k^2u_{\bs{k}}^{\alpha\gamma}u_{\bs{k}}^{\beta\gamma}}{\Omega}\right]\rho_{\bs{k}}^\alpha\rho_{-\bs{k}}^\beta + \frac{N_\alpha^2(\lambda_\alpha+\frac{1}{2})}{\Omega}.
\end{align}

\subsection{Equations that define the RPA Jastrow Pair Potentials}
Variance of $E_L$ can be minimized by setting the $\rho_{\bs{k}}^\alpha\rho_{-\bs{k}}^\beta$ term to zero. Define $\epsilon^\alpha_{\bs{k}}\equiv \lambda_\alpha k^2$
\begin{align}
\frac{v^{\alpha\beta}_{\bs{k}}}{2} - \frac{\epsilon^\alpha_{\bs{k}}+\epsilon^\beta_{\bs{k}}}{2}
u_{\bs{k}}^{\alpha\beta} -
\sum\limits_\gamma \frac{N_\gamma}{\Omega} \epsilon^\gamma_{\bs{k}} u_{\bs{k}}^{\alpha\gamma}u_{\bs{k}}^{\beta\gamma} = 0. \label{eq:mineq}
\end{align}
Equation~(\ref{eq:mineq}) can be solved for each $\bs{k}$ independently. We no longer need the collective coordinates or the label $\bs{k}$. It is now safe to recycle the symbol $\rho_\gamma\equiv\frac{N_\gamma}{\Omega}$ to mean the number density of species $\gamma$. Simplify eq.~(\ref{eq:mineq}) to
\begin{align}
\frac{v_{\alpha\beta}}{2} - \frac{1}{2}(\epsilon_\alpha+\epsilon_\beta)
u_{\alpha\beta} -
\sum\limits_\gamma \rho_\gamma \epsilon_\gamma 
u_{\alpha\gamma}u_{\beta\gamma} = 0 . \label{eq:mineq_nok}
\end{align}

\subsection{Solving for te RPA Jastrow Pair Function}

\textit{One Component}

For a one-component system, eq.~(\ref{eq:mineq_nok}) becomes a quadratic equation of one variable $u_{11}$
\begin{align}
\frac{v_{11}}{2} - \epsilon_1 u_{11} - \rho_1\epsilon_1u_{11}^2 = 0.
\end{align}
The solution is
\begin{align}
2\rho_1 u_{11} = -1 + \sqrt{1+2\rho_1v_{11}/\epsilon_1}, \label{eq:oc-rpa-uk}
\end{align}
which agrees with Gaskell's solution eq.~(\ref{eq:wf-gaskell-rpa-uk}), except $S_0(k)$ is replaced by $1$. Notice, if one uses a different Fourier convention, replacing volume $\Omega$ with number of particles $N$ in eq.~(\ref{eq:Uuk})
\begin{align}
U = \frac{1}{2N} \sum\limits_{\bs{k}\alpha\beta}
\tilde{u}^{\alpha\beta}_{\bs{k}} \left( \rho^\beta_{\bs{k}}\rho^\alpha_{-\bs{k}}-\delta_{\alpha\beta}N_\alpha
\right),
\end{align}
then the density $\rho$ drops from the expression for $\tilde{u}$, e.g., eq.~(8) in Ref.~\cite{Ceperley1978} and eq.~(3) in Ref.~\cite{Ceperley1981}
\begin{align}
2\tilde{u} = -1 + (1+2v_k/e_k).
\end{align}

\textit{Two Components}

Eq.~(\ref{eq:mineq_nok}) becomes a set of 3 coupled quadratic equations
\begin{align}
\left\{\begin{array}{l}
\frac{v_{11}}{2} - \epsilon_1 u_{11} - \rho_1\epsilon_1u_{11}^2 - \rho_2\epsilon_2u_{12}^2 = 0 \\
\frac{v_{12}}{2} - \frac{1}{2}(\epsilon_1+\epsilon_2) u_{12} - \rho_1\epsilon_1u_{11}u_{12} - \rho_2\epsilon_2u_{12}u_{22} = 0 \\
\frac{v_{22}}{2} - \epsilon_2 u_{22} - \rho_1\epsilon_1u_{12}^2 - \rho_2\epsilon_2u_{22}^2 = 0.
\end{array}\right.
\end{align}

%In the case of electron-ion matter, 
Suppose species $2$ has infinite mass $\lambda_2\rightarrow0$, thus no dispersion $\epsilon_2=0$. Then we should ignore the last equation ($\alpha=\beta=2$), which determines $u_{22}$ (when $u_{12}=0$). The remaining equations allow us to solve for the Jastrow pair potentials $u_{11}$ and $u_{12}$
\begin{align}
\left\{\begin{array}{l}
\frac{v_{11}}{2} - \epsilon_1 u_{11} - \rho_1\epsilon_1u_{11}^2 = 0 \\
\frac{v_{12}}{2} - \frac{\epsilon_1}{2} u_{12} - \rho_1\epsilon_1u_{11}u_{12} = 0
\end{array}\right..
\end{align}
The first equation provides the same Jastrow potential as in the one-component case eq.~(\ref{eq:oc-rpa-uk}). The second equation can be used to solve for $u_{12}$
\begin{align}
&(1+2\rho u_{11}) \epsilon_1 u_{12} = v_{12} \nonumber \\
\Rightarrow & u_{12} = \frac{v_{12}/\epsilon_1}{\sqrt{1+2\rho_1v_{11}/\epsilon_1}}.
\end{align}
For completeness, the exact solutions are (by Mathematica)
\begin{align}
2\rho_1u_{11} =& -1+\frac{\epsilon_1(1+a_{11}) \pm \epsilon_2\sqrt{A}}
{
\sqrt{\epsilon_1\epsilon_2}\sqrt{B}
},\\
u_{12} =& \frac{\pm v_{12} }{
\sqrt{\epsilon_1\epsilon_2}\sqrt{B}
},\\
2\rho_2u_{22} =& -1+\frac{\epsilon_2(1+a_{22}) \pm \epsilon_1\sqrt{A}}
{
\sqrt{\epsilon_1\epsilon_2}\sqrt{B}
},
\end{align}
where 
\begin{align}
A =& (1+a_{11})(1+a_{22})-a_{12}^2,\\
B =&\frac{\epsilon_2}{\epsilon_1}(1+a_{22}) + \frac{\epsilon_1}{\epsilon_2}(1+a_{11}) \pm 2\sqrt{A},
\end{align}
with
\begin{align}
\left\{\begin{array}{l}
a_{11}=\frac{2\rho_1v_{11}}{\epsilon_1} \\
a_{12}=\frac{2\sqrt{\rho_1\rho_2}v_{11}}{\sqrt{\epsilon_1\epsilon_2}} \\
a_{22}=\frac{2\rho_2v_{22}}{\epsilon_2} \\
\end{array}\right..
\end{align}

% 2018-08-10_slater-det-pw_thesis
\subsection{Slater Determinant in Plane Wave Basis}

\textit{Based on notes from D. M. Ceperley dated Aug. 1 2018}

When the orbitals are expressed in plane wave basis
\begin{align}
\phi_i(\bs{r}) = \sum\limits_{\bs{k}} c_{i\bs{k}} e^{i\bs{k}\cdot\bs{r}}, \label{eq:pw-orb}
\end{align}
and require the orbitals to be orthonormal.
\begin{align}
\int_\Omega d\bs{r} \phi_i(\bs{r})^*\phi_j(\bs{r}) = \delta_{ij} \Rightarrow 
\Omega \sum\limits_{\bs{k}} c_{i\bs{k}}^*c_{j\bs{k}} = \delta_{ij}. \label{eq:on_orbs}
\end{align}

We can verify that the determinant written in eq~(\ref{eq:det}) is normalized
\begin{align}
\braket{} \equiv& \int d\bs{r}_1\dots d\bs{r}_N \Psi^*~ \Psi \nonumber \\
=& \frac{1}{N!} \sum\limits_{\mathcal{P},\mathcal{P'}} (-1)^{\mathcal{P}} (-1)^{\mathcal{P'}}
\left(
\prod\limits_{l=1}^{N} \int d\bs{r}_l \phi^*_{\mathcal{P}_l}(\bs{r}_l)\phi_{\mathcal{P}_l'}(\bs{r}_l)
\right)\nonumber \\
=& \frac{1}{N!} \sum\limits_{\mathcal{P},\mathcal{P'}} (-1)^{\mathcal{P}} (-1)^{\mathcal{P'}}
\left(
\prod\limits_{l=1}^{N} \delta_{\mathcal{P}_l,\mathcal{P}_l'}
\right) \nonumber \\
=& \frac{1}{N!} \sum\limits_{\mathcal{P}} = 1. \label{eq:det-norm}
\end{align}
The key step in eq~(\ref{eq:det-norm}) is to separate and distribute the many-body integrals into the product.

\subsection{Properties of the Slater Determinant}

Many properties of the slater determinant can be evaluated analytically. Here we focus on reciprocal-space properties accessible by scattering experiments: the momentum distribution $n(\bs{k})$ and the static structure factor $S(\bs{k})$. %They are the Fourier transform of the one-body reduced density matrix (1RDM) and two-body reduced density matrix (2RDM), respectively.

\subsubsection{Momentum Distribution}

The momentum distribution is the Fourier transform of the 1RDM [IE (5.9)]. The 1RDM can be calculated from the many-body wavefunction
\begin{align}
\rho(\bs{x}, \bs{x}') = N\int d\bs{r}_2\dots\bs{r}_N \Psi^*(\bs{x}, \bs{r}_2, \dots) \Psi(\bs{x}', \bs{r}_2, \dots).
\end{align}
Given a Slater determinant wavefunction eq~(\ref{eq:det}), all the $d\bs{r}$ integrals can be done analytically
\begin{align}
\rho(\bs{x}, \bs{x}') =& N\int d\bs{r}_2\dots d\bs{r}_N
\left(
\frac{1}{N!}\sum\limits_{\mathcal{P}, \mathcal{P}'} (-1)^{\mathcal{P}} (-1)^{\mathcal{P}'}
\phi_{\mathcal{P}_1}^*(\bs{x})\phi_{\mathcal{P}_1'}(\bs{x}')
\prod\limits_{l=2}^{N}\phi_{\mathcal{P}_l^*}(\bs{r}_l)\phi_{\mathcal{P}_l'}(\bs{r}_l) 
\right) \nonumber \\
=& \frac{N}{N!} \sum\limits_{\mathcal{P}, \mathcal{P}'} (-1)^{\mathcal{P}} (-1)^{\mathcal{P}'}
\left(
\phi_{\mathcal{P}_1}^*(\bs{x})\phi_{\mathcal{P}_1'}(\bs{x}')
\prod\limits_{l=2}^N \delta_{\mathcal{P}_l,\mathcal{P}_l'}
\right) \nonumber \\
=& \frac{N}{N!} \sum\limits_{\mathcal{P}} \phi_{\mathcal{P}_1}^*(\bs{x})\phi_{\mathcal{P}_1}(\bs{x}') \nonumber \\
=& \sum\limits_{\mathcal{P}_1=1}^N \phi_{\mathcal{P}_1}^*(\bs{x})\phi_{\mathcal{P}_1}(\bs{x}'). \label{eq:det-1rdm}
\end{align}
Notice that the diagonal ($\bs{x}=\bs{x}'$) of the 1RDM is particle density. Given PW orbitals eq~(\ref{eq:pw-orb})
\begin{align}
n(\bs{k}) =& \frac{1}{(2\pi)^3N}\int d\bs{r} d\bs{r}'' e^{-i\bs{k}\cdot\bs{r}''} \rho(\bs{r}, \bs{r}-\bs{r}'') \nonumber \\
= & \frac{1}{(2\pi)^3N} \sum\limits_{i, \bs{g}, \bs{g}'} c_{i\bs{g}}^*c_{i\bs{g}'}
\int d\bs{r} d\bs{r}'' e^{-i\bs{g}\cdot\bs{r}}e^{i\bs{g}'\cdot(\bs{r}-\bs{r}'')} \nonumber \\
=&  \frac{1}{(2\pi)^3N} \sum\limits_{i, \bs{g}, \bs{g}'} c_{i\bs{g}}^*c_{i\bs{g}'} \Omega\delta_{\bs{g}, \bs{g}'}\Omega\delta_{\bs{g}',-\bs{k}} \nonumber \\
=& \frac{\Omega}{(2\pi)^3}\frac{\Omega}{N}\sum_{i=1}^{N} \vert c_{i,-\bs{k}}\vert^2.\label{eq:det-nofk}
\end{align}
Given the current definitions, $\int d\bs{k} n(\bs{k}) = 1$ for an infinite system. In practice, one bins the Fourier coefficient squared of all occupied orbitals at allowed momenta of the supercell.

\subsubsection{Static Structure Factor}

The static structure factor is the density-density correlation in reciprocal space
\begin{align}
S_{\bs{q}} \equiv& \braket{\rho_{\bs{q}}\rho_{-\bs{q}}} \equiv \braket{
(\frac{1}{\sqrt{N}}\sum\limits_{i=1}^N e^{i\bs{q}\cdot\bs{r}_i})
(\frac{1}{\sqrt{N}}\sum\limits_{j=1}^N e^{-i\bs{q}\cdot\bs{r}_j})
} \nonumber \\
=& \frac{1}{N}\sum_{ij}\braket{e^{i\bs{q}\cdot(\bs{r}_i-\bs{r}_j)}} =
1 + \frac{1}{N}\sum_{i\neq j}\braket{e^{i\bs{q}\cdot(\bs{r}_i-\bs{r}_j)}} \nonumber \\
=& 1+(N-1)\braket{e^{i\bs{q}\cdot(\bs{r}_1-\bs{r}_2)}}.
\end{align}
Focus on the many-body integral
\begin{align}
\braket{e^{i\bs{q}\cdot(\bs{r}_1-\bs{r}_2)}} = \dfrac{1}{N!} \sum\limits_{\mathcal{P},\mathcal{P'}} 
(-1)^{\mathcal{P}} (-1)^{\mathcal{P'}} \int d\bs{r}_1\dots d\bs{r}_N
e^{i\bs{q}\cdot(\bs{r}_1-\bs{r}_2)}
\prod\limits_{l=1}^{N} \phi^*_{\mathcal{P}_l}(\bs{r}_l)\phi_{\mathcal{P}_l'}(\bs{r}_l). \label{eq:eiqr1r2}
\end{align}
Similar to eq~(\ref{eq:det-norm}) and eq~(\ref{eq:det-nofk}), $\mathcal{P}_l=\mathcal{P}'_l, \forall l\neq1, 2$. Define $\mathcal{P}_1=i$, $\mathcal{P}_2=j$, then $\mathcal{P}_{1,2}'=i, j$ contributes a positive term, and $\mathcal{P}_{1,2}'=j, i$ contributes a negative term. Thus eq~(\ref{eq:eiqr1r2}) simplifies
\begin{align}
\braket{e^{i\bs{q}\cdot(\bs{r}_1-\bs{r}_2)}} =& \dfrac{1}{N(N-1)} \sum\limits_{i, j}
 \int d\bs{r}_1d\bs{r}_2
e^{i\bs{q}\cdot(\bs{r}_1-\bs{r}_2)} \times \nonumber \\
&\left[
\phi^*_{i}(\bs{r}_1)\phi_{i}(\bs{r}_1)\phi^*_{j}(\bs{r}_2)\phi_{j}(\bs{r}_2) - 
\phi^*_{i}(\bs{r}_1)\phi_{j}(\bs{r}_1)\phi^*_{j}(\bs{r}_2)\phi_{i}(\bs{r}_2)
\right] \nonumber \\
=& \dfrac{1}{N(N-1)} \sum\limits_{i\neq j} \left[
\int d\bs{r}_1 e^{i\bs{q}\cdot\bs{r}_1} \phi^*_{i}(\bs{r}_1)\phi_{i}(\bs{r}_1) \right.
\int d\bs{r}_1 e^{-i\bs{q}\cdot\bs{r}_2} \phi^*_{j}(\bs{r}_2)\phi_{j}(\bs{r}_2) \nonumber \\
& \left. - \int d\bs{r}_1 e^{i\bs{q}\cdot\bs{r}_1} \phi^*_{i}(\bs{r}_1)\phi_{j}(\bs{r}_1)
\int d\bs{r}_1 e^{-i\bs{q}\cdot\bs{r}_2} \phi^*_{j}(\bs{r}_2)\phi_{i}(\bs{r}_2)
\right] \nonumber \\
=& \dfrac{1}{N(N-1)} \sum\limits_{i\neq j} \left[
m_{ii}(\bs{q})m_{jj}(-\bs{q}) - m_{ij}(\bs{q})m_{ji}(-\bs{q})
\right],
\end{align}
where we have defined the matrix of integrals
\begin{align}
m_{ij}(\bs{q}) \equiv \int d\bs{r} \phi_i^*(\bs{r}) \phi_j(\bs{r}) e^{i\bs{q}\cdot\bs{r}}. \label{eq:det-mijq}
\end{align}
Notice $m_{ij}^*(\bs{q}) = m_{ji}(-\bs{q})$, thus
\begin{align}
S_{\bs{q}} =& 1+\frac{1}{N} \sum_{i\neq j} \left[ m_{ii}(\bs{q})m_{jj}^*(\bs{q}) - m_{ij}(\bs{q})m_{ij}^*(\bs{q}) \right] \nonumber \\
=& 1+\frac{1}{N} \left[
\vert \sum_{i} m_{ii}(\bs{q}) \vert^2 -\sum_{i, j} \vert m_{ij}(\bs{q}) \vert^2
\right]. \label{eq:det-sofk}
\end{align}

\subsubsection{Example: Free Fermions}
The ground-state wavefunction of non-interacting fermions is a determinant of plane waves. The first $N$ plane-wave orbitals with the lowest momenta are filled. In case of degeneracy, the wavefunction will have a non-zero net momentum. %If $N$ does not fully fill the outer-most k-shell (there are degenerate unfilled orbitals at the Fermi energy), then the ground state must be an equal superposition of determinants. The determinant expansion should have $\left( \begin{array}{c} N_s\\ N_{left} \end{array} \right)$ terms, where $N_s$ is the number of states in the partially filled k-shell.

Simply stated, the free fermion wavefunction is a determinant eq~(\ref{eq:det}) whose orbitals each have a single Fourier component
\begin{align}
c_{i\bs{k}} = \frac{1}{\sqrt{\Omega}} \delta_{\bs{k},\bs{k}_i}.
\end{align}
We see from eq~(\ref{eq:det-nofk}) that the momentum distribution $n(\bs{k})$ of the free fermions is a step function, which is constant within the Fermi surface and zero outside. As for the static structure factor $S(\bs{k})$, first note that the matrix of integrals $m_{ij}(\bs{q})$ eq~(\ref{eq:det-mijq}) is sparse
\begin{align}
m_{ij}(\bs{q}) = c_{i\bs{k}_i}^* c_{j\bs{k}_j} \Omega\delta_{\bs{q}, \bs{k}_i-\bs{k}_j} = \delta_{\bs{q}, \bs{k}_i-\bs{k}_j}. \label{eq:det-free-mijq}
\end{align}
Plug (\ref{eq:det-free-mijq}) into (\ref{eq:det-sofk}) and
\begin{align}
S_{\bs{q}} = \left\{
\begin{array}{lr}
N & \bs{q}=\bs{0} \\
 1-\frac{1}{N}\sum\limits_{i, j} \vert \delta_{\bs{q}, \bs{k}_i-\bs{k}_j} \vert^2 & \bs{q}\neq\bs{0}
\end{array}\right..
\end{align}
Eq~(\ref{eq:det-free-mijq}) has a simple geometric interpretation. Namely, $m_{ij}(\bs{q})$ is non-zero only if $\bs{q}$ connects two occupied plane wave orbitals. In the thermodynamic limit, the geometric interpretation allows $S(\bs{k})$ to be calculated from a simple integral
\begin{align}
S(\bs{k}\neq\bs{0}) =& 1-\left(\frac{4\pi k_F^3}{3}\right)^{-1}2\int_{q/2}^{k_F} dk \pi(k_F^2-k^2) \nonumber \\
=& \left\{ \begin{array}{lr}
\frac{3}{4}\left(\frac{q}{k_F}\right) - \frac{1}{16} \left(\frac{q}{k_F}\right)^3 & q<2k_F\\
1 & q\ge 2k_F
\end{array}\right..
\end{align}
%\section{Back flow transformation}

\section{Beyond Slater-Jastrow}
One way to systematically obtain more accurate wave functions than Slater-Jastrow is to expand the many-body wave function in a basis of determinants. This configuration interaction (CI) method is routinely used in quantum chemistry to study molecules. The CI method is especially effective when the correlation in the ground state is mostly static in nature, e.g., in benzene molecule where the ground-state wave function is approximately the equal superposition of six non-interacting determinants. This method is less successful in bulk calculations due to difficulties in performing finite-size extrapolation. Specifically, a fixed-length determinant expansion is more accurate for a small system than a large one. Thus, properties such as the total energy are not size-extensive and cannot be extrapolated to the thermodynamic limit using known asymptotic formula.

Another method for systematically improving the many-body wave function is to use the generalized Feynman-Kac formula (eq.~(6.12) in Ref.~\cite{Martin2016}), which is intimately related to the DMC method (eq.~(24.23) in Ref.~\cite{Martin2016})
\begin{align} \label{eq:wf-fk}
\Psi_0 = \dfrac{\Psi_T}{\braket{\Psi_T|\Psi_0}}\left\langle
\exp\left(
-\int_0^\infty dt [E_L(\bs{R}_t;\Psi_T)-E_0]
\right)
\right\rangle_{\Psi_T^2},
\end{align}
where $\braket{}_{\Psi_T^2}$ denotes average over drift-diffusion random walks guided by $\Psi_T$ eq.~(\ref{eq:dmc-dfdt}). $E_L(\bs{R}_t;\Psi_T)$ is the local energy of the trial wave function at walker position $\bs{R}_t$ during the random walk. If $\Psi_T$ is a good approximation for the ground state $\Psi_0$, then imaginary time propagation can be cut off at a small equilibrium ``time'' $\tau$ and eq.~(\ref{eq:wf-fk}) has a compelling physical interpretation eq.~(6.13) in Ref.~\cite{Martin2016}
\begin{align} \label{eq:wf-fk-small-tau}
\ln\Psi_0(\bs{R}_0) \approx \ln\Psi_T(\bs{R}_0) - \tau [
\left\langle E_L\right\rangle_{\Psi_T^2}(\bs{R}_0) - E_0
].
\end{align}
Equation~(\ref{eq:wf-fk-small-tau}) says the correction to the exponent of $\Psi_T$ at $\bs{R}_0$ is proportional to the average of its local energy over a small region around $\bs{R}_0$ having size $O(\sqrt{d \lambda\tau})$ in dN-dimensional space. This observation is quite useful for practitioners of the DMC method, because it implies longer projection time is needed for:
\begin{enumerate}
\item \textit{larger system}, because the space over which $E_L$ needs to be averaged increase with $N$.
\item \textit{heavier particles}, because they have smaller quantumness $\lambda$, thus smaller diffusion constant.
\item \textit{long-range properties}, because the effect of projection is local and expands slowly with projection time.
\end{enumerate}

Finally, the Feynman-Kac formula was shown to be capable of suggesting compact wave function that accurately capture electron correlation starting from a simple initial guess~\cite{Holzmann2003}.
