\section{Methods}
\label{sec:hsolid-methods}

\subsection{Candidate Structure Optimization}
% modified from supp1-geo.tex
We considered three candidates C2/c-24, Cmca-4, Cmca-12 for the molecular phase and one candidate I4$_1$/amd for the atomic phase at $T=0K$ over the pressure range 350 GPa to 700 GPa. The static-lattice structures in the molecular phase are optimized by vdW-DF at constant pressure.
As shown in Table~\ref{tab:hsolid-mol-press-rs}, all three molecular structures optimize to roughly the same density at each pressure.
\begin{table}[h]
% 2020/05/18 notes, extracted from h11/analysis/11a_axes_pos.json
\centering
\begin{tabular}{ccccccc}
\toprule
vdW-DF P(GPa) & 360 & 400 & 440 & 480 & 520 & 560 \\
\midrule
Cmca-4  & 1.303 & 1.283 & 1.265 & 1.250 & 1.235 & 1.222 \\
Cmca-12 & 1.306 & 1.286 & 1.268 & 1.252 & 1.237 & 1.224 \\
C2/c-24 & 1.307 & 1.287 & 1.269 & 1.253 & 1.239 & 1.225 \\
\bottomrule
\end{tabular}
\caption{vdW-DF pressure-density (expressed in $r_s$) relation of relaxed molecular candidate structures.}
\label{tab:hsolid-mol-press-rs}
\end{table}
In contrast, the atomic structure is optimized using DMC at constant volume.

All three molecular structures are monoclinic having $a=b\neq c$, $\alpha=\beta=90^\circ+\eta$, and $\gamma=120^\circ+\delta$. The slight distortions differ for each structure: $\eta=0$, $\delta\approx -0.5^\circ$ for Cmca-4, $\eta=0$, $\delta\approx+3.5^\circ$ for Cmca-12, and $\eta\approx0.1^\circ$, $\delta\approx-0.1^\circ$ for C2/c-24.
The evolution of the lattice parameters as a function of pressure are shown in Fig.~\ref{fig:hsolid-vdw-ca}(a).
Both $a$ and $c$ decrease with increasing pressure.
However, the $c/a$ ratio remains roughly constant at $1.062\pm0.003$ and $1.771\pm0.003$ for Cmca-12 and C2c-24, respectively.
In contrast, the $c/a$ ratio of the Cmca-4 structure decreases from $1.562$ at $350$ GPa to $1.530$ at $560$ GPa linearly with pressure.
Besides having slightly different unit cells, the Cmca-4 structure has only one type of H$_2$ molecule, whereas Cmca-12 and C2/c have two.

The vdW-DF optimized H$_2$ bond lengths of all three molecular structures are shown in Fig.~\ref{fig:hsolid-vdw-ca}(b).
The bond length in Cmca-4 is comparable to its isolated value of $1.4$ Bohr, whereas in C2/c-24 it is 3 to 4\% compressed.
One type of the H$_2$ molecules in Cmca-12 has pressure-sensitive bond length, increasing from $\sim 1.38$ Bohr at 360 GPa to $\sim 1.4$ Bohr at 560 GPa, while the other type has 3\% compressed bond length irrespective of pressure.

\begin{figure}[h]
\centering
\begin{minipage}{0.49\textwidth}
\centering
\includegraphics[width=\linewidth]{h11_mol-ca}
(a) $a$ and $c$ lattice parameters
\end{minipage}
\begin{minipage}{0.49\textwidth}
\centering
\includegraphics[width=\linewidth]{h11_mol-rb}
(a) H$_2$ bond length
\end{minipage}
\caption{vdW-DF optimized molecular candidate structures.}
\label{fig:hsolid-vdw-ca}
\end{figure}

While the vdW-DF optimized structures in Ref.~\cite{McMinis2015} are not published, we can infer from the enthalpy-pressure relations that the same structures have been reproduced in this study. Figure~\ref{fig:dft-opt-geo} shows the enthalpy of each candidate structure relative to C2/c-24 at the vdW-DF static-lattice minimum. The results agree well with those from McMinis et al.~\cite{McMinis2015} where available.
In comparison to predictions by Drummond \textit{et al.}~\cite{Drummond2015}, our C2/c structure is slightly more stable and Cmca-4 slightly less stable.
They used BLYP rather than vdW-DF functional, so differences are expected. In fact, it is encouraging to see two different functionals give similar results (within a few meV/p).

\begin{figure}[h]
\centering
\includegraphics[width=0.6\columnwidth]{010a_dh-vs-p}
\caption{DFT(vdW-DF) static-lattice enthalpy of optimized structures relative to C2/c-24. Thin solid lines are enthalpies of the Cmca-4, Cmca-12 and I4$_1$/amd using our optimized structures. Dashed lines are DFT(BLYP) enthalpies of Cmca-4 and Cmca-12 from Drummond \textit{et al.}~\cite{Drummond2015}. Dash-dot lines are DFT(vdW-DF) enthalpies from McMinis \textit{et al.}~\cite{McMinis2015}.}
\label{fig:dft-opt-geo}
\end{figure}

\subsection{Supercell Construction}
% modified from supp2-static-qmc.tex
To reliably obtain QMC energy values in the thermodynamic limit, we need to tile the optimized primitive cells to sufficiently large supercells so that pair correlation functions are converged.
The remaining finite-size error can be removed using methods to be discussed in Chap.~\ref{chap:fsc}.
The supercells also need to be small enough for dynamic-ion QMC to be practical.
In the end, we chose to perform all QMC calculations using 72-atom simulation cells. Each simulation cell is tiled from the optimized unit cell using a non-diagonal supercell matrix~\cite{Lloyd-Williams2015}, which is optimized to have large distances between minimum images.
A supercell matrix in 3D is a $3\times 3$ matrix of integers that map primitive lattice vectors $\bs{a}, \bs{b}, \bs{c}$ to supercell lattice vectors  $\bs{a}_s, \bs{b}_s, \bs{c}_s$
\begin{align}
\left(\begin{array}{c}
\bs{a}_s \\
\bs{b}_s \\
\bs{c}_s
\end{array}\right) =
\left(\begin{array}{ccc}
S_{11} & S_{12} & S_{13} \\
S_{21} & S_{22} & S_{23} \\
S_{31} & S_{32} & S_{33}
\end{array}\right)
\left(\begin{array}{c}
\bs{a} \\
\bs{b} \\
\bs{c}
\end{array}\right).
\end{align}
Once a supercell is chosen, the crystal structure can be created using the following cropping method. One first tiles the atoms in the primitive cell a large number of times along each dimension, then crop out only the atoms that fall inside the supercell.
The total number of atoms in the supercell should be $\det S$ times larger than that in the primitive cell.

Non-diagonal supercell matrices can be useful for maximizing the minimum image radius, a.k.a. radius of the real-space Wigner-Seitz cell, $R_{WS}$.
As shown in Fig.~\ref{fig:hsolid-square-vs-rhombus}, a rhombus supercell provides a larger $R_{WS}$ than a square having the same area.
This is because the images form a closed-packed lattice with a rhombus supercell.
If the primitive cell is square, then all diagonal supercell matrices result in square supercells. In contract, one can construct a supercell close to a rhombus using a non-diagonal matrix.
\begin{figure}[h]
\centering
\begin{minipage}{0.49\textwidth}
\centering
\includegraphics[width=0.66666\linewidth]{supercell-square}\\
(a) square $a=2$ \AA
\end{minipage}
\begin{minipage}{0.49\textwidth}
\centering
\includegraphics[width=\linewidth]{supercell-rhombus}\\
(b) rhombus $a=2.149$ \AA
\end{minipage}
\caption{Cubic vs. rhombus supercells. The black cell is the supercell. The gray cells are periodic images. The blue line points between nearest-neighbor images, while the red line between second-nearest neighbors. The yellow circle is the inscribed circle in the Wigner-Seitz cell (not shown) of the supercell.}
\label{fig:hsolid-square-vs-rhombus}
\end{figure}

The chosen supercell matrices and their resulting image radii are shown in Table~\ref{tab:hsolid-tmat72} and Fig.~\ref{fig:cell-radius}, respectively.
The inscribing radius of each supercell $R_{sc}$ are also shown in Fig.~\ref{fig:cell-radius} to give a sense of how far each supercell is from being orthorhombic. An orthorhombic cell has $R_{WS} = R_{sc}$.
\begin{table}[h]
\centering
\caption{Optimized 72-atom non-diagonal supercell matrices.}
\label{tab:hsolid-tmat72}
\begin{tabular}{cccc}
\hline\hline
Cmca-4 & Cmca-12 & C2/c-24 & I4$_1$/amd \\
\hline
$\left(\begin{array}{rrr}
-1 &  2 &  1 \\
 2 & -1 &  1 \\
 3 &  3 &  0 \\
\end{array}\right)$ & $\left(\begin{array}{rrr}
 2 &  1 &  -1 \\
 2 &  1 &   1 \\
 -1 &  1 &  0 \\
\end{array}\right)$ & $\left(\begin{array}{rrr}
 2 &  1 &  0 \\
 1 &  2 &  0 \\
 0 &  0 &  1
\end{array}\right)$ & $\left(\begin{array}{rrr}
 2 & -2 &  1 \\
 2 &  3 &  0 \\
-2 &  1 &  1
\end{array}\right)$ \\
\hline\hline
\end{tabular}
\end{table}

\begin{figure}[h]
\centering
\includegraphics[width=0.6\columnwidth]{101a1_cell-radius}
\caption{Supercell radius as a function of density. $r_s$ is the Wigner-Seitz radius, which is determined by the average electron density $\frac{4\pi}{3}r_s=\rho$, where $\rho=N_e/\Omega$, with $\Omega$ the supercell volume. $R_{WS}$ is the radius of the real-space Wigner-Seitz cell of the supercell. $2R_{WS}$ is the minimum distance between periodic images. \label{fig:cell-radius}}
\end{figure}

%\subsection{Geometry Optimization}
% hsolid/11-refine-struct/a-ecut
The molecular structures are optimized in DFT using the vdW-DF functional. We use quantum ESPRESSO v5.3.0 to perform variable-cell geometry optimization at constant pressure. The atomic positions in the optimized unit cell are re-optimized at constant-volume. We use a Troullier-Martins pseudopotential with a core cutoff radius of $r_c=0.5$. The plane-wave cutoff energy is set to 160 Ry. Brillouin zone integration is performed using a shifted Monkhorst-Pack grid with $24^3$, $16^3$, $12^3$ points for the Cmca-4, Cmca-12, and C2/c-24 unit cells, respectively. %The effective number of atoms are 55296, 49152, and 41472, respectively.
Pressure is converged to 0.1 kbar (0.01 GPa). % Forces are converged to

The atomic structure is optimized in DMC. At each density, the $c/a$ parameter determines the I4$_1$/amd-4 ($c/a>1$) crystal structure. To optimize the $c/a$ parameter, we performed 5 DMC calculations at each density. These calculations form a grid in the lattice a-c parameter space as shown in Fig.~\ref{fig:i4-rs-ca}. Please see QMC section for details of the DMC calculations.
\begin{figure}[h]
% 2018-02-01_ani-press
\includegraphics[width=0.8\columnwidth]{70_i4_ca_grid}
\caption{DMC calculations performed to optimize the atomic hydrogen solid structure. Each dot is a structure defined by the lattice parameters a and c. The color of each dot indicates the DMC energy. The gray contour lines mark structures with constant density or c/a ratio. Energy variation is dominated by density change. Energy variation in the c/a direction at fixed density is roughly quadratic around its minimum (black star). The black stars are the optimized geometries.\label{fig:i4-rs-ca}}.
\end{figure}
%\subsection{Supercell}
All quantum Monte Carlo (QMC) calculations are performed using 72-atom simulation cells. Each simulation cell is tiled from the optimized unit cell using a supercell matrix.
\begin{align}
A_s = S A \Rightarrow \left(\begin{array}{c}
\bs{a}_s \\
\bs{b}_s \\
\bs{c}_s
\end{array}\right) = S\left(\begin{array}{c}
\bs{a} \\
\bs{b} \\
\bs{c}
\end{array}\right),
\end{align}
where $\bs{a}$, $\bs{b}$, $\bs{c}$ are the lattice vectors of the unit cell. $\bs{a}_s$, $\bs{b}_s$, $\bs{c}_s$ are the lattice vectors of the simulation cell. $S$ is the supercell matrix. The supercell matrices are chosen to maximize the simulation cell radius. %Ideally, one should maximize the image radius $R_{WS}$ instead.
\begin{table}[h]
\caption{Supercell matrices.}
\begin{tabular}{cccc}
\hline\hline
Cmca-4 & Cmca-12 & C2/c-24 & I4$_1$/amd \\
\hline
$\left(\begin{array}{ccc}
 3 &  3 &  0 \\
-1 &  2 &  1 \\
 2 & -1 &  1
\end{array}\right)$ & $\left(\begin{array}{ccc}
 2 &  1 &  -1 \\
-1 &  1 &  0 \\
 2 &  1 &  1
\end{array}\right)$ & $\left(\begin{array}{ccc}
 2 &  1 &  0 \\
 1 &  2 &  0 \\
 0 &  0 &  1
\end{array}\right)$ & $\left(\begin{array}{ccc}
 2 & -2 &  1 \\
 2 &  3 &  0 \\
-2 &  1 &  1
\end{array}\right)$ \\
\hline\hline
\end{tabular}
\end{table}

In Fig.~\ref{fig:cell-radius}, the image radius (a.k.a. radius of the real-space Wigner-Seitz cell) $R_{WS}$ and simulation cell radius $R_{sc}$ are shown as a function of density.

\begin{figure}[h]
\includegraphics[width=0.8\columnwidth]{101a1_cell-radius}
\caption{Supercell radius as a function of density. $r_s$ is the Wigner-Seitz radius, which is determined by the average electron density $\frac{4\pi}{3}r_s=\rho$, where $\rho=N_e/\Omega$, with $\Omega$ the supercell volume. $R_{WS}$ is the radius of the real-space Wigner-Seitz cell of the supercell. $2R_{WS}$ is the minimum distance between periodic images. \label{fig:cell-radius}}
\end{figure}

\subsection{Wavefunction}

All QMC calculations are performed using the QMCPACK code. We use Slater-Jastrow-Backflow (SJB) wavefunction. The orbitals in the Slater determinant are cusp-corrected DFT orbitals. The vdw-DF functional is used to generate orbitals for the molecular structures, whereas the PBE functional is used for the atomic structure. The orbital generating DFT runs have different settings compared to the geometry optimization runs.

To generate the orbitals, we perform DFT directly in the supercell. All calculations use the bare Coulomb interaction and a plane wave cutoff of 50 Ry. First, we run a self-consistent calculation to converge the charge density on a shifted $8^3$ Monkhorst-Pack grid. Second, we run a non-self-consistent calculation on an unshifted Monkhorst-Pack grid to generate the orbitals needed by all twists. $4^3$ twists are used for the molecular phase, while $6^3$ twists are used for the atomic phase. Finally, we divide each orbital by an electron-ion Jastrow wavefunction to remove the electron-ion cusp from the orbital. This electron-ion Jastrow wavefunction is constructed using Fourier components commensurate with the simulation cell (i.e. on the reciprocal lattice vectors of the simulation cell $\bs{G}_s$)
\begin{align}
J_{ei}(\bs{r}_j; \bs{R}) \propto& \exp\left\{ \text{iFFT}\left[ 
U_{\bs{k}}^{ep}  \left(\sum\limits_{J=1}^{N_p} \frac{e^{-i\bs{k}\cdot\bs{R}_J}}{N_p}  \right)
\right] \right\} \nonumber\\
\propto& \exp\left\{ 
\sum\limits_{\bs{k}\neq\bs{0}}^{\bs{k}\in\bs{G}_s} e^{i\bs{k}\cdot\bs{r}_j}~
U_{\bs{k}}^{ep} 
\left(\sum\limits_{J=1}^{N_p} \frac{e^{-i\bs{k}\cdot\bs{R}_J}}{N_p}  \right)
\right\},\label{eq:rpa-ep-jas}
\end{align}
where $\bs{r}_j$ is any single electron coordinate. $\bs{R}$ contains all ionic coordinates. $N_p$ is the number of protons. iFFT stands for ``inverse fast Fourier transform''. The Jastrow potential $U_{\bs{k}}^{ep}$ in eq.~(\ref{eq:rpa-ep-jas}) is chosen to be the RPA form written by Ceperley and Alder
\begin{align}
2U^{ep}_k = -a_k(1+a_k)^{-1/2},
\end{align}
where $a_k=\frac{12}{r_s^3k^4}$ in Hartree atomic units. $r_s$ is the Wigner-Seitz radius.

When a single-particle orbital is divided by eq.~(\ref{eq:rpa-ep-jas}), the electron-ion cusp is removed from the orbital. We re-introduce the electron-ion cusp in the Jastrow part of the trial wavefunction. 

There are 48 optimizable parameters in our wavefunction. We use short-range B-spline Jastrow pair potentials which are smoothly cut off at $R_{WS}$ (the image radius). There are three Jastrow potentials (uu, ud, ep) between up and up electrons, up and down electrons, electron and proton. We use short-range B-spline back flow transformation functions which are smoothly cut off $R_{WS}$. There are three backflow correlation functions (uu, ud, ep) similar to the Jastrow setup. Each B-spline has 8 optimizable knots.
\begin{comment}
\subsection{QMC Data}

At each density, we perform one VMC and two DMC calculations. Each QMC calculation is labeled by a series index. The VMC calculation is series 0. The first DMC calculation with a relatively large time step is series 1. The second DMC calculation with a relatively small time step is series 2. We post-process the raw results (series 0 - 2) to produce series 3 and 4. We linearly extrapolate the DMC results (series 1, 2) to zero time step and label the results series 3. We linearly extrapolate the VMC and the t=0 DMC results (series 0, 3) to obtain pure-estimator kinetic and potential energies and label them series 4.

Twist-average QMC energies are displayed in the following table. The dUlr column contains the many-body finite size correction which will be described in the next section.

\begin{table}[h]
\small
\begin{tabular}{llrrrllll}
\toprule
         &   &  timestep &  natom &      dUlr & LocalEnergy\_pp &    Variance\_pp &     Kinetic\_pp &    Potential\_pp \\
rs & series &           &        &           &                &                &                &                 \\
\midrule
1.163891 & 0 &    0.0300 &     72 &  0.005586 &    -0.47572(1) &     0.01380(2) &      0.9819(1) &      -1.4576(1) \\
         & 1 &    0.0030 &     72 &  0.005417 &   -0.477138(9) &    0.013719(9) &     0.98177(9) &     -1.45890(8) \\
         & 2 &    0.0015 &     72 &  0.005404 &    -0.47712(1) &     0.01374(1) &      0.9822(1) &    -1.45933(10) \\
         & 3 &    0.0000 &     72 &  0.005391 &    -0.47711(2) &     0.01374(1) &      0.9827(2) &      -1.4598(2) \\
         & 4 &    0.0000 &     72 &  0.005213 &    -0.47711(2) &     0.01374(1) &      0.9834(2) &      -1.4598(2) \\
1.182675 & 0 &    0.0300 &     72 &  0.005455 &    -0.48354(1) &     0.01347(2) &      0.9581(1) &      -1.4416(1) \\
         & 1 &    0.0030 &     72 &  0.005269 &   -0.484951(9) &    0.013408(8) &     0.95828(9) &     -1.44323(9) \\
         & 2 &    0.0015 &     72 &  0.005287 &    -0.48496(1) &     0.01342(1) &      0.9585(1) &      -1.4434(1) \\
         & 3 &    0.0000 &     72 &  0.005305 &    -0.48496(2) &     0.01342(1) &      0.9587(2) &      -1.4436(2) \\
         & 4 &    0.0000 &     72 &  0.005170 &    -0.48496(2) &     0.01342(1) &      0.9592(2) &      -1.4436(2) \\
1.195717 & 0 &    0.0300 &     72 &  0.005373 &   -0.488647(6) &    0.012795(9) &     0.94365(7) &     -1.43230(7) \\
         & 1 &    0.0030 &     72 &  0.005188 &   -0.490019(9) &    0.012688(8) &     0.94357(9) &     -1.43361(9) \\
         & 2 &    0.0015 &     72 &  0.005183 &   -0.490017(9) &    0.012707(9) &     0.94391(9) &     -1.43394(9) \\
         & 3 &    0.0000 &     72 &  0.005178 &    -0.49002(2) &    0.012707(9) &      0.9443(2) &      -1.4343(2) \\
         & 4 &    0.0000 &     72 &  0.004997 &    -0.49002(2) &    0.012707(9) &      0.9449(2) &      -1.4343(2) \\
1.221845 & 0 &    0.0300 &     72 &  0.005135 &   -0.497976(6) &    0.012288(9) &     0.91589(7) &     -1.41386(7) \\
         & 1 &    0.0030 &     72 &  0.004973 &   -0.499383(8) &    0.012232(9) &     0.91531(9) &     -1.41469(9) \\
         & 2 &    0.0015 &     72 &  0.004998 &   -0.499352(8) &    0.012255(9) &     0.91567(9) &     -1.41501(9) \\
         & 3 &    0.0000 &     72 &  0.005023 &    -0.49932(2) &    0.012255(9) &      0.9160(2) &      -1.4153(2) \\
         & 4 &    0.0000 &     72 &  0.004922 &    -0.49932(2) &    0.012255(9) &      0.9162(2) &      -1.4153(2) \\
1.235307 & 0 &    0.0300 &     72 &  0.005067 &   -0.502414(6) &     0.01188(1) &     0.90195(6) &     -1.40436(7) \\
         & 1 &    0.0030 &     72 &  0.004900 &   -0.503809(8) &    0.011814(8) &     0.90131(9) &     -1.40508(9) \\
         & 2 &    0.0015 &     72 &  0.004908 &   -0.503783(9) &    0.011793(9) &     0.90181(9) &     -1.40559(9) \\
         & 3 &    0.0000 &     72 &  0.004915 &    -0.50376(2) &    0.011793(9) &      0.9023(2) &      -1.4061(2) \\
         & 4 &    0.0000 &     72 &  0.004776 &    -0.50376(2) &    0.011793(9) &      0.9027(2) &      -1.4061(2) \\
1.249707 & 0 &    0.0300 &     72 &  0.004887 &   -0.506825(6) &     0.01274(2) &     0.88739(7) &     -1.39422(7) \\
         & 1 &    0.0030 &     72 &  0.004748 &   -0.508268(9) &    0.012682(9) &      0.8869(1) &     -1.39525(9) \\
         & 2 &    0.0015 &     72 &  0.004765 &   -0.508273(9) &    0.012696(9) &     0.88747(9) &     -1.39574(9) \\
         & 3 &    0.0000 &     72 &  0.004783 &    -0.50828(2) &    0.012696(9) &      0.8880(2) &      -1.3962(2) \\
         & 4 &    0.0000 &     72 &  0.004690 &    -0.50828(2) &    0.012696(9) &      0.8886(2) &      -1.3962(2) \\
1.265425 & 0 &    0.0300 &     72 &  0.004870 &   -0.511407(6) &     0.01158(1) &     0.87289(6) &     -1.38430(7) \\
         & 1 &    0.0030 &     72 &  0.004684 &   -0.512853(8) &    0.011519(8) &     0.87292(9) &     -1.38577(9) \\
         & 2 &    0.0015 &     72 &  0.004697 &   -0.512872(8) &    0.011541(8) &     0.87333(9) &     -1.38621(9) \\
         & 3 &    0.0000 &     72 &  0.004710 &    -0.51289(2) &    0.011541(8) &      0.8738(2) &      -1.3866(2) \\
         & 4 &    0.0000 &     72 &  0.004565 &    -0.51289(2) &    0.011541(8) &      0.8746(2) &      -1.3866(2) \\
1.283017 & 0 &    0.0300 &     72 &  0.004733 &   -0.516220(6) &    0.011305(9) &     0.85757(7) &     -1.37379(7) \\
         & 1 &    0.0030 &     72 &  0.004574 &   -0.517656(8) &    0.011240(9) &     0.85753(9) &     -1.37518(9) \\
         & 2 &    0.0015 &     72 &  0.004575 &   -0.517673(9) &    0.011251(8) &     0.85793(9) &     -1.37560(9) \\
         & 3 &    0.0000 &     72 &  0.004575 &    -0.51769(2) &    0.011251(8) &      0.8583(2) &      -1.3760(2) \\
         & 4 &    0.0000 &     72 &  0.004430 &    -0.51769(2) &    0.011251(8) &      0.8591(2) &      -1.3760(2) \\
1.302685 & 0 &    0.0300 &     72 &  0.004606 &   -0.521222(6) &    0.010863(8) &     0.84146(7) &     -1.36269(7) \\
         & 1 &    0.0030 &     72 &  0.004447 &   -0.522642(8) &    0.010795(7) &     0.84153(8) &     -1.36415(8) \\
         & 2 &    0.0015 &     72 &  0.004452 &   -0.522665(8) &    0.010810(8) &     0.84172(8) &     -1.36439(8) \\
         & 3 &    0.0000 &     72 &  0.004457 &    -0.52269(2) &    0.010810(8) &      0.8419(2) &      -1.3646(2) \\
         & 4 &    0.0000 &     72 &  0.004322 &    -0.52269(2) &    0.010810(8) &      0.8424(2) &      -1.3646(2) \\
\bottomrule
\end{tabular}

\caption{Cmca-4}
\end{table}

\begin{table}[h]
\small
\begin{tabular}{llrrrllll}
\toprule
         &   &  timestep &  natom &      dUlr & LocalEnergy\_pp &     Variance\_pp &     Kinetic\_pp &   Potential\_pp \\
rs & series &           &        &           &                &                 &                &                \\
\midrule
1.223839 & 0 &    0.0300 &     72 &  0.005398 &   -0.498582(6) &     0.013259(9) &     0.91712(7) &    -1.41570(7) \\
         & 1 &    0.0030 &     72 &  0.005121 &   -0.500194(9) &     0.013059(9) &     0.91759(9) &    -1.41781(9) \\
         & 2 &    0.0015 &     72 &  0.005123 &   -0.500189(9) &     0.013063(9) &     0.91791(9) &    -1.41810(9) \\
         & 3 &    0.0000 &     72 &  0.005125 &    -0.50018(2) &     0.013063(9) &      0.9182(2) &     -1.4184(2) \\
         & 4 &    0.0000 &     72 &  0.004868 &    -0.50018(2) &     0.013063(9) &      0.9193(2) &     -1.4184(2) \\
1.251971 & 0 &    0.0300 &     72 &  0.005113 &   -0.507671(7) &     0.013093(9) &     0.89092(7) &    -1.39859(7) \\
         & 1 &    0.0030 &     72 &  0.004888 &   -0.509262(8) &     0.012973(9) &     0.89079(8) &    -1.40004(8) \\
         & 2 &    0.0015 &     72 &  0.004888 &   -0.509253(8) &     0.012967(9) &     0.89147(9) &    -1.40072(9) \\
         & 3 &    0.0000 &     72 &  0.004888 &    -0.50924(2) &     0.012967(9) &      0.8922(2) &     -1.4014(2) \\
         & 4 &    0.0000 &     72 &  0.004676 &    -0.50924(2) &     0.012967(9) &      0.8934(2) &     -1.4014(2) \\
1.268116 & 0 &    0.0300 &     72 &  0.005012 &   -0.512485(6) &     0.011951(8) &     0.87796(7) &    -1.39044(7) \\
         & 1 &    0.0030 &     72 &  0.004777 &   -0.514038(9) &     0.011822(9) &     0.87737(8) &    -1.39142(8) \\
         & 2 &    0.0015 &     72 &  0.004761 &   -0.514071(9) &     0.011820(9) &     0.87791(9) &    -1.39197(9) \\
         & 3 &    0.0000 &     72 &  0.004745 &    -0.51410(2) &     0.011820(9) &      0.8785(2) &     -1.3925(2) \\
         & 4 &    0.0000 &     72 &  0.004490 &    -0.51410(2) &     0.011820(9) &      0.8789(2) &     -1.3925(2) \\
1.286021 & 0 &    0.0300 &     72 &  0.004894 &    -0.51743(1) &      0.01155(1) &      0.8626(1) &     -1.3800(1) \\
         & 1 &    0.0030 &     72 &  0.004676 &   -0.518992(8) &     0.011411(9) &     0.86249(9) &    -1.38150(9) \\
         & 2 &    0.0015 &     72 &  0.004645 &    -0.51902(1) &    0.011398(10) &      0.8624(1) &     -1.3814(1) \\
         & 3 &    0.0000 &     72 &  0.004614 &    -0.51905(2) &    0.011398(10) &      0.8623(2) &     -1.3813(2) \\
         & 4 &    0.0000 &     72 &  0.004351 &    -0.51905(2) &    0.011398(10) &      0.8620(2) &     -1.3813(2) \\
1.306029 & 0 &    0.0300 &     72 &  0.004704 &   -0.522568(6) &      0.01157(2) &     0.84694(7) &    -1.36950(7) \\
         & 1 &    0.0030 &     72 &  0.004498 &   -0.524120(9) &     0.011457(9) &     0.84641(8) &    -1.37052(8) \\
         & 2 &    0.0015 &     72 &  0.004515 &   -0.524133(8) &     0.011450(8) &     0.84684(9) &    -1.37098(9) \\
         & 3 &    0.0000 &     72 &  0.004531 &    -0.52415(2) &     0.011450(8) &      0.8473(2) &     -1.3714(2) \\
         & 4 &    0.0000 &     72 &  0.004370 &    -0.52415(2) &     0.011450(8) &      0.8476(2) &     -1.3714(2) \\
\bottomrule
\end{tabular}

\caption{Cmca-12}
\end{table}

\begin{table}[h]
\small
\begin{tabular}{llrrrllll}
\toprule
         &   &  timestep &  natom &      dUlr & LocalEnergy\_pp &     Variance\_pp &     Kinetic\_pp &   Potential\_pp \\
rs & series &           &        &           &                &                 &                &                \\
\midrule
1.184674 & 0 &    0.0300 &     72 &  0.005670 &    -0.48422(1) &      0.01254(2) &      0.9616(1) &     -1.4459(1) \\
         & 1 &    0.0030 &     72 &  0.005413 &   -0.485676(9) &    0.012399(10) &     0.96117(9) &    -1.44684(9) \\
         & 2 &    0.0015 &     72 &  0.005411 &    -0.48568(1) &    0.012389(10) &      0.9616(1) &     -1.4473(1) \\
         & 3 &    0.0000 &     72 &  0.005409 &    -0.48569(2) &    0.012389(10) &      0.9620(2) &     -1.4477(2) \\
         & 4 &    0.0000 &     72 &  0.005167 &    -0.48569(2) &    0.012389(10) &      0.9624(2) &     -1.4477(2) \\
1.198103 & 0 &    0.0300 &     72 &  0.005490 &    -0.48946(1) &      0.01307(2) &      0.9490(1) &     -1.4384(1) \\
         & 1 &    0.0030 &     72 &  0.005271 &   -0.490925(8) &     0.012934(9) &     0.94809(9) &    -1.43904(8) \\
         & 2 &    0.0015 &     72 &  0.005273 &    -0.49093(1) &      0.01295(1) &      0.9487(1) &     -1.4396(1) \\
         & 3 &    0.0000 &     72 &  0.005275 &    -0.49093(2) &      0.01295(1) &      0.9493(2) &     -1.4402(2) \\
         & 4 &    0.0000 &     72 &  0.005080 &    -0.49093(2) &      0.01295(1) &      0.9496(2) &     -1.4402(2) \\
1.225107 & 0 &    0.0300 &     72 &  0.005412 &   -0.499215(7) &     0.011982(8) &     0.92039(7) &    -1.41960(7) \\
         & 1 &    0.0030 &     72 &  0.005135 &   -0.500680(9) &     0.011825(8) &     0.91983(9) &    -1.42051(9) \\
         & 2 &    0.0015 &     72 &  0.005138 &   -0.500694(9) &     0.011830(8) &     0.92029(9) &    -1.42099(9) \\
         & 3 &    0.0000 &     72 &  0.005140 &    -0.50071(2) &     0.011830(8) &      0.9207(2) &     -1.4215(2) \\
         & 4 &    0.0000 &     72 &  0.004886 &    -0.50071(2) &     0.011830(8) &      0.9211(2) &     -1.4215(2) \\
1.253286 & 0 &    0.0300 &     72 &  0.005220 &   -0.508316(6) &      0.01170(1) &     0.89260(7) &    -1.40092(7) \\
         & 1 &    0.0030 &     72 &  0.004942 &   -0.509765(9) &      0.01152(1) &     0.89257(8) &    -1.40232(8) \\
         & 2 &    0.0015 &     72 &  0.004929 &   -0.509798(8) &     0.011498(8) &      0.8929(1) &    -1.40282(9) \\
         & 3 &    0.0000 &     72 &  0.004917 &    -0.50983(2) &     0.011498(8) &      0.8933(2) &     -1.4033(2) \\
         & 4 &    0.0000 &     72 &  0.004634 &    -0.50983(2) &     0.011498(8) &      0.8940(2) &     -1.4033(2) \\
1.269426 & 0 &    0.0300 &     72 &  0.005052 &   -0.513065(6) &     0.011097(9) &     0.87708(7) &    -1.39014(7) \\
         & 1 &    0.0030 &     72 &  0.004806 &   -0.514529(8) &     0.010948(8) &     0.87756(9) &    -1.39209(9) \\
         & 2 &    0.0015 &     72 &  0.004799 &   -0.514517(8) &     0.010939(8) &     0.87786(9) &    -1.39238(9) \\
         & 3 &    0.0000 &     72 &  0.004792 &    -0.51451(2) &     0.010939(8) &      0.8782(2) &     -1.3927(2) \\
         & 4 &    0.0000 &     72 &  0.004549 &    -0.51451(2) &     0.010939(8) &      0.8792(2) &     -1.3927(2) \\
1.287311 & 0 &    0.0300 &     72 &  0.004946 &   -0.518053(6) &     0.010534(8) &     0.86353(7) &    -1.38158(7) \\
         & 1 &    0.0030 &     72 &  0.004690 &   -0.519484(9) &     0.010371(8) &     0.86319(9) &    -1.38267(9) \\
         & 2 &    0.0015 &     72 &  0.004683 &   -0.519499(9) &     0.010369(7) &     0.86374(8) &    -1.38325(9) \\
         & 3 &    0.0000 &     72 &  0.004675 &    -0.51951(2) &     0.010369(7) &      0.8643(2) &     -1.3838(2) \\
         & 4 &    0.0000 &     72 &  0.004423 &    -0.51951(2) &     0.010369(7) &      0.8651(2) &     -1.3838(2) \\
1.307314 & 0 &    0.0300 &     72 &  0.004801 &    -0.52318(1) &      0.01054(2) &      0.8471(1) &     -1.3703(1) \\
         & 1 &    0.0030 &     72 &  0.004552 &   -0.524617(8) &     0.010398(8) &     0.84732(9) &    -1.37196(9) \\
         & 2 &    0.0015 &     72 &  0.004552 &    -0.52464(1) &      0.01039(1) &      0.8478(1) &     -1.3724(1) \\
         & 3 &    0.0000 &     72 &  0.004552 &    -0.52466(2) &      0.01039(1) &      0.8483(2) &     -1.3729(2) \\
         & 4 &    0.0000 &     72 &  0.004322 &    -0.52466(2) &      0.01039(1) &      0.8494(2) &     -1.3729(2) \\
\bottomrule
\end{tabular}

\caption{C2/c-24}
\end{table}

\begin{table}[h]
\small
\begin{tabular}{llrrrllll}
\toprule
     &   &  timestep &  natom &      dUlr &  LocalEnergy\_pp &     Variance\_pp &      Kinetic\_pp &   Potential\_pp \\
rs & series &           &        &           &                 &                 &                 &                \\
\midrule
1.15 & 0 &    0.0300 &     72 &  0.006282 &    -0.468339(9) &      0.02039(2) &      0.97451(7) &    -1.44285(7) \\
     & 1 &    0.0030 &     72 &  0.005793 &     -0.47074(1) &      0.01978(1) &      0.97496(9) &    -1.44572(9) \\
     & 2 &    0.0015 &     72 &  0.005797 &     -0.47073(1) &      0.01980(1) &      0.97534(9) &    -1.44607(9) \\
     & 3 &    0.0000 &     72 &  0.005800 &     -0.47072(3) &      0.01980(1) &       0.9757(2) &     -1.4464(2) \\
     & 4 &    0.0000 &     72 &  0.005371 &     -0.47072(3) &      0.01980(1) &       0.9769(2) &     -1.4464(2) \\
1.17 & 0 &    0.0300 &     72 &  0.006168 &    -0.476700(9) &      0.02044(2) &      0.94912(7) &    -1.42582(8) \\
     & 1 &    0.0030 &     72 &  0.005678 &    -0.479108(9) &      0.01981(1) &      0.94928(7) &    -1.42839(7) \\
     & 2 &    0.0015 &     72 &  0.005659 &    -0.479149(9) &      0.01981(1) &      0.94962(7) &    -1.42876(7) \\
     & 3 &    0.0000 &     72 &  0.005639 &     -0.47919(2) &      0.01981(1) &       0.9499(2) &     -1.4291(2) \\
     & 4 &    0.0000 &     72 &  0.005158 &     -0.47919(2) &      0.01981(1) &       0.9508(2) &     -1.4291(2) \\
1.19 & 0 &    0.0300 &     72 &  0.005965 &   -0.484386(10) &      0.02117(2) &      0.92299(7) &    -1.40738(7) \\
     & 1 &    0.0030 &     72 &  0.005520 &     -0.48679(1) &      0.02062(2) &      0.92317(9) &    -1.40994(9) \\
     & 2 &    0.0015 &     72 &  0.005508 &     -0.48683(1) &      0.02062(2) &      0.92352(9) &    -1.41035(9) \\
     & 3 &    0.0000 &     72 &  0.005497 &     -0.48687(3) &      0.02062(2) &       0.9239(2) &     -1.4108(2) \\
     & 4 &    0.0000 &     72 &  0.005077 &     -0.48687(3) &      0.02062(2) &       0.9248(2) &     -1.4108(2) \\
1.21 & 0 &    0.0300 &     72 &  0.005837 &     -0.49145(1) &       0.0197(2) &       0.9000(1) &     -1.3914(1) \\
     & 1 &    0.0030 &     72 &  0.005390 &   -0.493853(10) &      0.01908(1) &      0.89966(7) &    -1.39351(7) \\
     & 2 &    0.0015 &     72 &  0.005394 &   -0.493893(10) &      0.01908(1) &      0.89987(7) &    -1.39376(7) \\
     & 3 &    0.0000 &     72 &  0.005398 &     -0.49393(2) &      0.01908(1) &       0.9001(2) &     -1.3940(2) \\
     & 4 &    0.0000 &     72 &  0.005002 &     -0.49393(2) &      0.01908(1) &       0.9002(2) &     -1.3940(2) \\
1.23 & 0 &    0.0300 &     72 &  0.005709 &     -0.49788(1) &      0.01952(6) &       0.8772(1) &     -1.3751(1) \\
     & 1 &    0.0030 &     72 &  0.005260 &    -0.500351(9) &      0.01906(1) &      0.87668(7) &    -1.37703(7) \\
     & 2 &    0.0015 &     72 &  0.005243 &    -0.500363(9) &      0.01905(1) &      0.87711(7) &    -1.37748(7) \\
     & 3 &    0.0000 &     72 &  0.005226 &     -0.50038(2) &      0.01905(1) &       0.8775(2) &     -1.3779(2) \\
     & 4 &    0.0000 &     72 &  0.004781 &     -0.50038(2) &      0.01905(1) &       0.8779(2) &     -1.3779(2) \\
1.25 & 0 &    0.0300 &     72 &  0.005701 &     -0.50380(1) &      0.01725(2) &       0.8546(1) &     -1.3584(1) \\
     & 1 &    0.0030 &     72 &  0.005192 &    -0.506311(9) &    0.016644(10) &      0.85476(7) &    -1.36107(7) \\
     & 2 &    0.0015 &     72 &  0.005195 &    -0.506295(9) &    0.016635(10) &      0.85502(7) &    -1.36131(7) \\
     & 3 &    0.0000 &     72 &  0.005199 &     -0.50628(2) &    0.016635(10) &       0.8553(2) &     -1.3615(2) \\
     & 4 &    0.0000 &     72 &  0.004741 &     -0.50628(2) &    0.016635(10) &       0.8560(2) &     -1.3615(2) \\
1.27 & 0 &    0.0300 &     72 &  0.005504 &     -0.50918(1) &      0.01858(2) &       0.8365(1) &     -1.3457(1) \\
     & 1 &    0.0030 &     72 &  0.005037 &    -0.511700(9) &      0.01807(1) &      0.83566(7) &    -1.34736(7) \\
     & 2 &    0.0015 &     72 &  0.005031 &   -0.511705(10) &      0.01808(1) &      0.83584(7) &    -1.34755(7) \\
     & 3 &    0.0000 &     72 &  0.005025 &     -0.51171(2) &      0.01808(1) &       0.8360(2) &     -1.3477(2) \\
     & 4 &    0.0000 &     72 &  0.004588 &     -0.51171(2) &      0.01808(1) &       0.8355(2) &     -1.3477(2) \\
1.29 & 0 &    0.0300 &     72 &  0.005491 &     -0.51409(1) &      0.01634(2) &     0.81539(10) &     -1.3295(1) \\
     & 1 &    0.0030 &     72 &  0.004980 &   -0.516579(10) &      0.01577(1) &      0.81540(7) &    -1.33198(7) \\
     & 2 &    0.0015 &     72 &  0.004968 &    -0.516601(9) &      0.01579(1) &      0.81556(7) &    -1.33216(7) \\
     & 3 &    0.0000 &     72 &  0.004955 &     -0.51662(2) &      0.01579(1) &       0.8157(2) &     -1.3323(2) \\
     & 4 &    0.0000 &     72 &  0.004467 &     -0.51662(2) &      0.01579(1) &       0.8160(2) &     -1.3323(2) \\
1.31 & 0 &    0.0300 &     72 &  0.005418 &     -0.51851(1) &      0.01569(2) &       0.7983(1) &     -1.3168(1) \\
     & 1 &    0.0030 &     72 &  0.004889 &    -0.521027(9) &    0.015035(10) &      0.79760(7) &    -1.31862(7) \\
     & 2 &    0.0015 &     72 &  0.004880 &    -0.521046(9) &     0.015035(9) &      0.79801(7) &    -1.31906(7) \\
     & 3 &    0.0000 &     72 &  0.004871 &     -0.52106(2) &     0.015035(9) &       0.7984(2) &     -1.3195(2) \\
     & 4 &    0.0000 &     72 &  0.004371 &     -0.52106(2) &     0.015035(9) &       0.7985(2) &     -1.3195(2) \\
\bottomrule
\end{tabular}

\caption{I4$_1$/amd}
\end{table}

\subsection{Many-body Finite Size Correction}

We use the fluctuating structure factor $\delta S(\bs{k})\equiv \braket{(\rho_{\bs{k}}-\bar{\rho}_{\bs{k}})(\rho_{-\bs{k}}-\bar{\rho}_{-\bs{k}})}$ to calculate many-body finite size correction (FSC) to the potential energy. The integrand is cut off using optimized long-range Coulomb potential in reciprocal space $v_k^{lr}$
\begin{align}
\Delta V^{lr} = \left[\int - \sum\right] \frac{1}{2}v^{lr}_k \delta S(\bs{k}).
\end{align}
Total energy FSC ($\Delta E$) is a sum of kinetic ($\Delta T$) and potential ($\Delta V$) corrections regardless of whether mixed-estimator (m) or pure-estimator (p) is used
\begin{align}
\Delta E = \Delta T_m + \Delta V_m = \Delta T_p + \Delta V_p.
\end{align}
Without long-range wavefunction components, the mixed-estimator kinetic FSC is approximately zero ($\Delta T_m \approx 0$). Therefore
\begin{align}
\left\{\begin{array}{l}
\Delta E \approx \Delta V_m \\
\Delta T_p \approx \Delta V_m - \Delta V_p
\end{array}\right..
\end{align}
FSC of the Virial pressure ($\Delta P$) is then
\begin{align}
\left\{\begin{array}{l}
\Delta P_m = (2\Delta T_m + \Delta V_m)/(3\Omega) \approx (\Delta V_m)/(3\Omega) \\
\Delta P_p = (2\Delta T_p + \Delta V_p)/(3\Omega) \approx (2\Delta V_m-\Delta V_p)/(3\Omega)
\end{array}\right., \label{eq:pm-pp-fsc}
\end{align}
where $\Omega$ is volume. Regardless of whether mix-estimator or pure-estimator value is used for the Virial pressure, the FSC DMC enthalpy-pressure data agree well with equation of state derived from the total energy (Fig.~\ref{fig:si-static-hp}).

\subsection{Finite Size Corrected Data}

\end{comment}

\begin{comment}
Finally, we divide each orbital by an electron-ion Jastrow constructed on the reciprocal lattice vectors of the simulation cell $\bs{G}_s$
\begin{align}
J_{ei}(\bs{r}; \bs{R}) = \exp\left\{ -\frac{1}{2\Omega} \sum\limits_{\bs{k}\neq\bs{0}}^{\bs{k}\in\bs{G}_s}
U_{\bs{k}}^{ei} 
\left(\frac{1}{N}\sum\limits_{j=1}^{N_e} e^{i\bs{k}\cdot\bs{r}_j} \right)
\left(\frac{1}{N}\sum\limits_{J=1}^{N_p} e^{-i\bs{k}\cdot\bs{R}_J}  \right)
\right\},\label{eq:rpa-ep-jas}
\end{align}
\begin{align}
J_{ei}(\bs{r}_j; \bs{R}) =& \exp\left\{ \text{iFFT}\left[ 
U_{\bs{k}}^{ep}  \left(\sum\limits_{J=1}^{N_p} \frac{e^{-i\bs{k}\cdot\bs{R}_J}}{N_p}  \right)
\right] \right\}\\
=& \exp\left\{ 
-\frac{(2\pi)^3}{\Omega N_k} \sum\limits_{\bs{k}\neq\bs{0}}^{\bs{k}\in\bs{G}_s} e^{i\bs{k}\cdot\bs{r}_j}~
U_{\bs{k}}^{ep} 
\left(\sum\limits_{J=1}^{N_p} \frac{e^{-i\bs{k}\cdot\bs{R}_J}}{N_p}  \right)
\right\},\label{eq:rpa-ep-jas}
\end{align}
where $\Omega$ is the simulation cell volume. $\bs{r}$ contain all electronic coordinates, $\bs{R}$ contain all ionic coordinates. $N_e$ is the number of electrons, while $N_p$ is the number of protons.
The Jastrow potential $U_{\bs{k}}^{ei}$ in eq.~(\ref{eq:rpa-ep-jas}) is chosen to be the RPA form written by Ceperley and Alder
\begin{align}
2U^{ei}_k = -a_k(1+a_k)^{-1/2},
\end{align}
where $a_k=\frac{12}{r_s^3k^4}$ in Hartree atomic units.
\end{comment}

\subsection{Electronic Wavefunction Optimization}


\section{Results}
\label{sec:hsolid-results}
\subsection{Energy vs. Volume}
\begin{figure}[h]
\begin{minipage}{0.48\textwidth}
\includegraphics[width=0.8\columnwidth]{101sd_ev-s3}
%\caption{Energy v.s. volume.\label{fig:si-static-ev}}
\end{minipage}
\begin{minipage}{0.48\textwidth}
\includegraphics[width=0.8\columnwidth]{101se_hp}
%\caption{Enthalpy v.s. pressure.\label{fig:si-static-hp}}
\end{minipage}
\caption{Static-lattice DMC equation-of-state (EOS) relative to Drummond reference $E(\nu) = \frac{2.14020118}{\nu^2} + \frac{0.60521235}{\nu} - 0.6073132$, where $\nu$ is volume per proton in bohr$^3$ and $E(\nu)$ is in Hartree atomic unit. Relative energies are shown in meV per proton (meV/p). Each solid line is obtained using a fitted energy-volume EOS. The EOS is obtained by fitting the finite-size corrected (FSC) total energy as a quadratic function of inverse volume. The markers are finite-size corrected simulation data without performing a fit. \label{fig:static-qmc-vs-drummond}}
\end{figure}

\subsection{Enthalpy vs. Pressure}
In Fig.~\ref{fig:static-enthalpy-vs-pressure}, enthalpies of the candidate structures are plotted relative the my C2/c reference EOS (Yang SJB N72 dskcorr). The EOSs are obtained by fitting the total energy to a quadratic function of inverse volume. The solid lines with symbols are my results. The dot-dashed lines are McMinis' relative enthalpies extracted from Fig. 1 of ref.~\cite{McMinis2015}. The dashed line is McMinis' atomic structure enthalpy shifted by 48 meV/p (1.8 mha/p) to match my energies. The dotted line is Azadi's atomic structure enthalpy.

\begin{figure}[h]
\includegraphics[width=0.8\textwidth]{87b_refit}
\caption{Enthalpy v.s. pressure using Yang SJB N72 dskcorr C2/c as reference.\label{fig:static-enthalpy-vs-pressure}}
\end{figure}
