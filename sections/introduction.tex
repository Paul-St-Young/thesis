\chapter{Introduction}

\section{The Electron-Ion Problem}
The ultimate goal of this thesis is the accurate simulation of a many-body system of charged particles in the non-relativistic limit.
This goal was not achieved, but some progress has been made.
For the remainder of this thesis, the ground truth is assumed to be established by the exact solution of the Schr\"odinger equation~(\ref{eq:intro-schro}) for the electron-ion wave function $\Psi(\bs{R},\RI)$ of $N$ electrons and $N_I$ ions
\begin{align} \label{eq:intro-schro}
\hat{H} \Psi(\bs{R},\RI) = i\hbar\frac{d}{dt}\Psi(\bs{R},\RI),
\end{align}
where the electron-ion hamiltonian consists of non-relativistic kinetic energies and Coulomb interactions
\begin{align} \label{eq:intro-ei-ham}
\hat{H} =
-\sum\limits_{I=1}^{N_I} \frac{\hbar^2}{2m_I} \nabla^2_I
-\sum\limits_{i=1}^N \frac{\hbar^2}{2m_i}\nabla^2_i
-\sum\limits_{I=1}^{N_I}\sum\limits_{i=1}^N
\frac{Z_Ik_ee^2}{\vert\ri-\bs{r}_i\vert}
+\frac{1}{2}\sum\limits_{i=1}^N\sum\limits_{j=1,j\neq i}^{N} \frac{k_ee^2}{\vert\bs{r}_i-\bs{r}_j\vert}
+\frac{1}{2}\sum\limits_{I=1}^{N_I}\sum\limits_{J=1,J\neq I}^{N_I} \frac{k_ee^2}{\vert\ri-\rj\vert}.
\end{align}
The lower-case $i,j$ and upper-case $I,J$ refer to the electrons and ions, respectively. The lower-case $\bs{r}_i$ labels a single electron position, whereas the upper-case $\bs{R}$ denotes the positions of all electrons $\bs{R}\equiv\{\bs{r}_i\}$. $\ri$ and $\RI$ play analogous roles for the ions. $Z_I$ is the atomic number of ion $I$. If any eigenstate of $\hat{H}$ can be constructed to arbitrary precision in a reasonable amount of time, which grows as a polynomial as the number of particles, then the many-body electron-ion problem can be declared solved. Unfortunately, even state-of-the-art methods struggle with just the ground state~\cite{Azadi2014,McMinis2015,Drummond2015}.

For equilibrium properties at high temperature, progress can be made by considering the Bloch equation~(\ref{eq:intro-bloch}) for the thermal density matrix $\rho\equiv\sum\limits_i e^{-\beta E_i} \ket{\Psi_i}\bra{\Psi_i}$
\begin{align} \label{eq:intro-bloch}
\hat{H} \rho = -\dfrac{d}{d\beta}\rho,
\end{align}
which results from the Schr\"odinger eq.~(\ref{eq:intro-schro}) after a rotation from real to imaginary time $\tau = it$, which is also the inverse temperature $\tau/\hbar=\beta\equiv1/(k_BT)$.
The partition function is a trace of the thermal density matrix and reduces to the classical Boltzmann distribution at high temperature
\begin{align}
Z=\Tr{\rho} \underset{\beta\rightarrow 0}{\Rightarrow} Z \propto e^{-\beta V}.
\end{align}
Further, low-temperature properties of boltzmannons and bosons can be exactly and efficiently calculated using the path integral method (Sec.\ref{sec:method-pimc}). The exact method is no longer practical when fermions, e.g., electrons, are involved. Nevertheless, impressive results have been obtained for hydrogen when only the ground electronic state is considered~\cite{Pierleoni2016b,Celliers2018}.
A complete treatment of the full electron-ion hamiltonian eq.~(\ref{eq:intro-ei-ham}) is rarely attempted~\cite{Ceperley1981,Natoli1995}.

The solution of the electron-ion problem would be an important milestone in computational condensed matter, because it is considered a quantitatively accurate model for the vast majority of solids and liquids in condensed matter experiments. Further, theoretically, it is a natural extension of the jellium model to multi-component system and provides a firm foundation upon which relativistic effects can be included, for example via perturbation. Finally, the laplacian in the non-relativistic kinetic energy operator can be interpreted as a generator of diffusion in imaginary time. This makes it more straightforward to develop intuitive understanding of the quantum kinetic energy as well as to deploy powerful computational techniques such as diffusion Monte Carlo (Sec.~\ref{sec:method-dmc}).

\subsection{The Born-Oppenheimer Approximation}

Suppose all eigenstates of the electronic hamiltonian $\{\psi_k\}$ are available at any ion configuration $\bs{R}_I$
\begin{align}
\ham (\bs{R};\RI) \psi_k(\bs{R};\RI) = E_k(\RI) \psi_k(\bs{R};\RI).
\end{align}
Then, one can attempt to separate the electron and ion problems by expanding an eigenstate of the full hamiltonian $\hat{H}$ in the complete basis of electronic states
\begin{align}
\Psi_l(\bs{R},\RI) = \sum_{k=0}^\infty \chi_{lk}(\RI)\psi_k(\bs{R};\RI),
\end{align}
where $\chi_{lk}(\RI)$ are expansion coefficients with no dependence on electron positions. This expansion results in a system of Schr\"odinger-like equations each describing the ions moving on an electronic energy surface $E_j$
\begin{align} \label{eq:intro-ei-schro}
\left(-\sum_I\frac{\hbar^2}{2M_I}\nabla^2_I+E_j\right)\chi_j - \Lambda_{kj} = i\hbar\dot{\chi}_j,
\end{align}
and coupled by the so-called nonadiabatic operator $\Lambda_{jk}$~\cite{Worth2004}.
Derivation and behavior of $\Lambda_{jk}$ are discussed in Appendix~\ref{app:nad}.
While exact, eq.~(\ref{eq:intro-ei-schro}) is difficult to solve because all electronic states are coupled via $\Lambda_{kj}$.
To fully separate the electron and ion problems, one must approximate $\Lambda_{kj}$.

There are two common approximations to $\Lambda_{kj}$, the first is to set the entire matrix to zero, the second is to set only the off-diagonal terms to zero. Both approximations decouple (\ref{eq:intro-ei-schro}), allowing the complete separation of electronic and ionic motions.
%Therefore both approximations fall under the umbrella of adiabatic approximation, where the ion positions are considered to be slowly changing parameters of the electronic hamiltonian.
Many different and sometimes conflicting names have been given to these two approximations, including Born-Huang, Born-Oppenheimer and adiabatic approximation.
To fix nomenclature, I will call the all-zero approximation, $\Lambda_{jk}=0,~\forall j, k$, the Born-Oppenheimer approximation (BOA).
The diagonal terms $\Lambda_{jj}$ are considered diagonal Born-Oppenheimer correction (DBOC).
Non-zero off-diagonal elements are responsible for \textit{nonadiabatic effects}.

\section{Jellium}
The jellium model eq.~(\ref{eq:intro-jellium}) brings the electronic problem into focus. It replaces the material-dependent ionic potential in the electronic hamiltonian with a rigid homogeneous background of positive charge and is an important stepping stone to the electron-ion problem. In Hartree atomic units
\begin{align} \label{eq:intro-jellium}
\ham_{HEG} = \sum\limits_{i=1}^N -\dfrac{1}{2}\nabla_i^2 + \frac{1}{2}\sum\limits_{i=1}^N\sum\limits_{j=1,j\neq i}^N
\dfrac{1}{\vert \bs{r}_i-\bs{r}_j \vert}
+E_{e-b}+E_{b-b},
\end{align}
where $E_{e-b}$ and $E_{b-b}$ are constants due to electron-background and background-background interactions. The isotropic background eliminates potential symmetry breaking interactions that can be introduced by a crystalline arrangement of the ions. The rigidity of the background also removes electron-ion coupling effects. This model was studied in great detail in the past century and its ground-state behavior was largely understood. Much progress has even been made regarding its excitations and finite-temperature properties.

There is only one length scale in the jellium model: the average electron-electron separation $r_s=a/a_B$
\begin{align}
\dfrac{1}{n} \equiv \dfrac{\Omega}{N} = \frac{4\pi}{3}(r_sa_B)^3
\end{align}
in 3D, where $a_B$ is the Bohr radius. The kinetic energy scales as $r_s^{-2}$ (due to $\nabla^2$) while the potential energies scales as $r_s^{-1}$ (for $1/r$ potential), so $r_s$ measures the relative strength of potential to kinetic energy. In this sense, $r_s$ is the zero-temperature analogue of the classical coulomb coupling parameter
\begin{align}
\Gamma \equiv \dfrac{e^2n^{1/d}}{k_BT}.
\end{align}
The valence electron density in alkaline metals is $r_s\sim2$, meaning the kinetic energy is important, so the electrons delocalize and form a liquid to minimize kinetic energy. At sufficiently large $r_s$ ($\sim 100$ in 3D and $\sim 30$ in 2D), the potential energy dominates, so the electrons localize to form a Wigner crystal.

\section{Hydrogen}
Hydrogen is a logical starting point for solving the electron-ion problem.
It has the simplest atomic structure and no core electrons.
The non-relativistic Schr\"odinger equation eq.~(\ref{eq:intro-schro}) and (\ref{eq:intro-ei-ham}) should work well for hydrogen. Further, the ground state of its electronic hamiltonian can be compactly and accurately represented~\cite{Holzmann2003}.
Without core electrons, no essential modification needs to be made to the hamiltonain eq.~(\ref{eq:intro-ei-ham}) for a practical simulation, e.g., pseudopotential.

At sufficiently high temperatures, the hydrogen plasma, equal mixture of isotropic positive charge (protons) and negative charge (electrons), is a straightforward generalization of the jellium model to two components. However, at low temperatures, the two-component analogue of the Wigner crystal, solid hydrogen, is surprisingly complex.
Since hydrogen is the lightest element, its zero-point motion has large amplitude. The ion wave function explores a sufficiently large space to invalidate the harmonic approximation for lattice vibrations.
Further complicating matters, one can expect a metal-to-insulator transition as well as an atomic-to-molecular transition that may or may not coincide as temperature or pressure is decreased.
On top of all that, naturally occurring isotopes, e.g., deuterium, and spin isomers, e.g., para- and ortho-H$_2$, allow the possibility of an intriguing blend of quantum effects at low temperatures.

Hydrogen is also interesting due to its practical relevance. Being the most abundant element in the observable universe, hydrogen and its isotopes are crucial for the understanding of stars and gas giants.
Consider Jupiter, which contains liquid metallic hydrogen deep inside and insulating gaseous H$_2$ in the outer envelope. If there was a first-order liquid-liquid transition between the two phases, then there would be an interface across which density changes discontinuously.
Depending on the solubility of helium in the two phases, there is the possibility of helium rain across the interface and extra heat radiation due to this condensation~\cite{Militzer_thesis}.
Further, interior models of stars and gas giants rely on numerically accurate equation-of-state (EOS) of various chemical species involving hydrogen (H$^+$, H, H$_2$).
A few percent change to the hydrogen EOS is enough to eliminate/create a rocky core for Jupiter~\cite{Hubbard2016}.

In addition to its relevance in astrophysics, hydrogen is also important in energy applications. Accurate understanding of hydrogen EOS at high temperatures and pressures has obvious benefits to fusion experimental design.
Even at low temperatures, hydrogen-rich compounds, at sufficiently high pressures, have recently smashed the superconducting transition temperature records held by the so-called ``high-temperature'' superconductors~\cite{Drozdov2015,Errea2020}.

Finally, the 85-year-old prediction for a low-temperature insulating-to-metallic transition of solid hydrogen, the Wigner-Huntington transition, is close to being established~\cite{Wigner1935,Dias2016,Loubeyre2020}.
Experimental observations~\cite{Dias2016,Loubeyre2020} and theoretical calculations~\cite{McMinis2015,Gorelov2019} are converging, although more experimental characterizations and theoretical corrections are needed to settle current debates.

%At high pressure and low temperature, hydrogen may form a zero temperature liquid. If that were the case, then it may be a superconducting super fluid, the first of its kind.

\section{Lithium, Diamond, and Silicon}
Despite complications introduced by core electrons, the jellium model is arguably better realized in the valence of alkaline metals, e.g., lithium and sodium, than it is in hydrogen.
The heavier nuclei are less quantum and the core electrons screen their interaction with the valence electrons.
This allows the harmonic approximation to be more widely applicable.
More importantly, they are easier to handle in experiments than hydrogen and scatter X-rays more strongly, which facilitates precise experimental determination of lattice structure along with other properties.
These advantages allowed us to obtain excellent agreement with experiment using an electron-ion QMC simulation performed within the BOA in chapter~\ref{chap:licp}.

In addition to alkaline metals, elemental insulators, e.g., diamond carbon and silicon, remain important testing grounds for electronic QMC methods. Accurate and practical prediction of excitation energies using QMC is still an active area of research. We make some progress in chapter~\ref{chap:gap} by reducing finite-size error in bandgap calculations.

\section{Thesis Outline}
The remainder of this thesis is organized as follows. In chapter~\ref{chap:method}, I start by introducing the Monte Carlo methods that we use to accurately treat electron correlation in the electronic problem as well as to solve the full electron-ion problem at times. I then introduce the effective one-particle electronic structure methods used to generate trial wave function for the aforementioned QMC methods.
Chapter~\ref{chap:sj-wf} displays the form and properties of the Slater-Jastrow wave function in detail, while chapter~\ref{chap:fsc} discusses many-body finite-size correction, sometimes based on the properties of the many-body wave function.
Both wave function form and finite-size correction are crucial for accurate and practical QMC simulations.
The next four chapters display accurate QMC results for a few simple electron-ion systems.
Chapter~\ref{chap:na-mol} benchmarks the QMC method as a complete solver for the electron-ion problem without invoking the BOA on small atoms and molecules.
Chapter~\ref{chap:hsolid} applies this dynamic-ion QMC method to solve for the ground state of solid hydrogen.
Chapter~\ref{chap:licp} considers the effect of the ions on the momentum distribution of the valence electrons in lithium within the BOA.
Finally, chapter~\ref{chap:gap} presents an improved finite-size correction to the (purely electronic) bandgap calculated in QMC.