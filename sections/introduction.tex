\chapter{Introduction}

%Exact simulation of electron-ion systems is the grand challenge of this century.
Exact simulation of the homogeneous electron gas was the holy grail of the last century. It has led to quantitative understanding of metals and semiconductors as well as the popular local density approximation of the density functional which opened the flood gates on countless materials simulations. However, the grand challenge of this century is the exact simulation of electron-ion systems.
While the jellium model mimic many essential features of electrons in real materials, valence electrons in real materials interact with point charges (nuclei) and the surrounding inner electrons that screen them.
To complicate matters, these nuclei form a crystalline arrangement only on average. They move around and can change the electronic structure significantly. Even at absolute zero, the zero-point motion of the ions can still be the deciding factor of the stability between two candidate crystal structures.
The electron-ion problem is far from being solved.
Along with excited states.

Suppose a particle can be described by a probability distribution over space $x$ and time $t$, then the ``uniform'' particle is described by a plane wave
\begin{align}
\psi \propto \exp\left(
i~\dfrac{px-Et}{\hbar}
\right),
\end{align}
with constant momentum $p$ and energy $E$. The imaginary $i$ is needed to keep the amplitude of this wave $\psi^*\psi$ constant, whereas $\hbar$ is needed to remove units from the exponent. From derivatives of this plane wave, it is natural to conjecture the relations of conjugate variables to space and time, i.e. momentum and energy
\begin{align}
\left\{
\begin{array}{l}
\frac{d\psi}{dx} =  i\frac{p}{\hbar}\psi \\ [8 pt]
\frac{d\psi}{dt} = -i\frac{E}{\hbar}\psi
\end{array}
\right. \Rightarrow
\left\{
\begin{array}{l}
p\psi = -i\hbar\frac{d}{dx} \psi \\ [8 pt]
E\psi =  i\hbar\frac{d}{dt} \psi
\end{array}
\right..
\end{align}
Suppose total energy is a sum of non-relativist kinetic energy and local potential energy
\begin{align}
E = \frac{p^2}{2m} + V(x),
\end{align}
then the Hamiltonian operator
\begin{align}
\hat{\mathcal{H}} = \frac{\hat{p^2}}{2m} + V(x) = -\frac{\hbar^2}{2m}\nabla^2 + V(x).
\end{align}

Density matrix
\begin{align}
\rho\propto\exp\left(i\dfrac{S}{\hbar}\right)
\end{align}

\section{The Electron-Ion Problem}
The main goal of this thesis is the quantitatively accurate simulation of a many-body system of charged particles in the non-relativistic limit.
It will soon become clear that this is more of an aspiration than a practical goal at this time.
Nevertheless, progress has been made in a few areas.
For the remainder of this thesis, ground truth is assumed to be established by the exact solution of the Shcr\"odinger equation~(\ref{eq:intro-schro}) for the electron-ion wave function $\Psi$
\begin{align} \label{eq:intro-schro}
\hat{H} \Psi = i\hbar\frac{d}{dt}\Psi,
\end{align}
where the electron-ion hamiltonian consists of non-relativistic kinetic energy and Coulomb interactions
\begin{align} \label{eq:intro-ei-ham}
\hat{H} =
-\sum_I \frac{\hbar^2}{2m_I} \nabla^2_I 
-\sum_i \frac{\hbar^2}{2m_i}\nabla^2_i
-\sum_I\sum_i\frac{Z_Ie^2}{\vert\ri-\bs{r}_i\vert}
+\frac{1}{2}\sum_i\sum_{j\neq i} \frac{e^2}{\vert\bs{r}_i-\bs{r}_j\vert}
+\frac{1}{2}\sum_I\sum_{J\neq I} \frac{e^2}{\vert\ri-\rj\vert}.
\end{align}
The lower-case $i,j$ and upper-case $I,J$ loop over the electrons and ions, respectively. Further, the lower-case $\bs{r}_i$ labels a single electron position, whereas the upper-case $\bs{R}$ denotes the positions of all electrons $\bs{R}\equiv\{\bs{r}_i\}$. $\ri$ and $\RI$ play analogous roles for the ions. $Z_I$ is the atomic number of ion $I$. If any eigenstate of $\hat{H}$ can be constructed to arbitrary precision in a reasonable amount of time that grows as a polynomial as the number of particles, then the many-body problem can be declared solved. Unfortunately, even state-of-the-art methods struggle with just the ground state~\cite{Azadi2014,McMinis2015,Drummond2015}.

For equilibrium properties at high temperature, progress can be made by considering the Bloch equation~(\ref{eq:intro-bloch}) for the thermal density matrix $\rho\equiv\ket{\Psi}\bra{\Psi}$
\begin{align} \label{eq:intro-bloch}
\hat{H} \rho = -\hbar\dfrac{d}{d\beta}\rho,
\end{align}
which results from the Schr\"odinger eq.~(\ref{eq:intro-schro}) after a rotation from real to imaginary time $t\rightarrow it$, a.k.a. inverse temperature $\beta=1/(k_BT)$.
The partition function is a trace of the thermal density matrix and reduces to the classical Boltzmann distribution at high temperature
\begin{align}
Z=\Tr{\rho} \underset{\beta\rightarrow 0}{\Rightarrow} Z \propto e^{-\beta V}.
\end{align}
Further, lower temperature properties of boltzmannons and bosons can be exactly and efficiently calculated using the path integral method (Sec.\ref{sec:method-pimc}). Impressive results have been obtained when only the ground electronic state is considered~\cite{Pierleoni2016b,Celliers2018}.
However, a complete treatment of the full electron-ion hamiltonian eq.~(\ref{eq:intro-ei-ham}) is rarely attempted~\cite{Ceperley1981,Natoli1995}.

I consider the solution of the electron-ion problem an important milestone in computational condensed matter, because it is a natural extension of the jellium model to multi-component system and provides a firm foundation upon which relativistic effects can be included perturbatively. Further, the laplacian in the non-relativistic kinetic energy operator can be interpreted as a generator of diffusion in imaginary time. This makes it easy to develop intuitive understanding the behavior of quantum kinetic energy as well as the deployment of powerful computational techniques such as diffusion Monte Carlo (Sec.~\ref{sec:method-dmc}).

%\subsection{Hydrogen Atom}
%\subsection{H$_2$ Molecule}

\subsection{The Born-Oppenheimer Approximation (BOA)}
Given the aforementioned difficulties and the physical observation that ions move much slower than electrons due to their heavy mass ($m_I\approx 10^3\sim 10^5 m_i$), it is sensible to split the electron-ion problem into two parts
\begin{align} \label{eq:intro-ei-to-e-ham}
\hat{H} = -\sum_I \frac{\hbar^2}{2m_I} \nabla^2_I
+ \ham(\bs{R};\RI),
\end{align}
where $\ham$ is the electronic hamiltonian. The semicolon in $\ham(\bs{R};\RI)$ indicates the electronic hamiltonian is only parametrically dependent on the ion positions $\RI$. This means $\ham$ has no non-local operator involving the ionic degrees of freedom. M. Born and R. Oppenheimer (BO)~\cite{Born1927} first utilized this separation of time scales to study diatomic molecules in 1927. As explained around eq.~(27) and (28) in Ref.~\cite{Born1927}, BO expressed the electronic hamiltonian as a Taylor expansion around the equilibrium positions of the ions. They discussed results to the first fourth leading orders in the vibration amplitude of the ions, so what we define as ``the'' Born-Oppenheimer approximation (BOA) can be ambiguous. Here, I follow the interpretation by G. A. Worth and L. S. Cederbaum~\cite{Worth2004}, as described below.

% Beyond BO/Tutorial/h2_summary
If one has obtained the eigenstates of the electronic hamiltonian $\{\psi_k\}$
\begin{align}
\ham (\bs{R};\RI) \psi_k(\bs{R};\RI) = E_k(\RI) \psi_k(\bs{R};\RI),
\end{align}
then one can expand an eigenstate of the full hamiltonian $\hat{H}$ in the basis of electronic eigenstates
\begin{align} \label{eq:bo expansion}
\Psi_l(\bs{R},\RI) = \sum_{k=0}^\infty \chi_{lk}(\RI)\psi_k(\bs{R};\RI), 
\end{align}
where the expansion coefficients $\chi_{lk}(\RI)$ will later be identified with the ionic wave function in the Born-Oppenheimer approximation. $l$ runs over the full electron-ion hamiltonian's eigenstates, which can have both ionic/vibrational and electronic characters. According to the Schr\"odinger equation, the coefficient for one of these \textit{vibronic} states cannot be determined separately for each electronic level $k$ in general. To see this, substitute the $l=0$ expansion eq.~(\ref{eq:bo expansion}) into the time-dependent Schr\"odinger equation for the full electron-ion hamiltonian (drop $l$ for simplicity)
\begin{align}
\left(\ham - \sum_I\frac{\hbar^2}{2M_I}\nabla_I^2\right)\left(\sum_k\chi_k\psi_k\right) = i\hbar\frac{d}{dt}\left(\sum_k\chi_k\psi_k\right) \nonumber \Rightarrow \text{apply operators} \\
\sum_k E_k\chi_k\psi_k-\sum_I\frac{\hbar^2}{2M_I}\left(\nabla^2_I\chi_k\psi_k+2\bs{\nabla}_I\chi_k\cdot\bs{\nabla}_I\psi_k+\chi_k\nabla_I^2\psi_k\right) = \sum_k i\hbar\dot{\chi_k}\psi_k \nonumber \Rightarrow \text{apply }\int\psi_j^* \\
\sum_k E_k\chi_k\delta_{jk}-\sum_I\frac{\hbar^2}{2M_I}\left(\nabla^2_I\chi_k\delta_{jk}+2\bs{\nabla}_I\chi_k\cdot\bs{F}_{jk}+\chi_kG_{jk}\right) = \sum_k i\hbar\dot{\chi_k}\delta_{jk} \nonumber \Rightarrow \text{perform }\sum_k \\
\left(-\sum_I\frac{\hbar^2}{2M_I}\nabla^2_I+E_j\right)\chi_j - \left(\sum_k\sum_I \frac{\hbar^2}{2M_I}\left(2\bs{F}_I^{jk}\cdot\bs{\nabla}_I+G_I^{jk}\right)\chi_k\right) = i\hbar\dot{\chi}_j, \label{eq:intro-boa-coupled}
\end{align}
where the matrix elements for gradient (derivative-coupling terms) and laplacian (scalar-coupling terms) in the electronic eigenstates basis are
\begin{align}
\left\{\begin{array}{l}
\bs{F}^{jk}_I = \int d\bs{r} \psi_j^*(\bs{r};\bs{R}) \bs{\nabla}_I\psi_k(\bs{r};\bs{R}) \\
G_I^{jk} = \int d\bs{r} \psi_j^*(\bs{r};\bs{R}) \nabla_I^2 \psi_k(\bs{r};\bs{R})
\end{array}\right..
\end{align}
The matrix elements that couple different electronic states in eq.~(\ref{eq:intro-boa-coupled}) are named \emph{nonadibatic coupling operators} by Worth and Cederbaum~\cite{Worth2004}
\begin{align} \label{eq:intro-boa-nona}
\Lambda_{jk} = \sum_I \frac{\hbar^2}{2M_I}\left(2\bs{F}_I^{jk}\cdot\bs{\nabla}_I+G_I^{jk}\right).
\end{align}
Every term in $\Lambda_{jk}$ has an inverse ion mass prefactor $\frac{\hbar^2}{2M_I}$, so they are expected to be small in most cases. There are two common approximations of $\Lambda_{kj}$, the first is to set the entire matrix to zero, the second is to set only the off-diagonal terms to zero. Both approximations decouple (\ref{eq:intro-boa-coupled}), which allows the complete separation of electronic and ionic motions.
%Therefore both approximations fall under the umbrella of adiabatic approximation, where the ion positions are considered to be slowly changing parameters of the electronic hamiltonian. 
Many different and sometimes conflicting names have been given to these two approximations including Born-Huang, Born-Oppenheimer and adiabatic approximation.
To fix nomenclature, I will call the all-zero approximation, $\Lambda_{jk}=0,~\forall j, k$, the Born-Oppenheimer approximation (BOA).
The diagonal terms $\Lambda_{jj}$ are considered diagonal Born-Oppenheimer correction (DBOC).
Non-zero off-diagonal elements are responsible for \textit{nonadiabatic effects}.

\subsection{The Born-Oppenheimer Approximation}
The ground state in the BOA is a product of an ionic and an electronic component
\begin{align} \label{eq:intro-boa-product}
\Psi_{lk}^{BO}(\bs{R},\RI) = \chi_{lk}(\RI) \psi_k(\bs{R};\RI),
\end{align}
where a set of vibrational states labeled by $l$ can be defined over a particular electronic state $k$. $\chi_l(\RI)$ obeys its own Schr\"odinger equation on an effective potential energy surface provided by an eigenvalue of the electronic hamiltonian $E_k(\RI) = \braket{\psi_k|\ham|\psi_k}$, a.k.a. the Born-Oppenheimer potential energy surface (BO-PES)
\begin{align} \label{eq:boa-ion}
\left(-\sum_I\frac{\hbar^2}{2M_I}\nabla^2_I+E_k\right)\chi_l = i\hbar\dot{\chi}_l.
\end{align}
Once the ionic eigenstates are obtained by diagonalizing eq.~(\ref{eq:boa-ion}), the total energy of the electron-ion system is finally obtained as
\begin{align}
E_{lk}^{BO} \equiv \braket{\chi_l| E_k -\sum_I\frac{\hbar^2}{2M_I}\nabla^2_I |\chi_l}.
\end{align}
$E_{00}^{BO}$ differs in two ways from the electronic ground-state energy
\begin{align}
E_0 \equiv \braket{\psi_0|\ham|\psi_0}(\RI),
\end{align}
which is a function of the positions of the ions $\RI$. First, in $E_{00}^{BO}$ the electronic energy is averaged over a distribution of ion configurations $\vert\chi_0\vert^2(\RI)$ rather than evaluated at one fixed configuration $\RI$. This quantum delocalization effect raises the total energy from the bottom of the BO-PES $\RI^e=\underset{\RI}{\text{argmin}}~E_0(\RI)$, which would have been the electron-ion ground state if the ions were classical. Second, the ions have kinetic energy even at absolute zero, which also contributes a positive term to the total energy.
The difference between the electron-ion ground-state energy and the electronic one is the zero-point energy (ZPE). In the BOA, ZPE contains only two terms from delocalization and kinetic energy of the ions.

The main short-fall of the BOA is its lack of pathways for the ions to transfer energy to the electrons. This is critical in the study of radiation damage, where a fast moving ion can transfer energy to both the electrons and the ions in a material. Further, bond breaking cannot occur via an electronic transition from binding to anti-binding state. The BOA can also break down if the electrons interact with a particle much lighter than an atomic nucleus, e.g., a positron or a muon. Finally, the nonadiabatic coupling terms can diverge when two electronic states cross, e.g., at a conical intersection. Thus, it is sometimes important to go beyond the BOA.

\subsection{Beyond the Born-Oppenheimer Approximation}

\subsubsection{Small Parameters in the Nonadiabatic Coupling Operators}
There are two small parameters that control the scale of nonadiabatic coupling eq.~(\ref{eq:intro-boa-nona}). One is clearly the inverse ionic mass $\frac{1}{M_I}$, while the other is the difference between electronic energy levels $\epsilon_j-\epsilon_k$. This can be seen from an explicit form of $\bs{F}_I^{jk}\equiv\braket{\psi_j|\bs{\nabla}_I|\psi_k}$ in the derivative-coupling term. Consider the effect of ion motion on the electronic problem, i.e., take $\bs{\nabla}_I$ of the time-independent electronic Schr\"odinger equation
\begin{align}
&\bs{\nabla}_I(\ham\psi_k) = \bs{\nabla}_I(\epsilon_k\psi_k ) \Rightarrow \psi_k\bs{\nabla}_I\ham + \ham\bs{\nabla}_I\psi_k = \psi_k\bs{\nabla}_I\epsilon_k + \epsilon_k\bs{\nabla}_I\psi_k \Rightarrow \nonumber \text{apply }\int\psi_j^* \\ &\left(\int \psi_j^*\psi_k\bs{\nabla}_I\ham\right) + \epsilon_j \bs{F}_I^{jk} = \bs{\nabla}_I\epsilon_k \delta_{jk}+\epsilon_k\bs{F}_I^{jk} \Rightarrow \nonumber \text{solve for }\bs{F}_I^{jk} \\
&\bs{F}_I^{jk} = \dfrac{\braket{\psi_j|\bs{\nabla}_I\ham|\psi_k}+\bs{\nabla}_I\epsilon_k\delta_{jk}}{\epsilon_k-\epsilon_j}.
\end{align}
If the electronic eigenstates are defined to be orthonormal, then the real part of the derivative-coupling vectors vanish $(\bs{F}_I^{kj})^*+\bs{F}_I^{jk}=\bs{0} \Rightarrow \text{Re}[\bs{F}_I^{jk}]=0$ as derived in eq.~(\ref{eq:intro-boa-refji}).
\begin{align}
&\left\{\begin{array}{l}
\bs{F}^{jk}_I = \braket{\psi_j|\bs{\nabla}_I\psi_k} \\
\braket{\psi_j|\psi_k} = \delta_{jk}
\end{array}\right. \Rightarrow \bs{\nabla}_I \braket{\psi_j|\psi_k} = \bs{0} \nonumber \Rightarrow \\
& \braket{\bs{\nabla}_I\psi_j|\psi_k} + \braket{\psi_j|\bs{\nabla}_I\psi_k}   = (\bs{F}_I^{kj})^*+\bs{F}_I^{jk} = \bs{0}.
\label{eq:intro-boa-refji}
\end{align}
$\bs{F}_I^{kj}$ can be interpreted as follows: the motion of the ions apply an imaginary ``force'' $\braket{\psi_j|\bs{\nabla}_I|\psi_k}$ that drives an electronic transition from state $k$ to state $j$. Further, this ``force'' is inversely proportional to the energy separation between the two eigenstates. The derivative-coupling term is considered more interesting than the scalar-coupling term due to its potential divergence as $\epsilon_k\rightarrow\epsilon_j$.

Further, if the wave function is real then there is no derivative coupling within the same electronic state $\bs{F}_I^{jj}=\braket{\psi_j|\bs{\nabla}_I\psi_j} = \braket{\bs{\nabla}_I\psi_j|\psi_j} = (\bs{F}_I^{jj})^* \Rightarrow \bs{F}_I^{jj}=\bs{0}$. In this case, the DBOC is simply the expectation value of the ion kinetic operator in the electronic state
\begin{align}
\Lambda_{jj} = \sum_I \frac{\hbar^2}{2M_I}G_I^{jj} = \int \psi_j^* \sum_I \frac{\hbar^2}{2M_I} \nabla^2_I\psi_j = \braket{\hat{T}_I}_j.
\end{align}

\subsubsection{Diagonal Born-Oppenheimer Correction to H$_2$}
The diagonal correction for the hydrogen molecule was studied extensively by Kolos and Wolniewicz~\cite{Kolos1964,Kolos1968,Wolniewicz1983}.
For the atomization energy of $H_2$, the DBOC was found to be $4.947$ cm$^{-1}$, which is only a $0.0129$\% correction of its clamped-ion value of $38292.7$ cm$^{-1}$.
%They did not explicitly calculate the diagonal correction for HD and D$_2$ because ``The corrections for HD and D$_2$ are readily obtained from those for H$_2$ by multiplications by the mass ratios''. I think the assumption here is that the geometry of the isotopes of H$_2$ are sufficiently close to that of H$_2$ that the electronic ground-state is unperturbed.

\section{Helium and Jellium}

\section{Hydrogen}
Hydrogen is a logical starting point for solving the electron-ion problem.
It has the simplest atomic structure and no core electrons.
The non-relativistic Schr\"odinger equation eq.~(\ref{eq:intro-schro}) and (\ref{eq:intro-ei-ham}) should work well for hydrogen. Further, the ground state of its electronic hamiltonian can be compactly and accurately represented~\cite{Holzmann2003}.
Without core electrons, no essential modification needs to be made to the hamiltonain eq.~(\ref{eq:intro-ei-ham}) for a practical simulation, e.g., pseudopotential.

At sufficiently high temperatures, hydrogen plasma, equal mixture of isotropic positive charge (protons) and negative charge (electrons), is a straightforward generalization of the jellium model to two-component system. However, at low temperatures, the two-component analogue to the Wigner crystal, solid hydrogen, is surprisingly complex.
Since hydrogen is the lightest element, its zero-point motion has large amplitude. The ion wave function explores a sufficiently large space to invalidate the harmonic approximation for lattice vibrations.
Further complicating matters, one can expect a metal-to-insulator transition as well as an atomic-to-molecular transition that may or may not coincide as temperature or pressure is decreased.
On top of all that, naturally occurring isotopes, e.g., deuterium, and spin isomers make for an intriguing blend of intricate quantum effects.

Hydrogen is also interesting due to its practical relevance. Being the most abundant element in the observable universe, hydrogen and its isotopes are crucial for understanding of starts and giant gas planets. Knowing the properties of hydrogen can also help us harness the power of the sun to produce the clean energy source that is fusion.
At high pressure and low temperature, hydrogen may form a zero temperature liquid. If that were the case, then it may be a superconducting super fluid, the first of its kind. More realistically, solid atomic hydrogen is predicted to be a room temperature superconductor. Hydrogen-rich compounds have been shown to smash all superconducting temperature records held by the so-called ``high-temperature'' superconductors via what is believed to be conventional BCS mechanism.

\section{Lithium}

\section{Silicon}

\section{Thesis Outline}