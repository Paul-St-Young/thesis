\chapter{Introduction}

Suppose a particle can be described by a probability distribution over space $x$ and time $t$, then the ``uniform'' particle is described by a plane wave
\begin{align}
\psi \propto \exp\left(
i~\dfrac{px-Et}{\hbar}
\right),
\end{align}
with constant momentum $p$ and energy $E$. The imaginary $i$ is needed to keep the amplitude of this wave $\psi^*\psi$ constant, whereas $\hbar$ is needed to remove units from the exponent. From derivatives of this plane wave, it is natural to conjecture the relations of conjugate variables to space and time, i.e. momentum and energy
\begin{align}
\left\{
\begin{array}{l}
\frac{d\psi}{dx} =  i\frac{p}{\hbar}\psi \\ [8 pt]
\frac{d\psi}{dt} = -i\frac{E}{\hbar}\psi
\end{array}
\right. \Rightarrow
\left\{
\begin{array}{l}
p\psi = -i\hbar\frac{d}{dx} \psi \\ [8 pt]
E\psi =  i\hbar\frac{d}{dt} \psi
\end{array}
\right..
\end{align}
Suppose total energy is a sum of non-relativist kinetic energy and local potential energy
\begin{align}
E = \frac{p^2}{2m} + V(x),
\end{align}
then the Hamiltonian operator
\begin{align}
\hat{\mathcal{H}} = \frac{\hat{p^2}}{2m} + V(x) = -\frac{\hbar^2}{2m}\nabla^2 + V(x).
\end{align}

Density matrix
\begin{align}
\rho\propto\exp\left(i\dfrac{S}{\hbar}\right)
\end{align}

\section{The Many-electron Problem}
%\subsection{Hydrogen Atom}
%\subsection{H$_2$ Molecule}
\subsection{Born-Oppenheimer Approximation (BOA)}
\subsection{Beyond the BOA}
\section{Lithium Compton Profile}
\section{Insulator Band Gap}
\section{Outline}