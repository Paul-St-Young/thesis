\section{Multi-Component RPA Jastrow}
\label{sec:mc-rpa-jas}

\textit{Based on notes from D. M. Ceperley dated Sep. 1980}

Given Jastrow wavefunction $\Psi=\exp(-U)$, where
\begin{align}
U = \sum\limits_{i<j} u(r_{ij}) = \frac{1}{2} \sum\limits_{\alpha, \beta} \sum\limits_{i=1}^{N_\alpha} \sum\limits_{j=1}^{N_\beta, (j,\beta)\neq(i,\alpha)}
u^{\alpha\beta}(\vert\bs{r}_i^\alpha-\bs{r}_j^\beta\vert), \label{eq:Uurij}
\end{align}
and non-relativistic Coulomb hamiltonian
\begin{align}
H = \hat{T}+V = \sum\limits_{\alpha}\sum\limits_{j=1}^{N_\alpha} -\lambda_{\alpha}\nabla^{\alpha2}_j + \frac{1}{2}\sum\limits_{\alpha, \beta} \sum\limits_{i=1}^{N_\alpha} \sum\limits_{j=1}^{N_\beta, (j,\beta)\neq(i,\alpha)} v^{\alpha\beta}(\vert\bs{r}_i^\alpha - \bs{r}_j^\beta \vert),
\end{align}
where $\alpha, \beta$ label particle species, $i, j$ label particle positions. $\lambda_\alpha=\frac{\hbar^2}{2m_\alpha}$, $v^{\alpha\beta}(\bs{r})=\frac{Q_\alpha Q_\beta}{r}$. In terms of pair potentials and collective coordinates (see Fourier convention eq.~(\ref{eq:forward-ft}) and its corollaries eq.~(\ref{eq:inverse-ft}-\ref{eq:rhok}))
\begin{align}
U = \frac{1}{2\Omega} \sum\limits_{\bs{k}\alpha\beta}
u^{\alpha\beta}_{\bs{k}} \left( \rho^\beta_{\bs{k}}\rho^\alpha_{-\bs{k}}-\delta_{\alpha\beta}N_\alpha
\right), \label{eq:Uuk} \\
V = \frac{1}{2\Omega} \sum\limits_{\bs{k}\alpha\beta}
v^{\alpha\beta}_{\bs{k}} \left( \rho^\beta_{\bs{k}}\rho^\alpha_{-\bs{k}}-\delta_{\alpha\beta}N_\alpha \right).
\end{align}

The \textbf{goal} is to obtain good Jastrow pair potentials $u_{\bs{k}}^{\alpha\beta}$. The \textit{strategy} is to minimize the variance the local energy $E_L\equiv \Psi^{-1}H\Psi = T + V$, where
\begin{align}
T =& \sum\limits_{\gamma}-\lambda_{\gamma}
\sum\limits_{l=1}^{N_\gamma}  \left(\bs{\nabla}_l^{\gamma}U\cdot\bs{\nabla}_l^{\gamma}U - \nabla^{\gamma2}_l U \right).
\end{align}

In the following, I will detail the few steps needed to obtain the RPA Jastrow potentials. First, we express the local energy in terms of the collective coordinates eq.~(\ref{eq:rhok}). Second, we find the equations that make the local energy invariant to changes in the collective coordinates. Third and finally, we solve these equations for one and two component systems. Assume $u^{\alpha\beta}=u^{\beta\alpha}$ and $u_{\bs{k}}=u_{-\bs{k}}$.

\subsection{Local Energy of Jastrow Wavefunction}
The gradient, laplacian, and gradient squared of eq.~(\ref{eq:Uuk}) are
\begin{align}
%\left\{\begin{array}{l}
\bs{\nabla}_l^\gamma U =& \frac{1}{2\Omega} \sum\limits_{\bs{k}\alpha} (i\bs{k})u_{\bs{k}}^{\gamma\alpha} \left(
e^{i\bs{k}\cdot\bs{r}_l^\gamma}\rho_{-\bs{k}}^\alpha - \rho_{\bs{k}}^\alpha e^{-ik\cdot\bs{r}_l^\gamma}
\right) \\
\bs{\nabla}_l^\gamma\cdot\bs{\nabla}_l^\gamma U =& -\frac{1}{2\Omega}\sum\limits_{\bs{k}\alpha} k^2u_{\bs{k}}^{\gamma\alpha}\left(
e^{i\bs{k}\cdot\bs{r}_l^\gamma}\rho_{-\bs{k}}^\alpha + \rho_{\bs{k}}^\alpha e^{-\bs{k}\cdot\bs{r}_l^\gamma} - 2\delta_{\alpha\gamma}
\right) \label{eq:lapl}\\
\bs{\nabla}_l^\gamma U\cdot\bs{\nabla}_l^\gamma U =& -\frac{1}{4\Omega^2}\sum\limits_{\bs{k}\bs{q}\alpha\beta}\bs{k}\cdot\bs{q} u_{\bs{k}}^{\gamma\alpha}u_{\bs{q}}^{\gamma\beta} \times\left(\right.\nonumber \\
e^{i(\bs{k}+\bs{q})\cdot\bs{r}}\rho_{-\bs{k}}^\alpha\rho_{-\bs{q}}^\beta -
e^{i(\bs{k}-\bs{q})\cdot\bs{r}}\rho_{-\bs{k}}^\alpha\rho_{\bs{q}}^\beta -&
e^{i(\bs{q}-\bs{k})\cdot\bs{r}}\rho_{\bs{k}}^\alpha\rho_{-\bs{q}}^\beta +
e^{-i(\bs{k}+\bs{q})\cdot\bs{r}}\rho_{\bs{k}}^\alpha\rho_{\bs{q}}^\beta \nonumber \\
&\left.\right) \label{eq:grad2l}.
%\end{array}\right..
\end{align}
Summing over $l$ turns $e^{i\bs{k}\cdot\bs{r}_l^\gamma}$ into $\rho_{\bs{k}}^\gamma$ in eq.~(\ref{eq:lapl}) and (\ref{eq:grad2l}). Thus
\begin{align}
\sum\limits_{l=1}^{N_\gamma}\bs{\nabla}_l^\gamma\cdot\bs{\nabla}_l^\gamma U =& -\frac{1}{2\Omega}\sum\limits_{\bs{k}\alpha} k^2u_{\bs{k}}^{\gamma\alpha}\left(
\rho_{\bs{k}}^\gamma\rho_{-\bs{k}}^\alpha + \rho_{\bs{k}}^\alpha \rho_{-\bs{k}}^\gamma - 2N_\gamma\delta_{\alpha\gamma}
\right) \label{eq:lap}\\
\sum\limits_{l=1}^{N_\gamma}\bs{\nabla}_l^\gamma U\cdot\bs{\nabla}_l^\gamma U =& -\frac{1}{4\Omega^2}\sum\limits_{\bs{k}\bs{q}\alpha\beta}\bs{k}\cdot\bs{q} u_{\bs{k}}^{\gamma\alpha}u_{\bs{q}}^{\gamma\beta} \times\left(\right.\nonumber \\
\rho_{\bs{k}+\bs{q}}^\gamma\rho_{-\bs{k}}^\alpha\rho_{-\bs{q}}^\beta -
\rho_{\bs{k}-\bs{q}}^\gamma\rho_{-\bs{k}}^\alpha\rho_{\bs{q}}^\beta -&
\rho_{\bs{q}-\bs{k}}^\gamma\rho_{\bs{k}}^\alpha\rho_{-\bs{q}}^\beta +
\rho_{-(\bs{k}+\bs{q})}^\gamma\rho_{\bs{k}}^\alpha\rho_{\bs{q}}^\beta \nonumber \\
&\left.\right) \label{eq:grad2}.
\end{align}
Equation (\ref{eq:grad2}) contains terms that couple three wave vectors, i.e. $O(\rho^3)$. In the spirit of RPA, we will drop all such \emph{mode coupling} terms. Note $\rho_{\bs{0}}^\gamma = N_\gamma$, and use $u_{\bs{k}}=u_{-\bs{k}}$
\begin{align}
\sum\limits_{l=1}^{N_\gamma}\bs{\nabla}_l^\gamma U\cdot\bs{\nabla}_l^\gamma U =&
\frac{N_\gamma}{2\Omega^2}\sum\limits_{\bs{k}\alpha\beta} k^2u_{\bs{k}}^{\gamma\alpha}u_{\bs{k}}^{\gamma\beta} \left( \rho_{-\bs{k}}^\alpha\rho_{\bs{k}}^\beta + \rho_{\bs{k}}^\alpha\rho_{-\bs{k}}^\beta \right). \label{eq:grad2_rpa}
\end{align}
Finally, sum over $\gamma$ with $-\lambda_\gamma$ to obtain terms in the kinetic energy. To simplify later assembly of the local energy, rename dummy variables $\alpha, \beta, \gamma$ such that every $O(\rho^2)$ term contains $\rho_{\bs{k}}^\alpha\rho_{-\bs{k}}^\beta$ (use $u^{\alpha\beta}=u^{\beta\alpha}$)
\begin{align}
\sum\limits_{\gamma}-\lambda_\gamma\sum\limits_{l=1}^{N_\gamma}\bs{\nabla}_l^\gamma U\cdot\bs{\nabla}_l^\gamma U =&-\frac{1}{\Omega}
\sum\limits_{\bs{k}\alpha\beta\gamma} \lambda_\gamma\frac{N_\gamma}{\Omega}
k^2u_{\bs{k}}^{\gamma\alpha}u_{\bs{k}}^{\gamma\beta} \rho_{\bs{k}}^\alpha\rho_{-\bs{k}}^\beta, \\
\sum\limits_{\gamma}-\lambda_\gamma\sum\limits_{l=1}^{N_\gamma}\bs{\nabla}_l^\gamma \cdot\bs{\nabla}_l^\gamma U =& -\frac{1}{\Omega}\left( 
\sum\limits_{\bs{k}\alpha\beta} \frac{\lambda_\alpha+\lambda_\beta}{2} u_{\bs{k}}^{\alpha\beta}\rho_{\bs{k}}^\alpha\rho_{-\bs{k}}^\beta -N_\alpha\lambda_\alpha\delta_{\alpha,\beta} \right).
\end{align}

Finally, the local energy can be assembled
\begin{align}
E_L = \sum\limits_{\bs{k}}\left[\frac{v^{\alpha\beta}_{\bs{k}}}{2\Omega} - 
\frac{\frac{\lambda_\alpha+\lambda_\beta}{2}k^2u_{\bs{k}}^{\alpha\beta}}{\Omega} -
\sum\limits_\gamma \frac{(N_\gamma/\Omega)\lambda_\gamma k^2u_{\bs{k}}^{\alpha\gamma}u_{\bs{k}}^{\beta\gamma}}{\Omega}\right]\rho_{\bs{k}}^\alpha\rho_{-\bs{k}}^\beta + \frac{N_\alpha^2(\lambda_\alpha+\frac{1}{2})}{\Omega}.
\end{align}

\subsection{Equations that define the RPA Jastrow Pair Potentials}
Variance of $E_L$ can be minimized by setting the $\rho_{\bs{k}}^\alpha\rho_{-\bs{k}}^\beta$ term to zero. Define $\epsilon^\alpha_{\bs{k}}\equiv \lambda_\alpha k^2$
\begin{align}
\frac{v^{\alpha\beta}_{\bs{k}}}{2} - \frac{\epsilon^\alpha_{\bs{k}}+\epsilon^\beta_{\bs{k}}}{2}
u_{\bs{k}}^{\alpha\beta} -
\sum\limits_\gamma \frac{N_\gamma}{\Omega} \epsilon^\gamma_{\bs{k}} u_{\bs{k}}^{\alpha\gamma}u_{\bs{k}}^{\beta\gamma} = 0. \label{eq:mineq}
\end{align}
Equation~(\ref{eq:mineq}) can be solved for each $\bs{k}$ independently. We no longer need the collective coordinates or the label $\bs{k}$. It is now safe to recycle the symbol $\rho_\gamma\equiv\frac{N_\gamma}{\Omega}$ to mean the number density of species $\gamma$. Simplify eq.~(\ref{eq:mineq}) to
\begin{align}
\frac{v_{\alpha\beta}}{2} - \frac{1}{2}(\epsilon_\alpha+\epsilon_\beta)
u_{\alpha\beta} -
\sum\limits_\gamma \rho_\gamma \epsilon_\gamma 
u_{\alpha\gamma}u_{\beta\gamma} = 0 . \label{eq:mineq_nok}
\end{align}

\subsection{Solving for te RPA Jastrow Pair Function}

\textit{One Component}

For a one-component system, eq.~(\ref{eq:mineq_nok}) becomes a quadratic equation of one variable $u_{11}$
\begin{align}
\frac{v_{11}}{2} - \epsilon_1 u_{11} - \rho_1\epsilon_1u_{11}^2 = 0.
\end{align}
The solution is
\begin{align}
2\rho_1 u_{11} = -1 + \sqrt{1+2\rho_1v_{11}/\epsilon_1}, \label{eq:oc-rpa-uk}
\end{align}
which agrees with Gaskell's solution eq.~(\ref{eq:wf-gaskell-rpa-uk}), except $S_0(k)$ is replaced by $1$. Notice, if one uses a different Fourier convention, replacing volume $\Omega$ with number of particles $N$ in eq.~(\ref{eq:Uuk})
\begin{align}
U = \frac{1}{2N} \sum\limits_{\bs{k}\alpha\beta}
\tilde{u}^{\alpha\beta}_{\bs{k}} \left( \rho^\beta_{\bs{k}}\rho^\alpha_{-\bs{k}}-\delta_{\alpha\beta}N_\alpha
\right),
\end{align}
then the density $\rho$ drops from the expression for $\tilde{u}$, e.g., eq.~(8) in Ref.~\cite{Ceperley1978} and eq.~(3) in Ref.~\cite{Ceperley1981}
\begin{align}
2\tilde{u} = -1 + (1+2v_k/e_k).
\end{align}

\textit{Two Components}

For two-component system, eq.~(\ref{eq:mineq_nok}) becomes a set of 3 coupled quadratic equations
\begin{align}
\left\{\begin{array}{l}
\frac{v_{11}}{2} - \epsilon_1 u_{11} - \rho_1\epsilon_1u_{11}^2 - \rho_2\epsilon_2u_{12}^2 = 0 \\
\frac{v_{12}}{2} - \frac{1}{2}(\epsilon_1+\epsilon_2) u_{12} - \rho_1\epsilon_1u_{11}u_{12} - \rho_2\epsilon_2u_{12}u_{22} = 0 \\
\frac{v_{22}}{2} - \epsilon_2 u_{22} - \rho_1\epsilon_1u_{12}^2 - \rho_2\epsilon_2u_{22}^2 = 0.
\end{array}\right.
\end{align}

%In the case of electron-ion matter, 
Suppose species $2$ has infinite mass $\lambda_2\rightarrow0$, thus no dispersion $\epsilon_2=0$. Then we should ignore the last equation ($\alpha=\beta=2$), which determines $u_{22}$ (when $u_{12}=0$). The remaining equations allow us to solve for the Jastrow pair potentials $u_{11}$ and $u_{12}$
\begin{align}
\left\{\begin{array}{l}
\frac{v_{11}}{2} - \epsilon_1 u_{11} - \rho_1\epsilon_1u_{11}^2 = 0 \\
\frac{v_{12}}{2} - \frac{\epsilon_1}{2} u_{12} - \rho_1\epsilon_1u_{11}u_{12} = 0
\end{array}\right..
\end{align}
The first equation provides the same Jastrow potential as in the one-component case eq.~(\ref{eq:oc-rpa-uk}). The second equation can be used to solve for $u_{12}$
\begin{align}
&(1+2\rho u_{11}) \epsilon_1 u_{12} = v_{12} \nonumber \\
\Rightarrow & u_{12} = \frac{v_{12}/\epsilon_1}{\sqrt{1+2\rho_1v_{11}/\epsilon_1}}.
\end{align}
For completeness, the exact solutions are (by Mathematica)
\begin{align}
2\rho_1u_{11} =& -1+\frac{\epsilon_1(1+a_{11}) \pm \epsilon_2\sqrt{A}}
{
\sqrt{\epsilon_1\epsilon_2}\sqrt{B}
},\\
u_{12} =& \frac{\pm v_{12} }{
\sqrt{\epsilon_1\epsilon_2}\sqrt{B}
},\\
2\rho_2u_{22} =& -1+\frac{\epsilon_2(1+a_{22}) \pm \epsilon_1\sqrt{A}}
{
\sqrt{\epsilon_1\epsilon_2}\sqrt{B}
},
\end{align}
where 
\begin{align}
A =& (1+a_{11})(1+a_{22})-a_{12}^2,\\
B =&\frac{\epsilon_2}{\epsilon_1}(1+a_{22}) + \frac{\epsilon_1}{\epsilon_2}(1+a_{11}) \pm 2\sqrt{A},
\end{align}
with
\begin{align}
\left\{\begin{array}{l}
a_{11}=\frac{2\rho_1v_{11}}{\epsilon_1} \\
a_{12}=\frac{2\sqrt{\rho_1\rho_2}v_{11}}{\sqrt{\epsilon_1\epsilon_2}} \\
a_{22}=\frac{2\rho_2v_{22}}{\epsilon_2} \\
\end{array}\right..
\end{align}