\section{Hydrogen under Pressure}

\subsection{Solid phases}

As element number one with the simplest atomic structure, hydrogen has surprising complex phases at megabar pressures.
The phase diagram differs for the isotope (hydrogen H, deuterium D, tritium T) and spin isomers of molecular hydrogen.
We have para-hydrogen (p-H$_2$) when the proton spins are anti-aligned to form a singlet, and ortho-hydrogen (o-H$_2$) when they are aligned to form a triplet.
At low pressure ($<100$ GPa) and below the melting temperature, spherically symmetric p-H$_2$ form a hexagonal close packed (HCP) crystal, whereas o-H$_2$ has orientation order at ambient pressure.
ortho-para conversion takes days to reach equilibrium at low temperature and pressure. However, at megabar pressures, ortho-para conversion is fast so most samples will contain equilibrium concentrations of ortho- and para- species \textbf{[Silvera]}.
Phases with natural concentrations of H$_2$ (n-H$_2$) are labeled I, II, III~\cite{Dias2019}.
The vibration of the $H_2$ molecules are both Raman- and IR-active, displaying a sharp ``vibron'' peak around 4200 cm$^{-1}$ at 100 GPa and decreasing roughly linearly to 4060 cm$^{-1}$ at 150 GPa, with a 3 cm$^{-1}$ jump at 110 GPa~\cite{Lorenzana1990}.
Transitions between the rotational states of the molecule can also be observed in Raman spectrum as ``roton bands'' around 350 cm$^{-1}$. At 8 K and above 110 GPa, the roton bands broaden, indicating that the hydrogen molecules start to lose their spherical symmetry and become rotationally ordered. This broken-symmetry phase (BSP) or phase II, stable under 140 K, is considered a quantum phase based on observed strong isotope shift of the transition pressure. o-D$_2$ reaches the BSP phase at as low as 28 GPa at 1.1 K~\cite{Silvera1981}.
Phase II is stabilized by anisotropic intermolecular interactions, which increase with pressure and lower the energy the $J=2$ rotational state relative to that of the $J=0$ state.
Below 140 K and above 150 GPa, hydrogen enters a different orientation-ordered phase III regardless of ortho-para content and isotope~\cite{Cui1995,Goncharov1998}.
In this new phase, the rather broad and pressure-independent roton band weakens, disappears, and is replaced by a few sharp and strongly pressure-dependent peaks in the frequency range 100$\sim$700 cm$^{-1}$~\cite{Goncharov1998}.
These new modes are considered to be lattice libration modes due to their pressure dependence.
At 300 K and above 220 GPa, we enter yet another solid phase IV, characterized by a splitting of the vibron peak~\cite{Zha2013}.
Both theory and experiment suggest that phase IV consists of alternating layers having rather different in-plane structures.
At $<100$ K and above 350 GPa, molecular hydrogen becomes semi-metallic, possibly due to the closure of an indirect band gap~\cite{Eremets2019}.
Then, above 425 GPa, all IR radiation is absorbed indicating a closure of the direct band gap~\cite{Loubeyre2020}.
Finally, at sufficiently high pressures, the hydrogen molecules will dissociate to form an atomic solid, reportedly at 495 GPa~\cite{Silvera2017}. Although consensus has yet to be reached.

While the phase boundaries of solid hydrogen are reasonably well-established below 400 K and 400 GPa by diamond-anvil cell (DAC) experiments~\cite{Dias2019}, characterizations of the solid structures are limited. Due to small scattering cross section and small sample size in DAC experiments, only a handful of X-ray~\cite{Hazen1987,MAO1988,Loubeyre1996,Kawamura2002,Goncharenko2005a,Akahama2010,Ji2019} and only one neutron~\cite{Goncharenko2005a} scattering experiments haven been published over the past 40 years. Most of our understanding of solid hydrogen is built upon IR and Raman spectra, which provide partial information on the microscopic details of the solid structures. This lack of definitive structural information poses significant difficulty for both theoretical and experimental understanding of solid hydrogen. Experimentally, this has lead to the misidentification of a triple point as a critical point~\cite{Lorenzana1990,Cui1994}, subtlety in the detection of a new phase~\cite{Eremets2009,Howie2012}, among many debates over interpretation of optical data. One particularly challenging transition to detect is the melting of the molecular crystal.

The melting temperature of solid hydrogen is typically determined in one of three ways: 1. visual observation of motion due to sample, contaminant, or laser speckle~\cite{Gregoryanz2003}, 2. discontinuous change in the Raman spectrum, e.g. vibron frequency jump, disappearance of librons~\cite{Gregoryanz2003,Subramanian2011,Zha2017}, 3. plateau temperature during laser heating~\cite{Deemyad2008}, and 4. rapid change in interferences pattern~\cite{Eremets2009}.
As sample size decreases with pressure, visual detection of melting becomes more difficult. Further, the refractive indexes of hydrogen and diamond approach each other around 140 GPa, defeating detection by interference pattern~\cite{Zha2017}.
The vibron discontinuity caused by melting reduces from $\sim$ 3 cm$^{-1}$ at 10 GPa to $\sim$ 1 cm$^{-1}$ at 45 GPa~\cite{Gregoryanz2003}, making melting detection more difficult at higher pressures.
Finally, disappearance of librons alone cannot be taken as proof of melting, because it could be due to loss of sample.
Due to these difficulties, the precise melting temperatures above 100 GPa are debate. Although, there is broad consensus that the melting line increases from 200 K at ambient pressure to $\sim$ 1000 K at 90 GPa, then decreases with increasing pressure.

Melting of $H_2$ solid around 150 GPa is particularly interesting. Shock wave compression data indicate a molecular liquid to atomic liquid transition $\sim$ 1000 K at 150 GPa. Thus, $H_2$ molecules should dissociate almost immediately after the solid melts. Eremets and Trojan observed a shark drop in measured resistivity as the $H_2$ solid melts at 146 GPa~\cite{Eremets2009}. This observation is consistent with the current liquid-liquid transition (LLT) line.

%\subsection{Liquid phases and the liquid-liquid transition}

\subsection{The melting transition and critical point}
