\chapter{QMC Study of Solid Hydrogen at Megabar Pressures}

\section{Introduction}

% why is hydrogen under pressure interesting?
Properties and phase transitions of hydrogen under megabar pressures is important in diverse fields of study. For astronomy, models of the interior of giant gas planets such as Jupiter and Saturn depend critically on the nature of the molcular liquid to atomic liquid transition (LLT), namely whether it is first-order or continuous~\cite{Hubbard2016,Wahl2017}. For condensed matter, metallic hydrogen holds promise for a room temperature conventional superconductor~\cite{McMahon2011,McMahon2012}. For computational physics, hydrogen remains an important benchmark for both electronic structure~\cite{Motta2017} and ion dynamics methods. With no need for a pseudopotential, simulations of hydrogen avoid a significant source of bias. However, the low mass of the nuclear necessitates quantum treatment of the lattice degree of freedom (often beyond the harmonic approximation) in accurate simulations.

% what controversies are there in experiment?
Established experimental results on high-pressure hydrogen are limited in variety. At room temperature and below, diamond anvil cell (DAC) is the dominant apparatus to achieve such high pressures. Small size of the cell and fragility of the sample limit experimental probes to low-power optics such as infrared and Raman spectroscopy\cite{RangaI.F.2017}. Hydrogen is a weak scatterer of X-Rays~\cite{Zha2014}, thus excluding this excellent tool for structural determination in most experiments. Only recently has X-ray been performed up to 254 GPa~\cite{Akahama2010,Ji2019}.
At high temperatures, shock wave compression is the main method to achieve megabar pressures. Due to the transient nature of theses experiments, acquiring and analyzing shock-wave data is challenging. Most notably, one cannot directly measure temperature, which may cause misinterpretation of raw data~\cite{Celliers2018,Knudson2004,Knudson2017}.
Given the experimental difficulties, predictive simulations are highly desirable as they can inform and verify experiments~\cite{Pierleoni2016b}.

% what controversies are there in calculations?
Simulation of high-pressure hydrogen is challenging. Without experimental structural information from X-ray, many theoretical calculations have been performed on structures found in density functional theory (DFT) random structure searches~\cite{Pickard2007}. Constraint by computational cost, these searches are limited to Classical protons, causing the methods to miss, for example, saddle-point structures that can be stabilized by nuclear quantum effect~\cite{Monserrat2016}.
%Most theoretical studies of high-pressure hydrogen suffer a similar dilemma between accuracy and practicality.
%On the one hand,
Predictive simulations of hydrogen require accurate methods both in the description of the electronic ground-state Born-Oppenheimer (BO) potential energy surface (PES) and in the inclusion of nuclear quantum effect beyond the quasi-harmonic approximation.
%On the other hand, to explore such large pressure and temperature ranges using first-principle methods is not only time consuming, but also prone to other limitations such as inadequate treatment of finite-size effect, nuclear quantum effect, and simulation timescale. Further, some experimentally measurable properties, such as emission, absorption, infrared, and Raman spectra, heat conductivity, and diffusion constants are difficult to calculate in the most accurate electronic methods, e.g., DMC.
The popular Perdew-Burke-Ernzerhof (PBE) density functional (DF) in DFT erroneously predicts some molecular structures to be metallic~\cite{Drummond2015}. However, its use in conjunction with Classical molecular dynamics (MD) results in reasonable transition pressure for the LLT due to error cancellation~\cite{Morales2013a}.
This and other fortuitous cancellations of error has lead many to believe that the PBE functional provides a good description of solid hydrogen and caused much confusion in the community. PBE predicts a conductive molecular structure above 200 GPa, a molecular-to-atomic transition around 300 GPa~\cite{McMinis2015}, and low-temperature superconducting liquid. All these predictions contradict experimental evidence. Systematic benchmark of the PES from various DFT functionals against QMC found the vdW-DF1 functional to be the most accurate for molecular hydrogen~\cite{Clay2016}. However, this functional has yet to gain widespread adoption due to higher computational cost and lower popularity compared to PBE.

In this chapter, I will focus on the solid phases of hydrogen. Section~\ref{sec:hsoli-expt} summarizes experimental observations, while section~\ref{sec:hsoli-calcs} summarizes relevant computational studies.

\subsection{Experiments}
\label{sec:hsoli-expt}

As element number one with the simplest atomic structure, hydrogen has surprising complex phases at megabar pressures.
The phase diagram differs for the isotopes, e.g., hydrogen H, deuterium D, and spin isomers of molecular hydrogen.
We have para-hydrogen (p-H$_2$) when the proton spins anti-align to form a singlet, and ortho-hydrogen (o-H$_2$) when they align to form a triplet.
At low pressure ($<100$ GPa) and below the melting temperature, spherically symmetric p-H$_2$ form a hexagonal close packed (HCP) crystal, whereas o-H$_2$ has orientation order at ambient pressure.
At low temperature and pressure, ortho-para conversion takes days to reach equilibrium, so it is possible to have almost pure p-H$_2$ in experiment. In contrast, at higher temperature and pressure, free roton states mix so the total angular momentum of the H$_2$ molecule, $J$, is no longer a good quantum number. However, the ortho-para distinction can still be made based on intramolecular nucleon exchange~\cite{Silvera1998}. 
The vibration of p-$H_2$ molecules is both Raman- and IR-active. The Raman vibron displays a sharp ``vibron'' peak around 4200 cm$^{-1}$ at 100 GPa and decreasing roughly linearly to 4060 cm$^{-1}$ at 150 GPa, with a 3 cm$^{-1}$ jump at 110 GPa~\cite{Lorenzana1990}.
Transitions between the rotational states of the molecule can also be observed in Raman spectrum as ``roton bands'' around 350 cm$^{-1}$. As pressure increases, anisotropic intermolecular interactions mix the free roton states until the ground state distorts and molecules become orientationally ordered to lower potential energy. For p-H$_2$, this transition is observed above 110 GPa at 8 K. The roton bands broaden, indicating that the hydrogen molecules start to lose their spherical symmetry and become rotationally ordered. This broken-symmetry phase (BSP) or phase II, stable under 140 K, is considered a quantum phase based on observed strong isotope shift of the transition pressure. o-D$_2$ reaches the BSP phase at as low as 28 GPa at 1.1 K~\cite{Silvera1981}.
Phase II is stabilized by anisotropic intermolecular interactions, which increase with pressure and lower the energy the $J=2$ rotational state relative to that of the $J=0$ state.
Below 140 K and above 150 GPa, p-H$_2$ enters a different orientation-ordered H-A regardless of ortho-para content and isotope~\cite{Cui1995,Goncharov1998}.
In this new phase, the rather broad and pressure-independent roton band weakens, disappears, and is replaced by a few sharp and strongly pressure-dependent peaks in the frequency range 100$\sim$700 cm$^{-1}$~\cite{Goncharov1998}.
These new modes are considered to be lattice libration modes due to their pressure dependence.
Phases with natural concentrations of H$_2$ (n-H$_2$) are labeled I, II, and III~\cite{Dias2019}, which correspond to the HCP, BSP, and H-A phases of pure p-H$_2$, respectively.
At 300 K and above 220 GPa, we enter yet another solid phase IV, characterized by a splitting of the vibron peak~\cite{Zha2013}.
Both theory and experiment suggest that phase IV consists of alternating layers having rather different in-plane structures, possibly two types of molecules
At $<100$ K and above 350 GPa, molecular hydrogen becomes semi-metallic, possibly due to the closure of an indirect band gap~\cite{Eremets2019}.
Then, above 425 GPa, all IR radiation is absorbed indicating a closure of the direct band gap~\cite{Loubeyre2020}.
Finally, at sufficiently high pressures, the hydrogen molecules will dissociate to form an atomic solid, reportedly at 495 GPa~\cite{Silvera2017}. Although consensus has yet to be reached.

While the phase boundaries of solid hydrogen are reasonably well-established below 400 K and 400 GPa by diamond-anvil cell (DAC) experiments~\cite{Dias2019}, characterizations of the solid structures are limited. Due to small scattering cross section and small sample size in DAC experiments, only a handful of X-ray~\cite{Hazen1987,MAO1988,Loubeyre1996,Kawamura2002,Goncharenko2005a,Akahama2010,Ji2019} and only one neutron~\cite{Goncharenko2005a} scattering experiments haven been published over the past 40 years. Most of our understanding of solid hydrogen is built upon IR and Raman spectra, which provide partial information on the microscopic details of the solid structures. This lack of definitive structural information poses significant difficulty for both theoretical and experimental understanding of solid hydrogen. Experimentally, this has lead to the misidentification of a triple point as a critical point~\cite{Lorenzana1990,Cui1994}, subtlety in the detection of a new phase~\cite{Eremets2009,Howie2012}, among many debates over interpretation of optical data. One particularly challenging transition to detect is the melting of the molecular crystal.

\subsection{Calculations}
\label{sec:hsoli-calcs}

Early computational studies~\cite{Kaxiras1991,Nagara1992,Mazin1995,Kohanoff1997,Johnson2000} of solid hydrogen rely on assumed crystal structures based on symmetry considerations and the local density approximation (LDA) for density functional. This situation changed in 2007 when random structure searching found new candidate crystal structures that have lower enthalpy than all previously proposals~\cite{Pickard2007}. Layered structures having C2/c and Cmca symmetries became the main candidates for phase III.
Three diffusion Monte Carlo studies followed to characterize the candidate structures by Azadi \textit{et al.}~\cite{Azadi2014}, McMinis \textit{et al.}~\cite{McMinis2015}, and Drummond \textit{et al.}~\cite{Drummond2015}.
Azadi used PBE-optimized geometries and included anharmonic phonon zero-point energy and predicted molecular dissociation at 374 GPa, from Cmca-12 to I4$_1$/amd. In contrast, McMinis used vdW-DF-optimized geometries and harmonic phonon zero-point energy and predicted a dissociation pressure of 447(3) GPa.
In hind sight, the McMinis prediction is in better agreement of experiments performed after the calculation.

At the low pressure side, a new hexagonal candidate structure for phase III was proposed by Monserrat \textit{et al.}~\cite{Monserrat2016}, then calculated to be more stable than C2/c below 210 GPa~\cite{Azadi2019}.

Bandgap of the C2/c structure shows closure around 460 GPa, when extrapolated from  was IR measurements up to 420 GPa~\cite{Loubeyre2020}. Most recent DMC calculation of the bandgap is in excellent agreement with experiment~\cite{Gorelov2019}. This new calculation is at variance with the previous prediction by Azadi \textit{et al.}~\cite{Azadi2019}, presumably due to different treatments of finite-size effect.

Finally, a recent coupled cluster calculation of the molecular candidate structures show good agreement with DMC results~\cite{Liao2019} at the static lattice level. Although, lattice zero-point energy has yet to be included.


%\subsection{The melting transition and critical point}
%
%The melting temperature of solid hydrogen is typically determined in one of three ways: 1. visual observation of motion due to sample, contaminant, or laser speckle~\cite{Gregoryanz2003}, 2. discontinuous change in the Raman spectrum, e.g. vibron frequency jump, disappearance of librons~\cite{Gregoryanz2003,Subramanian2011,Zha2017}, 3. plateau temperature during laser heating~\cite{Deemyad2008}, and 4. rapid change in interferences pattern~\cite{Eremets2009}.
%As sample size decreases with pressure, visual detection of melting becomes more difficult. Further, the refractive indexes of hydrogen and diamond approach each other around 140 GPa, defeating detection by interference pattern~\cite{Zha2017}.
%The vibron discontinuity caused by melting reduces from $\sim$ 3 cm$^{-1}$ at 10 GPa to $\sim$ 1 cm$^{-1}$ at 45 GPa~\cite{Gregoryanz2003}, making melting detection more difficult at higher pressures.
%Finally, disappearance of librons alone cannot be taken as proof of melting, because it could be due to loss of sample.
%Due to these difficulties, the precise melting temperatures above 100 GPa are debate. Although, there is broad consensus that the melting line increases from 200 K at ambient pressure to $\sim$ 1000 K at 90 GPa, then decreases with increasing pressure.
%
%Melting of $H_2$ solid around 150 GPa is particularly interesting. Shock wave compression data indicate a molecular liquid to atomic liquid transition $\sim$ 1000 K at 150 GPa. Thus, $H_2$ molecules should dissociate almost immediately after the solid melts. Eremets and Trojan observed a shark drop in measured resistivity as the $H_2$ solid melts at 146 GPa~\cite{Eremets2009}. This observation is consistent with the current liquid-liquid transition (LLT) line.

%\subsection{Liquid phases and the liquid-liquid transition}

\section{Methods}
\subsection{Geometry Optimization}
% hsolid/11-refine-struct/a-ecut
The molecular structures are optimized in DFT using the vdW-DF functional. We use quantum ESPRESSO v5.3.0 to perform variable-cell geometry optimization at constant pressure. The atomic positions in the optimized unit cell are re-optimized at constant-volume. We use a Troullier-Martins pseudopotential with a core cutoff radius of $r_c=0.5$. The plane-wave cutoff energy is set to 160 Ry. Brillouin zone integration is performed using a shifted Monkhorst-Pack grid with $24^3$, $16^3$, $12^3$ points for the Cmca-4, Cmca-12, and C2/c-24 unit cells, respectively. %The effective number of atoms are 55296, 49152, and 41472, respectively.
Pressure is converged to 0.1 kbar (0.01 GPa). % Forces are converged to

The atomic structure is optimized in DMC. At each density, the $c/a$ parameter determines the I4$_1$/amd-4 ($c/a>1$) crystal structure. To optimize the $c/a$ parameter, we performed 5 DMC calculations at each density. These calculations form a grid in the lattice a-c parameter space as shown in Fig.~\ref{fig:i4-rs-ca}. Please see QMC section for details of the DMC calculations.
\begin{figure}[h]
% 2018-02-01_ani-press
\includegraphics[width=0.8\columnwidth]{70_i4_ca_grid}
\caption{DMC calculations performed to optimize the atomic hydrogen solid structure. Each dot is a structure defined by the lattice parameters a and c. The color of each dot indicates the DMC energy. The gray contour lines mark structures with constant density or c/a ratio. Energy variation is dominated by density change. Energy variation in the c/a direction at fixed density is roughly quadratic around its minimum (black star). The black stars are the optimized geometries.\label{fig:i4-rs-ca}}.
\end{figure}
\subsection{Supercell}
All quantum Monte Carlo (QMC) calculations are performed using 72-atom simulation cells. Each simulation cell is tiled from the optimized unit cell using a supercell matrix.
\begin{align}
A_s = S A \Rightarrow \left(\begin{array}{c}
\bs{a}_s \\
\bs{b}_s \\
\bs{c}_s
\end{array}\right) = S\left(\begin{array}{c}
\bs{a} \\
\bs{b} \\
\bs{c}
\end{array}\right),
\end{align}
where $\bs{a}$, $\bs{b}$, $\bs{c}$ are the lattice vectors of the unit cell. $\bs{a}_s$, $\bs{b}_s$, $\bs{c}_s$ are the lattice vectors of the simulation cell. $S$ is the supercell matrix. The supercell matrices are chosen to maximize the simulation cell radius. %Ideally, one should maximize the image radius $R_{WS}$ instead.
\begin{table}[h]
\caption{Supercell matrices.}
\begin{tabular}{cccc}
\hline\hline
Cmca-4 & Cmca-12 & C2/c-24 & I4$_1$/amd \\
\hline
$\left(\begin{array}{ccc}
 3 &  3 &  0 \\
-1 &  2 &  1 \\
 2 & -1 &  1
\end{array}\right)$ & $\left(\begin{array}{ccc}
 2 &  1 &  -1 \\
-1 &  1 &  0 \\
 2 &  1 &  1
\end{array}\right)$ & $\left(\begin{array}{ccc}
 2 &  1 &  0 \\
 1 &  2 &  0 \\
 0 &  0 &  1
\end{array}\right)$ & $\left(\begin{array}{ccc}
 2 & -2 &  1 \\
 2 &  3 &  0 \\
-2 &  1 &  1
\end{array}\right)$ \\
\hline\hline
\end{tabular}
\end{table}

In Fig.~\ref{fig:cell-radius}, the image radius (a.k.a. radius of the real-space Wigner-Seitz cell) $R_{WS}$ and simulation cell radius $R_{sc}$ are shown as a function of density.

\begin{figure}[h]
\includegraphics[width=0.8\columnwidth]{101a1_cell-radius}
\caption{Supercell radius as a function of density. $r_s$ is the Wigner-Seitz radius, which is determined by the average electron density $\frac{4\pi}{3}r_s=\rho$, where $\rho=N_e/\Omega$, with $\Omega$ the supercell volume. $R_{WS}$ is the radius of the real-space Wigner-Seitz cell of the supercell. $2R_{WS}$ is the minimum distance between periodic images. \label{fig:cell-radius}}
\end{figure}

\subsection{Wavefunction}

All QMC calculations are performed using the QMCPACK code. We use Slater-Jastrow-Backflow (SJB) wavefunction. The orbitals in the Slater determinant are cusp-corrected DFT orbitals. The vdw-DF functional is used to generate orbitals for the molecular structures, whereas the PBE functional is used for the atomic structure. The orbital generating DFT runs have different settings compared to the geometry optimization runs.

To generate the orbitals, we perform DFT directly in the supercell. All calculations use the bare Coulomb interaction and a plane wave cutoff of 50 Ry. First, we run a self-consistent calculation to converge the charge density on a shifted $8^3$ Monkhorst-Pack grid. Second, we run a non-self-consistent calculation on an unshifted Monkhorst-Pack grid to generate the orbitals needed by all twists. $4^3$ twists are used for the molecular phase, while $6^3$ twists are used for the atomic phase. Finally, we divide each orbital by an electron-ion Jastrow wavefunction to remove the electron-ion cusp from the orbital. This electron-ion Jastrow wavefunction is constructed using Fourier components commensurate with the simulation cell (i.e. on the reciprocal lattice vectors of the simulation cell $\bs{G}_s$)
\begin{align}
J_{ei}(\bs{r}_j; \bs{R}) \propto& \exp\left\{ \text{iFFT}\left[ 
U_{\bs{k}}^{ep}  \left(\sum\limits_{J=1}^{N_p} \frac{e^{-i\bs{k}\cdot\bs{R}_J}}{N_p}  \right)
\right] \right\} \nonumber\\
\propto& \exp\left\{ 
\sum\limits_{\bs{k}\neq\bs{0}}^{\bs{k}\in\bs{G}_s} e^{i\bs{k}\cdot\bs{r}_j}~
U_{\bs{k}}^{ep} 
\left(\sum\limits_{J=1}^{N_p} \frac{e^{-i\bs{k}\cdot\bs{R}_J}}{N_p}  \right)
\right\},\label{eq:rpa-ep-jas}
\end{align}
where $\bs{r}_j$ is any single electron coordinate. $\bs{R}$ contains all ionic coordinates. $N_p$ is the number of protons. iFFT stands for ``inverse fast Fourier transform''. The Jastrow potential $U_{\bs{k}}^{ep}$ in eq.~(\ref{eq:rpa-ep-jas}) is chosen to be the RPA form written by Ceperley and Alder
\begin{align}
2U^{ep}_k = -a_k(1+a_k)^{-1/2},
\end{align}
where $a_k=\frac{12}{r_s^3k^4}$ in Hartree atomic units. $r_s$ is the Wigner-Seitz radius.

When a single-particle orbital is divided by eq.~(\ref{eq:rpa-ep-jas}), the electron-ion cusp is removed from the orbital. We re-introduce the electron-ion cusp in the Jastrow part of the trial wavefunction. 

There are 48 optimizable parameters in our wavefunction. We use short-range B-spline Jastrow pair potentials which are smoothly cut off at $R_{WS}$ (the image radius). There are three Jastrow potentials (uu, ud, ep) between up and up electrons, up and down electrons, electron and proton. We use short-range B-spline back flow transformation functions which are smoothly cut off $R_{WS}$. There are three backflow correlation functions (uu, ud, ep) similar to the Jastrow setup. Each B-spline has 8 optimizable knots.
\begin{comment}
\subsection{QMC Data}

At each density, we perform one VMC and two DMC calculations. Each QMC calculation is labeled by a series index. The VMC calculation is series 0. The first DMC calculation with a relatively large time step is series 1. The second DMC calculation with a relatively small time step is series 2. We post-process the raw results (series 0 - 2) to produce series 3 and 4. We linearly extrapolate the DMC results (series 1, 2) to zero time step and label the results series 3. We linearly extrapolate the VMC and the t=0 DMC results (series 0, 3) to obtain pure-estimator kinetic and potential energies and label them series 4.

Twist-average QMC energies are displayed in the following table. The dUlr column contains the many-body finite size correction which will be described in the next section.

\begin{table}[h]
\small
\begin{tabular}{llrrrllll}
\toprule
         &   &  timestep &  natom &      dUlr & LocalEnergy\_pp &    Variance\_pp &     Kinetic\_pp &    Potential\_pp \\
rs & series &           &        &           &                &                &                &                 \\
\midrule
1.163891 & 0 &    0.0300 &     72 &  0.005586 &    -0.47572(1) &     0.01380(2) &      0.9819(1) &      -1.4576(1) \\
         & 1 &    0.0030 &     72 &  0.005417 &   -0.477138(9) &    0.013719(9) &     0.98177(9) &     -1.45890(8) \\
         & 2 &    0.0015 &     72 &  0.005404 &    -0.47712(1) &     0.01374(1) &      0.9822(1) &    -1.45933(10) \\
         & 3 &    0.0000 &     72 &  0.005391 &    -0.47711(2) &     0.01374(1) &      0.9827(2) &      -1.4598(2) \\
         & 4 &    0.0000 &     72 &  0.005213 &    -0.47711(2) &     0.01374(1) &      0.9834(2) &      -1.4598(2) \\
1.182675 & 0 &    0.0300 &     72 &  0.005455 &    -0.48354(1) &     0.01347(2) &      0.9581(1) &      -1.4416(1) \\
         & 1 &    0.0030 &     72 &  0.005269 &   -0.484951(9) &    0.013408(8) &     0.95828(9) &     -1.44323(9) \\
         & 2 &    0.0015 &     72 &  0.005287 &    -0.48496(1) &     0.01342(1) &      0.9585(1) &      -1.4434(1) \\
         & 3 &    0.0000 &     72 &  0.005305 &    -0.48496(2) &     0.01342(1) &      0.9587(2) &      -1.4436(2) \\
         & 4 &    0.0000 &     72 &  0.005170 &    -0.48496(2) &     0.01342(1) &      0.9592(2) &      -1.4436(2) \\
1.195717 & 0 &    0.0300 &     72 &  0.005373 &   -0.488647(6) &    0.012795(9) &     0.94365(7) &     -1.43230(7) \\
         & 1 &    0.0030 &     72 &  0.005188 &   -0.490019(9) &    0.012688(8) &     0.94357(9) &     -1.43361(9) \\
         & 2 &    0.0015 &     72 &  0.005183 &   -0.490017(9) &    0.012707(9) &     0.94391(9) &     -1.43394(9) \\
         & 3 &    0.0000 &     72 &  0.005178 &    -0.49002(2) &    0.012707(9) &      0.9443(2) &      -1.4343(2) \\
         & 4 &    0.0000 &     72 &  0.004997 &    -0.49002(2) &    0.012707(9) &      0.9449(2) &      -1.4343(2) \\
1.221845 & 0 &    0.0300 &     72 &  0.005135 &   -0.497976(6) &    0.012288(9) &     0.91589(7) &     -1.41386(7) \\
         & 1 &    0.0030 &     72 &  0.004973 &   -0.499383(8) &    0.012232(9) &     0.91531(9) &     -1.41469(9) \\
         & 2 &    0.0015 &     72 &  0.004998 &   -0.499352(8) &    0.012255(9) &     0.91567(9) &     -1.41501(9) \\
         & 3 &    0.0000 &     72 &  0.005023 &    -0.49932(2) &    0.012255(9) &      0.9160(2) &      -1.4153(2) \\
         & 4 &    0.0000 &     72 &  0.004922 &    -0.49932(2) &    0.012255(9) &      0.9162(2) &      -1.4153(2) \\
1.235307 & 0 &    0.0300 &     72 &  0.005067 &   -0.502414(6) &     0.01188(1) &     0.90195(6) &     -1.40436(7) \\
         & 1 &    0.0030 &     72 &  0.004900 &   -0.503809(8) &    0.011814(8) &     0.90131(9) &     -1.40508(9) \\
         & 2 &    0.0015 &     72 &  0.004908 &   -0.503783(9) &    0.011793(9) &     0.90181(9) &     -1.40559(9) \\
         & 3 &    0.0000 &     72 &  0.004915 &    -0.50376(2) &    0.011793(9) &      0.9023(2) &      -1.4061(2) \\
         & 4 &    0.0000 &     72 &  0.004776 &    -0.50376(2) &    0.011793(9) &      0.9027(2) &      -1.4061(2) \\
1.249707 & 0 &    0.0300 &     72 &  0.004887 &   -0.506825(6) &     0.01274(2) &     0.88739(7) &     -1.39422(7) \\
         & 1 &    0.0030 &     72 &  0.004748 &   -0.508268(9) &    0.012682(9) &      0.8869(1) &     -1.39525(9) \\
         & 2 &    0.0015 &     72 &  0.004765 &   -0.508273(9) &    0.012696(9) &     0.88747(9) &     -1.39574(9) \\
         & 3 &    0.0000 &     72 &  0.004783 &    -0.50828(2) &    0.012696(9) &      0.8880(2) &      -1.3962(2) \\
         & 4 &    0.0000 &     72 &  0.004690 &    -0.50828(2) &    0.012696(9) &      0.8886(2) &      -1.3962(2) \\
1.265425 & 0 &    0.0300 &     72 &  0.004870 &   -0.511407(6) &     0.01158(1) &     0.87289(6) &     -1.38430(7) \\
         & 1 &    0.0030 &     72 &  0.004684 &   -0.512853(8) &    0.011519(8) &     0.87292(9) &     -1.38577(9) \\
         & 2 &    0.0015 &     72 &  0.004697 &   -0.512872(8) &    0.011541(8) &     0.87333(9) &     -1.38621(9) \\
         & 3 &    0.0000 &     72 &  0.004710 &    -0.51289(2) &    0.011541(8) &      0.8738(2) &      -1.3866(2) \\
         & 4 &    0.0000 &     72 &  0.004565 &    -0.51289(2) &    0.011541(8) &      0.8746(2) &      -1.3866(2) \\
1.283017 & 0 &    0.0300 &     72 &  0.004733 &   -0.516220(6) &    0.011305(9) &     0.85757(7) &     -1.37379(7) \\
         & 1 &    0.0030 &     72 &  0.004574 &   -0.517656(8) &    0.011240(9) &     0.85753(9) &     -1.37518(9) \\
         & 2 &    0.0015 &     72 &  0.004575 &   -0.517673(9) &    0.011251(8) &     0.85793(9) &     -1.37560(9) \\
         & 3 &    0.0000 &     72 &  0.004575 &    -0.51769(2) &    0.011251(8) &      0.8583(2) &      -1.3760(2) \\
         & 4 &    0.0000 &     72 &  0.004430 &    -0.51769(2) &    0.011251(8) &      0.8591(2) &      -1.3760(2) \\
1.302685 & 0 &    0.0300 &     72 &  0.004606 &   -0.521222(6) &    0.010863(8) &     0.84146(7) &     -1.36269(7) \\
         & 1 &    0.0030 &     72 &  0.004447 &   -0.522642(8) &    0.010795(7) &     0.84153(8) &     -1.36415(8) \\
         & 2 &    0.0015 &     72 &  0.004452 &   -0.522665(8) &    0.010810(8) &     0.84172(8) &     -1.36439(8) \\
         & 3 &    0.0000 &     72 &  0.004457 &    -0.52269(2) &    0.010810(8) &      0.8419(2) &      -1.3646(2) \\
         & 4 &    0.0000 &     72 &  0.004322 &    -0.52269(2) &    0.010810(8) &      0.8424(2) &      -1.3646(2) \\
\bottomrule
\end{tabular}

\caption{Cmca-4}
\end{table}

\begin{table}[h]
\small
\begin{tabular}{llrrrllll}
\toprule
         &   &  timestep &  natom &      dUlr & LocalEnergy\_pp &     Variance\_pp &     Kinetic\_pp &   Potential\_pp \\
rs & series &           &        &           &                &                 &                &                \\
\midrule
1.223839 & 0 &    0.0300 &     72 &  0.005398 &   -0.498582(6) &     0.013259(9) &     0.91712(7) &    -1.41570(7) \\
         & 1 &    0.0030 &     72 &  0.005121 &   -0.500194(9) &     0.013059(9) &     0.91759(9) &    -1.41781(9) \\
         & 2 &    0.0015 &     72 &  0.005123 &   -0.500189(9) &     0.013063(9) &     0.91791(9) &    -1.41810(9) \\
         & 3 &    0.0000 &     72 &  0.005125 &    -0.50018(2) &     0.013063(9) &      0.9182(2) &     -1.4184(2) \\
         & 4 &    0.0000 &     72 &  0.004868 &    -0.50018(2) &     0.013063(9) &      0.9193(2) &     -1.4184(2) \\
1.251971 & 0 &    0.0300 &     72 &  0.005113 &   -0.507671(7) &     0.013093(9) &     0.89092(7) &    -1.39859(7) \\
         & 1 &    0.0030 &     72 &  0.004888 &   -0.509262(8) &     0.012973(9) &     0.89079(8) &    -1.40004(8) \\
         & 2 &    0.0015 &     72 &  0.004888 &   -0.509253(8) &     0.012967(9) &     0.89147(9) &    -1.40072(9) \\
         & 3 &    0.0000 &     72 &  0.004888 &    -0.50924(2) &     0.012967(9) &      0.8922(2) &     -1.4014(2) \\
         & 4 &    0.0000 &     72 &  0.004676 &    -0.50924(2) &     0.012967(9) &      0.8934(2) &     -1.4014(2) \\
1.268116 & 0 &    0.0300 &     72 &  0.005012 &   -0.512485(6) &     0.011951(8) &     0.87796(7) &    -1.39044(7) \\
         & 1 &    0.0030 &     72 &  0.004777 &   -0.514038(9) &     0.011822(9) &     0.87737(8) &    -1.39142(8) \\
         & 2 &    0.0015 &     72 &  0.004761 &   -0.514071(9) &     0.011820(9) &     0.87791(9) &    -1.39197(9) \\
         & 3 &    0.0000 &     72 &  0.004745 &    -0.51410(2) &     0.011820(9) &      0.8785(2) &     -1.3925(2) \\
         & 4 &    0.0000 &     72 &  0.004490 &    -0.51410(2) &     0.011820(9) &      0.8789(2) &     -1.3925(2) \\
1.286021 & 0 &    0.0300 &     72 &  0.004894 &    -0.51743(1) &      0.01155(1) &      0.8626(1) &     -1.3800(1) \\
         & 1 &    0.0030 &     72 &  0.004676 &   -0.518992(8) &     0.011411(9) &     0.86249(9) &    -1.38150(9) \\
         & 2 &    0.0015 &     72 &  0.004645 &    -0.51902(1) &    0.011398(10) &      0.8624(1) &     -1.3814(1) \\
         & 3 &    0.0000 &     72 &  0.004614 &    -0.51905(2) &    0.011398(10) &      0.8623(2) &     -1.3813(2) \\
         & 4 &    0.0000 &     72 &  0.004351 &    -0.51905(2) &    0.011398(10) &      0.8620(2) &     -1.3813(2) \\
1.306029 & 0 &    0.0300 &     72 &  0.004704 &   -0.522568(6) &      0.01157(2) &     0.84694(7) &    -1.36950(7) \\
         & 1 &    0.0030 &     72 &  0.004498 &   -0.524120(9) &     0.011457(9) &     0.84641(8) &    -1.37052(8) \\
         & 2 &    0.0015 &     72 &  0.004515 &   -0.524133(8) &     0.011450(8) &     0.84684(9) &    -1.37098(9) \\
         & 3 &    0.0000 &     72 &  0.004531 &    -0.52415(2) &     0.011450(8) &      0.8473(2) &     -1.3714(2) \\
         & 4 &    0.0000 &     72 &  0.004370 &    -0.52415(2) &     0.011450(8) &      0.8476(2) &     -1.3714(2) \\
\bottomrule
\end{tabular}

\caption{Cmca-12}
\end{table}

\begin{table}[h]
\small
\begin{tabular}{llrrrllll}
\toprule
         &   &  timestep &  natom &      dUlr & LocalEnergy\_pp &     Variance\_pp &     Kinetic\_pp &   Potential\_pp \\
rs & series &           &        &           &                &                 &                &                \\
\midrule
1.184674 & 0 &    0.0300 &     72 &  0.005670 &    -0.48422(1) &      0.01254(2) &      0.9616(1) &     -1.4459(1) \\
         & 1 &    0.0030 &     72 &  0.005413 &   -0.485676(9) &    0.012399(10) &     0.96117(9) &    -1.44684(9) \\
         & 2 &    0.0015 &     72 &  0.005411 &    -0.48568(1) &    0.012389(10) &      0.9616(1) &     -1.4473(1) \\
         & 3 &    0.0000 &     72 &  0.005409 &    -0.48569(2) &    0.012389(10) &      0.9620(2) &     -1.4477(2) \\
         & 4 &    0.0000 &     72 &  0.005167 &    -0.48569(2) &    0.012389(10) &      0.9624(2) &     -1.4477(2) \\
1.198103 & 0 &    0.0300 &     72 &  0.005490 &    -0.48946(1) &      0.01307(2) &      0.9490(1) &     -1.4384(1) \\
         & 1 &    0.0030 &     72 &  0.005271 &   -0.490925(8) &     0.012934(9) &     0.94809(9) &    -1.43904(8) \\
         & 2 &    0.0015 &     72 &  0.005273 &    -0.49093(1) &      0.01295(1) &      0.9487(1) &     -1.4396(1) \\
         & 3 &    0.0000 &     72 &  0.005275 &    -0.49093(2) &      0.01295(1) &      0.9493(2) &     -1.4402(2) \\
         & 4 &    0.0000 &     72 &  0.005080 &    -0.49093(2) &      0.01295(1) &      0.9496(2) &     -1.4402(2) \\
1.225107 & 0 &    0.0300 &     72 &  0.005412 &   -0.499215(7) &     0.011982(8) &     0.92039(7) &    -1.41960(7) \\
         & 1 &    0.0030 &     72 &  0.005135 &   -0.500680(9) &     0.011825(8) &     0.91983(9) &    -1.42051(9) \\
         & 2 &    0.0015 &     72 &  0.005138 &   -0.500694(9) &     0.011830(8) &     0.92029(9) &    -1.42099(9) \\
         & 3 &    0.0000 &     72 &  0.005140 &    -0.50071(2) &     0.011830(8) &      0.9207(2) &     -1.4215(2) \\
         & 4 &    0.0000 &     72 &  0.004886 &    -0.50071(2) &     0.011830(8) &      0.9211(2) &     -1.4215(2) \\
1.253286 & 0 &    0.0300 &     72 &  0.005220 &   -0.508316(6) &      0.01170(1) &     0.89260(7) &    -1.40092(7) \\
         & 1 &    0.0030 &     72 &  0.004942 &   -0.509765(9) &      0.01152(1) &     0.89257(8) &    -1.40232(8) \\
         & 2 &    0.0015 &     72 &  0.004929 &   -0.509798(8) &     0.011498(8) &      0.8929(1) &    -1.40282(9) \\
         & 3 &    0.0000 &     72 &  0.004917 &    -0.50983(2) &     0.011498(8) &      0.8933(2) &     -1.4033(2) \\
         & 4 &    0.0000 &     72 &  0.004634 &    -0.50983(2) &     0.011498(8) &      0.8940(2) &     -1.4033(2) \\
1.269426 & 0 &    0.0300 &     72 &  0.005052 &   -0.513065(6) &     0.011097(9) &     0.87708(7) &    -1.39014(7) \\
         & 1 &    0.0030 &     72 &  0.004806 &   -0.514529(8) &     0.010948(8) &     0.87756(9) &    -1.39209(9) \\
         & 2 &    0.0015 &     72 &  0.004799 &   -0.514517(8) &     0.010939(8) &     0.87786(9) &    -1.39238(9) \\
         & 3 &    0.0000 &     72 &  0.004792 &    -0.51451(2) &     0.010939(8) &      0.8782(2) &     -1.3927(2) \\
         & 4 &    0.0000 &     72 &  0.004549 &    -0.51451(2) &     0.010939(8) &      0.8792(2) &     -1.3927(2) \\
1.287311 & 0 &    0.0300 &     72 &  0.004946 &   -0.518053(6) &     0.010534(8) &     0.86353(7) &    -1.38158(7) \\
         & 1 &    0.0030 &     72 &  0.004690 &   -0.519484(9) &     0.010371(8) &     0.86319(9) &    -1.38267(9) \\
         & 2 &    0.0015 &     72 &  0.004683 &   -0.519499(9) &     0.010369(7) &     0.86374(8) &    -1.38325(9) \\
         & 3 &    0.0000 &     72 &  0.004675 &    -0.51951(2) &     0.010369(7) &      0.8643(2) &     -1.3838(2) \\
         & 4 &    0.0000 &     72 &  0.004423 &    -0.51951(2) &     0.010369(7) &      0.8651(2) &     -1.3838(2) \\
1.307314 & 0 &    0.0300 &     72 &  0.004801 &    -0.52318(1) &      0.01054(2) &      0.8471(1) &     -1.3703(1) \\
         & 1 &    0.0030 &     72 &  0.004552 &   -0.524617(8) &     0.010398(8) &     0.84732(9) &    -1.37196(9) \\
         & 2 &    0.0015 &     72 &  0.004552 &    -0.52464(1) &      0.01039(1) &      0.8478(1) &     -1.3724(1) \\
         & 3 &    0.0000 &     72 &  0.004552 &    -0.52466(2) &      0.01039(1) &      0.8483(2) &     -1.3729(2) \\
         & 4 &    0.0000 &     72 &  0.004322 &    -0.52466(2) &      0.01039(1) &      0.8494(2) &     -1.3729(2) \\
\bottomrule
\end{tabular}

\caption{C2/c-24}
\end{table}

\begin{table}[h]
\small
\begin{tabular}{llrrrllll}
\toprule
     &   &  timestep &  natom &      dUlr &  LocalEnergy\_pp &     Variance\_pp &      Kinetic\_pp &   Potential\_pp \\
rs & series &           &        &           &                 &                 &                 &                \\
\midrule
1.15 & 0 &    0.0300 &     72 &  0.006282 &    -0.468339(9) &      0.02039(2) &      0.97451(7) &    -1.44285(7) \\
     & 1 &    0.0030 &     72 &  0.005793 &     -0.47074(1) &      0.01978(1) &      0.97496(9) &    -1.44572(9) \\
     & 2 &    0.0015 &     72 &  0.005797 &     -0.47073(1) &      0.01980(1) &      0.97534(9) &    -1.44607(9) \\
     & 3 &    0.0000 &     72 &  0.005800 &     -0.47072(3) &      0.01980(1) &       0.9757(2) &     -1.4464(2) \\
     & 4 &    0.0000 &     72 &  0.005371 &     -0.47072(3) &      0.01980(1) &       0.9769(2) &     -1.4464(2) \\
1.17 & 0 &    0.0300 &     72 &  0.006168 &    -0.476700(9) &      0.02044(2) &      0.94912(7) &    -1.42582(8) \\
     & 1 &    0.0030 &     72 &  0.005678 &    -0.479108(9) &      0.01981(1) &      0.94928(7) &    -1.42839(7) \\
     & 2 &    0.0015 &     72 &  0.005659 &    -0.479149(9) &      0.01981(1) &      0.94962(7) &    -1.42876(7) \\
     & 3 &    0.0000 &     72 &  0.005639 &     -0.47919(2) &      0.01981(1) &       0.9499(2) &     -1.4291(2) \\
     & 4 &    0.0000 &     72 &  0.005158 &     -0.47919(2) &      0.01981(1) &       0.9508(2) &     -1.4291(2) \\
1.19 & 0 &    0.0300 &     72 &  0.005965 &   -0.484386(10) &      0.02117(2) &      0.92299(7) &    -1.40738(7) \\
     & 1 &    0.0030 &     72 &  0.005520 &     -0.48679(1) &      0.02062(2) &      0.92317(9) &    -1.40994(9) \\
     & 2 &    0.0015 &     72 &  0.005508 &     -0.48683(1) &      0.02062(2) &      0.92352(9) &    -1.41035(9) \\
     & 3 &    0.0000 &     72 &  0.005497 &     -0.48687(3) &      0.02062(2) &       0.9239(2) &     -1.4108(2) \\
     & 4 &    0.0000 &     72 &  0.005077 &     -0.48687(3) &      0.02062(2) &       0.9248(2) &     -1.4108(2) \\
1.21 & 0 &    0.0300 &     72 &  0.005837 &     -0.49145(1) &       0.0197(2) &       0.9000(1) &     -1.3914(1) \\
     & 1 &    0.0030 &     72 &  0.005390 &   -0.493853(10) &      0.01908(1) &      0.89966(7) &    -1.39351(7) \\
     & 2 &    0.0015 &     72 &  0.005394 &   -0.493893(10) &      0.01908(1) &      0.89987(7) &    -1.39376(7) \\
     & 3 &    0.0000 &     72 &  0.005398 &     -0.49393(2) &      0.01908(1) &       0.9001(2) &     -1.3940(2) \\
     & 4 &    0.0000 &     72 &  0.005002 &     -0.49393(2) &      0.01908(1) &       0.9002(2) &     -1.3940(2) \\
1.23 & 0 &    0.0300 &     72 &  0.005709 &     -0.49788(1) &      0.01952(6) &       0.8772(1) &     -1.3751(1) \\
     & 1 &    0.0030 &     72 &  0.005260 &    -0.500351(9) &      0.01906(1) &      0.87668(7) &    -1.37703(7) \\
     & 2 &    0.0015 &     72 &  0.005243 &    -0.500363(9) &      0.01905(1) &      0.87711(7) &    -1.37748(7) \\
     & 3 &    0.0000 &     72 &  0.005226 &     -0.50038(2) &      0.01905(1) &       0.8775(2) &     -1.3779(2) \\
     & 4 &    0.0000 &     72 &  0.004781 &     -0.50038(2) &      0.01905(1) &       0.8779(2) &     -1.3779(2) \\
1.25 & 0 &    0.0300 &     72 &  0.005701 &     -0.50380(1) &      0.01725(2) &       0.8546(1) &     -1.3584(1) \\
     & 1 &    0.0030 &     72 &  0.005192 &    -0.506311(9) &    0.016644(10) &      0.85476(7) &    -1.36107(7) \\
     & 2 &    0.0015 &     72 &  0.005195 &    -0.506295(9) &    0.016635(10) &      0.85502(7) &    -1.36131(7) \\
     & 3 &    0.0000 &     72 &  0.005199 &     -0.50628(2) &    0.016635(10) &       0.8553(2) &     -1.3615(2) \\
     & 4 &    0.0000 &     72 &  0.004741 &     -0.50628(2) &    0.016635(10) &       0.8560(2) &     -1.3615(2) \\
1.27 & 0 &    0.0300 &     72 &  0.005504 &     -0.50918(1) &      0.01858(2) &       0.8365(1) &     -1.3457(1) \\
     & 1 &    0.0030 &     72 &  0.005037 &    -0.511700(9) &      0.01807(1) &      0.83566(7) &    -1.34736(7) \\
     & 2 &    0.0015 &     72 &  0.005031 &   -0.511705(10) &      0.01808(1) &      0.83584(7) &    -1.34755(7) \\
     & 3 &    0.0000 &     72 &  0.005025 &     -0.51171(2) &      0.01808(1) &       0.8360(2) &     -1.3477(2) \\
     & 4 &    0.0000 &     72 &  0.004588 &     -0.51171(2) &      0.01808(1) &       0.8355(2) &     -1.3477(2) \\
1.29 & 0 &    0.0300 &     72 &  0.005491 &     -0.51409(1) &      0.01634(2) &     0.81539(10) &     -1.3295(1) \\
     & 1 &    0.0030 &     72 &  0.004980 &   -0.516579(10) &      0.01577(1) &      0.81540(7) &    -1.33198(7) \\
     & 2 &    0.0015 &     72 &  0.004968 &    -0.516601(9) &      0.01579(1) &      0.81556(7) &    -1.33216(7) \\
     & 3 &    0.0000 &     72 &  0.004955 &     -0.51662(2) &      0.01579(1) &       0.8157(2) &     -1.3323(2) \\
     & 4 &    0.0000 &     72 &  0.004467 &     -0.51662(2) &      0.01579(1) &       0.8160(2) &     -1.3323(2) \\
1.31 & 0 &    0.0300 &     72 &  0.005418 &     -0.51851(1) &      0.01569(2) &       0.7983(1) &     -1.3168(1) \\
     & 1 &    0.0030 &     72 &  0.004889 &    -0.521027(9) &    0.015035(10) &      0.79760(7) &    -1.31862(7) \\
     & 2 &    0.0015 &     72 &  0.004880 &    -0.521046(9) &     0.015035(9) &      0.79801(7) &    -1.31906(7) \\
     & 3 &    0.0000 &     72 &  0.004871 &     -0.52106(2) &     0.015035(9) &       0.7984(2) &     -1.3195(2) \\
     & 4 &    0.0000 &     72 &  0.004371 &     -0.52106(2) &     0.015035(9) &       0.7985(2) &     -1.3195(2) \\
\bottomrule
\end{tabular}

\caption{I4$_1$/amd}
\end{table}

\subsection{Many-body Finite Size Correction}

We use the fluctuating structure factor $\delta S(\bs{k})\equiv \braket{(\rho_{\bs{k}}-\bar{\rho}_{\bs{k}})(\rho_{-\bs{k}}-\bar{\rho}_{-\bs{k}})}$ to calculate many-body finite size correction (FSC) to the potential energy. The integrand is cut off using optimized long-range Coulomb potential in reciprocal space $v_k^{lr}$
\begin{align}
\Delta V^{lr} = \left[\int - \sum\right] \frac{1}{2}v^{lr}_k \delta S(\bs{k}).
\end{align}
Total energy FSC ($\Delta E$) is a sum of kinetic ($\Delta T$) and potential ($\Delta V$) corrections regardless of whether mixed-estimator (m) or pure-estimator (p) is used
\begin{align}
\Delta E = \Delta T_m + \Delta V_m = \Delta T_p + \Delta V_p.
\end{align}
Without long-range wavefunction components, the mixed-estimator kinetic FSC is approximately zero ($\Delta T_m \approx 0$). Therefore
\begin{align}
\left\{\begin{array}{l}
\Delta E \approx \Delta V_m \\
\Delta T_p \approx \Delta V_m - \Delta V_p
\end{array}\right..
\end{align}
FSC of the Virial pressure ($\Delta P$) is then
\begin{align}
\left\{\begin{array}{l}
\Delta P_m = (2\Delta T_m + \Delta V_m)/(3\Omega) \approx (\Delta V_m)/(3\Omega) \\
\Delta P_p = (2\Delta T_p + \Delta V_p)/(3\Omega) \approx (2\Delta V_m-\Delta V_p)/(3\Omega)
\end{array}\right., \label{eq:pm-pp-fsc}
\end{align}
where $\Omega$ is volume. Regardless of whether mix-estimator or pure-estimator value is used for the Virial pressure, the FSC DMC enthalpy-pressure data agree well with equation of state derived from the total energy (Fig.~\ref{fig:si-static-hp}).

\subsection{Finite Size Corrected Data}

\end{comment}

\begin{comment}
Finally, we divide each orbital by an electron-ion Jastrow constructed on the reciprocal lattice vectors of the simulation cell $\bs{G}_s$
\begin{align}
J_{ei}(\bs{r}; \bs{R}) = \exp\left\{ -\frac{1}{2\Omega} \sum\limits_{\bs{k}\neq\bs{0}}^{\bs{k}\in\bs{G}_s}
U_{\bs{k}}^{ei} 
\left(\frac{1}{N}\sum\limits_{j=1}^{N_e} e^{i\bs{k}\cdot\bs{r}_j} \right)
\left(\frac{1}{N}\sum\limits_{J=1}^{N_p} e^{-i\bs{k}\cdot\bs{R}_J}  \right)
\right\},\label{eq:rpa-ep-jas}
\end{align}
\begin{align}
J_{ei}(\bs{r}_j; \bs{R}) =& \exp\left\{ \text{iFFT}\left[ 
U_{\bs{k}}^{ep}  \left(\sum\limits_{J=1}^{N_p} \frac{e^{-i\bs{k}\cdot\bs{R}_J}}{N_p}  \right)
\right] \right\}\\
=& \exp\left\{ 
-\frac{(2\pi)^3}{\Omega N_k} \sum\limits_{\bs{k}\neq\bs{0}}^{\bs{k}\in\bs{G}_s} e^{i\bs{k}\cdot\bs{r}_j}~
U_{\bs{k}}^{ep} 
\left(\sum\limits_{J=1}^{N_p} \frac{e^{-i\bs{k}\cdot\bs{R}_J}}{N_p}  \right)
\right\},\label{eq:rpa-ep-jas}
\end{align}
where $\Omega$ is the simulation cell volume. $\bs{r}$ contain all electronic coordinates, $\bs{R}$ contain all ionic coordinates. $N_e$ is the number of electrons, while $N_p$ is the number of protons.
The Jastrow potential $U_{\bs{k}}^{ei}$ in eq.~(\ref{eq:rpa-ep-jas}) is chosen to be the RPA form written by Ceperley and Alder
\begin{align}
2U^{ei}_k = -a_k(1+a_k)^{-1/2},
\end{align}
where $a_k=\frac{12}{r_s^3k^4}$ in Hartree atomic units.
\end{comment}

\section{Results}

\subsection{Energy vs. Volume}
\begin{figure}[h]
\begin{minipage}{0.48\textwidth}
\includegraphics[width=0.8\columnwidth]{101sd_ev-s3}
%\caption{Energy v.s. volume.\label{fig:si-static-ev}}
\end{minipage}
\begin{minipage}{0.48\textwidth}
\includegraphics[width=0.8\columnwidth]{101se_hp}
%\caption{Enthalpy v.s. pressure.\label{fig:si-static-hp}}
\end{minipage}
\caption{Static-lattice DMC equation-of-state (EOS) relative to Drummond reference $E(\nu) = \frac{2.14020118}{\nu^2} + \frac{0.60521235}{\nu} - 0.6073132$, where $\nu$ is volume per proton in bohr$^3$ and $E(\nu)$ is in Hartree atomic unit. Relative energies are shown in meV per proton (meV/p). Each solid line is obtained using a fitted energy-volume EOS. The EOS is obtained by fitting the finite-size corrected (FSC) total energy as a quadratic function of inverse volume. The markers are finite-size corrected simulation data without performing a fit. \label{fig:static-qmc-vs-drummond}}
\end{figure}

\subsection{Enthalpy vs. Pressure}
In Fig.~\ref{fig:static-enthalpy-vs-pressure}, enthalpies of the candidate structures are plotted relative the my C2/c reference EOS (Yang SJB N72 dskcorr). The EOSs are obtained by fitting the total energy to a quadratic function of inverse volume. The solid lines with symbols are my results. The dot-dashed lines are McMinis' relative enthalpies extracted from Fig. 1 of ref.~\cite{McMinis2015}. The dashed line is McMinis' atomic structure enthalpy shifted by 48 meV/p (1.8 mha/p) to match my energies. The dotted line is Azadi's atomic structure enthalpy.

\begin{figure}[h]
\includegraphics[width=0.8\textwidth]{87b_refit}
\caption{Enthalpy v.s. pressure using Yang SJB N72 dskcorr C2/c as reference.\label{fig:static-enthalpy-vs-pressure}}
\end{figure}
