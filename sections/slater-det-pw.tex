\section{Slater Determinant in Plane Wave Basis}
\label{sec:wf-pw-sdet}
\textit{Based on notes from D. M. Ceperley dated Aug. 1 2018}

When the orbitals are expressed in plane wave basis
\begin{align}
\phi_i(\bs{r}) = \sum\limits_{\bs{k}} c_{i\bs{k}} e^{i\bs{k}\cdot\bs{r}}. \label{eq:pw-orb}
\end{align}
We require the orbitals to be orthonormal
\begin{align}
\int_\Omega d\bs{r} \phi_i(\bs{r})^*\phi_j(\bs{r}) = \delta_{ij} \Rightarrow 
\Omega \sum\limits_{\bs{k}} c_{i\bs{k}}^*c_{j\bs{k}} = \delta_{ij}. \label{eq:on_orbs}
\end{align}

We can verify that the determinant written in eq~(\ref{eq:det}) is normalized
\begin{align}
\braket{} \equiv& \int d\bs{r}_1\dots d\bs{r}_N \Psi^*~ \Psi \nonumber \\
=& \frac{1}{N!} \sum\limits_{\mathcal{P},\mathcal{P'}} (-1)^{\mathcal{P}} (-1)^{\mathcal{P'}}
\left(
\prod\limits_{l=1}^{N} \int d\bs{r}_l \phi^*_{\mathcal{P}_l}(\bs{r}_l)\phi_{\mathcal{P}_l'}(\bs{r}_l)
\right)\nonumber \\
=& \frac{1}{N!} \sum\limits_{\mathcal{P},\mathcal{P'}} (-1)^{\mathcal{P}} (-1)^{\mathcal{P'}}
\left(
\prod\limits_{l=1}^{N} \delta_{\mathcal{P}_l,\mathcal{P}_l'}
\right) \nonumber \\
=& \frac{1}{N!} \sum\limits_{\mathcal{P}} = 1. \label{eq:det-norm}
\end{align}
The key step in eq~(\ref{eq:det-norm}) is to separate and distribute the many-body integrals into the product.

%\subsection{Properties of the Slater Determinant}

Many properties of the slater determinant can be evaluated analytically. Here we focus on reciprocal-space properties accessible by scattering experiments: the momentum distribution $n(\bs{k})$ and the static structure factor $S(\bs{k})$. %They are the Fourier transform of the one-body reduced density matrix (1RDM) and two-body reduced density matrix (2RDM), respectively.

\subsection{Momentum Distribution}
\label{sec:wf-pw-sdet-nk}
The momentum distribution is the Fourier transform of the 1RDM (eq.~(5.9) in Ref.~\cite{Martin2016}). The 1RDM can be calculated from the many-body wavefunction
\begin{align}
\rho(\bs{x}, \bs{x}') = N\int d\bs{r}_2\dots\bs{r}_N \Psi^*(\bs{x}, \bs{r}_2, \dots) \Psi(\bs{x}', \bs{r}_2, \dots).
\end{align}
Given a Slater determinant wavefunction eq~(\ref{eq:det}), all the $d\bs{r}$ integrals can be done analytically
\begin{align}
\rho(\bs{x}, \bs{x}') =& N\int d\bs{r}_2\dots d\bs{r}_N
\left(
\frac{1}{N!}\sum\limits_{\mathcal{P}, \mathcal{P}'} (-1)^{\mathcal{P}} (-1)^{\mathcal{P}'}
\phi_{\mathcal{P}_1}^*(\bs{x})\phi_{\mathcal{P}_1'}(\bs{x}')
\prod\limits_{l=2}^{N}\phi_{\mathcal{P}_l^*}(\bs{r}_l)\phi_{\mathcal{P}_l'}(\bs{r}_l) 
\right) \nonumber \\
=& \frac{N}{N!} \sum\limits_{\mathcal{P}, \mathcal{P}'} (-1)^{\mathcal{P}} (-1)^{\mathcal{P}'}
\left(
\phi_{\mathcal{P}_1}^*(\bs{x})\phi_{\mathcal{P}_1'}(\bs{x}')
\prod\limits_{l=2}^N \delta_{\mathcal{P}_l,\mathcal{P}_l'}
\right) \nonumber \\
=& \frac{N}{N!} \sum\limits_{\mathcal{P}} \phi_{\mathcal{P}_1}^*(\bs{x})\phi_{\mathcal{P}_1}(\bs{x}') \nonumber \\
=& \sum\limits_{\mathcal{P}_1=1}^N \phi_{\mathcal{P}_1}^*(\bs{x})\phi_{\mathcal{P}_1}(\bs{x}'). \label{eq:det-1rdm}
\end{align}
Notice that the diagonal ($\bs{x}=\bs{x}'$) of the 1RDM is particle density. Given PW orbitals eq~(\ref{eq:pw-orb})
\begin{align}
n(\bs{k}) =& \frac{1}{(2\pi)^3N}\int d\bs{r} d\bs{r}'' e^{-i\bs{k}\cdot\bs{r}''} \rho(\bs{r}, \bs{r}-\bs{r}'') \nonumber \\
= & \frac{1}{(2\pi)^3N} \sum\limits_{i, \bs{g}, \bs{g}'} c_{i\bs{g}}^*c_{i\bs{g}'}
\int d\bs{r} d\bs{r}'' e^{-i\bs{g}\cdot\bs{r}}e^{i\bs{g}'\cdot(\bs{r}-\bs{r}'')} \nonumber \\
=&  \frac{1}{(2\pi)^3N} \sum\limits_{i, \bs{g}, \bs{g}'} c_{i\bs{g}}^*c_{i\bs{g}'} \Omega\delta_{\bs{g}, \bs{g}'}\Omega\delta_{\bs{g}',-\bs{k}} \nonumber \\
=& \frac{\Omega}{(2\pi)^3}\frac{\Omega}{N}\sum_{i=1}^{N} \vert c_{i,-\bs{k}}\vert^2.\label{eq:det-nofk}
\end{align}
Given the current definitions, $\int d\bs{k} n(\bs{k}) = 1$ for an infinite system. In practice, one bins the Fourier coefficient squared of all occupied orbitals at allowed momenta of the supercell.

\subsection{Static Structure Factor}
\label{sec:wf-pw-sdet-sk}
The static structure factor is the density-density correlation in reciprocal space
\begin{align}
S_{\bs{q}} \equiv& \frac{1}{N}\braket{\rho_{\bs{q}}\rho_{-\bs{q}}} \equiv
\frac{1}{N}\braket{
(\sum\limits_{i=1}^N e^{i\bs{q}\cdot\bs{r}_i})
(\sum\limits_{j=1}^N e^{-i\bs{q}\cdot\bs{r}_j})
} \nonumber \\
=& \frac{1}{N}\sum_{ij}\braket{e^{i\bs{q}\cdot(\bs{r}_i-\bs{r}_j)}} =
1 + \frac{1}{N}\sum_{i\neq j}\braket{e^{i\bs{q}\cdot(\bs{r}_i-\bs{r}_j)}} \nonumber \\
=& 1+(N-1)\braket{e^{i\bs{q}\cdot(\bs{r}_1-\bs{r}_2)}}.
\end{align}
Focus on the many-body integral
\begin{align}
\braket{e^{i\bs{q}\cdot(\bs{r}_1-\bs{r}_2)}} = \dfrac{1}{N!} \sum\limits_{\mathcal{P},\mathcal{P'}} 
(-1)^{\mathcal{P}} (-1)^{\mathcal{P'}} \int d\bs{r}_1\dots d\bs{r}_N
e^{i\bs{q}\cdot(\bs{r}_1-\bs{r}_2)}
\prod\limits_{l=1}^{N} \phi^*_{\mathcal{P}_l}(\bs{r}_l)\phi_{\mathcal{P}_l'}(\bs{r}_l). \label{eq:eiqr1r2}
\end{align}
Similar to eq~(\ref{eq:det-norm}) and eq~(\ref{eq:det-nofk}), $\mathcal{P}_l=\mathcal{P}'_l, \forall l\neq1, 2$. Define $\mathcal{P}_1=i$, $\mathcal{P}_2=j$, then $\mathcal{P}_{1,2}'=i, j$ contributes a positive term, and $\mathcal{P}_{1,2}'=j, i$ contributes a negative term. Thus eq~(\ref{eq:eiqr1r2}) simplifies
\begin{align}
\braket{e^{i\bs{q}\cdot(\bs{r}_1-\bs{r}_2)}} =& \dfrac{1}{N(N-1)} \sum\limits_{i, j}
 \int d\bs{r}_1d\bs{r}_2
e^{i\bs{q}\cdot(\bs{r}_1-\bs{r}_2)} \times \nonumber \\
&\left[
\phi^*_{i}(\bs{r}_1)\phi_{i}(\bs{r}_1)\phi^*_{j}(\bs{r}_2)\phi_{j}(\bs{r}_2) - 
\phi^*_{i}(\bs{r}_1)\phi_{j}(\bs{r}_1)\phi^*_{j}(\bs{r}_2)\phi_{i}(\bs{r}_2)
\right] \nonumber \\
=& \dfrac{1}{N(N-1)} \sum\limits_{i\neq j} \left[
\int d\bs{r}_1 e^{i\bs{q}\cdot\bs{r}_1} \phi^*_{i}(\bs{r}_1)\phi_{i}(\bs{r}_1) \right.
\int d\bs{r}_1 e^{-i\bs{q}\cdot\bs{r}_2} \phi^*_{j}(\bs{r}_2)\phi_{j}(\bs{r}_2) \nonumber \\
& \left. - \int d\bs{r}_1 e^{i\bs{q}\cdot\bs{r}_1} \phi^*_{i}(\bs{r}_1)\phi_{j}(\bs{r}_1)
\int d\bs{r}_1 e^{-i\bs{q}\cdot\bs{r}_2} \phi^*_{j}(\bs{r}_2)\phi_{i}(\bs{r}_2)
\right] \nonumber \\
=& \dfrac{1}{N(N-1)} \sum\limits_{i\neq j} \left[
m_{ii}(\bs{q})m_{jj}(-\bs{q}) - m_{ij}(\bs{q})m_{ji}(-\bs{q})
\right],
\end{align}
where we have defined the matrix of integrals
\begin{align}
m_{ij}(\bs{q}) \equiv \int d\bs{r} \phi_i^*(\bs{r}) \phi_j(\bs{r}) e^{i\bs{q}\cdot\bs{r}}. \label{eq:det-mijq}
\end{align}
Notice $m_{ij}^*(\bs{q}) = m_{ji}(-\bs{q})$, thus
\begin{align}
S_{\bs{q}} =& 1+\frac{1}{N} \sum_{i\neq j} \left[ m_{ii}(\bs{q})m_{jj}^*(\bs{q}) - m_{ij}(\bs{q})m_{ij}^*(\bs{q}) \right] \nonumber \\
=& 1+\frac{1}{N} \left[
\vert \sum_{i} m_{ii}(\bs{q}) \vert^2 -\sum_{i, j} \vert m_{ij}(\bs{q}) \vert^2
\right]. \label{eq:det-sofk}
\end{align}

\subsubsection{Example: Free Fermions}
The ground-state wavefunction of non-interacting fermions is a determinant of plane waves. The first $N$ plane-wave orbitals with the lowest momenta are filled. In case of degeneracy, the wavefunction will have a non-zero net momentum. %If $N$ does not fully fill the outer-most k-shell (there are degenerate unfilled orbitals at the Fermi energy), then the ground state must be an equal superposition of determinants. The determinant expansion should have $\left( \begin{array}{c} N_s\\ N_{left} \end{array} \right)$ terms, where $N_s$ is the number of states in the partially filled k-shell.

Simply stated, the free fermion wavefunction is a determinant eq~(\ref{eq:det}) whose orbitals each have a single Fourier component
\begin{align}
c_{i\bs{k}} = \frac{1}{\sqrt{\Omega}} \delta_{\bs{k},\bs{k}_i}.
\end{align}
We see from eq~(\ref{eq:det-nofk}) that the momentum distribution $n(\bs{k})$ of the free fermions is a step function, which is constant within the Fermi surface and zero outside. As for the static structure factor $S(\bs{k})$, first note that the matrix of integrals $m_{ij}(\bs{q})$ eq~(\ref{eq:det-mijq}) is sparse
\begin{align}
m_{ij}(\bs{q}) = c_{i\bs{k}_i}^* c_{j\bs{k}_j} \Omega\delta_{\bs{q}, \bs{k}_i-\bs{k}_j} = \delta_{\bs{q}, \bs{k}_i-\bs{k}_j}. \label{eq:det-free-mijq}
\end{align}
Plug (\ref{eq:det-free-mijq}) into (\ref{eq:det-sofk}) and
\begin{align}
S_{\bs{q}} = \left\{
\begin{array}{lr}
N & \bs{q}=\bs{0} \\
 1-\frac{1}{N}\sum\limits_{i, j} \vert \delta_{\bs{q}, \bs{k}_i-\bs{k}_j} \vert^2 & \bs{q}\neq\bs{0}
\end{array}\right..
\end{align}
Eq~(\ref{eq:det-free-mijq}) has a simple geometric interpretation. Namely, $m_{ij}(\bs{q})$ is non-zero only if $\bs{q}$ connects two occupied plane wave orbitals. In the thermodynamic limit, the geometric interpretation allows $S(\bs{k})$ to be calculated from a simple integral
\begin{align}
S(\bs{k}\neq\bs{0}) =& 1-\left(\frac{4\pi k_F^3}{3}\right)^{-1}2\int_{q/2}^{k_F} dk \pi(k_F^2-k^2) \nonumber \\
=& \left\{ \begin{array}{lr}
\frac{3}{4}\left(\frac{q}{k_F}\right) - \frac{1}{16} \left(\frac{q}{k_F}\right)^3 & q<2k_F\\
1 & q\ge 2k_F
\end{array}\right..
\end{align}