\chapter{Derivation of Higher-order Band Gap Correction}

Total potential energy
\begin{align}
V = \frac{N_e}{\Omega}\sum_{\bs{k}\neq\bs{0}} \frac{1}{2} v_k S_{\bs{k}},
\end{align}
where $S_{\bs{k}}\equiv\frac{1}{N_e}\braket{\rho_{\bs{k}}\rho_{-\bs{k}}}$. $\rho_{\bs{k}}\equiv \sum_{j=1}^{N_e} e^{i\bs{k}\cdot\bs{r}_j}$ is the electronic density in reciprocal space. $\Omega$ is system volume. $\braket{}$ denotes average over walker ensemble. The potential energy $V$ can be written as a sum of static (density) contribution and fluctuating contribution
\begin{align}
V = \left(
\frac{1}{\Omega} \sum_{\bs{k}\neq\bs{0}}
\frac{1}{2} v_k \braket{\rho_{\bs{k}}}\braket{\rho_{-\bs{k}}}
\right) +
\left(
\frac{1}{\Omega}\sum_{\bs{k}\neq\bs{0}}
\frac{1}{2} v_k  (\rho_{\bs{k}}-\braket{\rho_{\bs{k}}}) (\rho_{-\bs{k}}-\braket{\rho_{-\bs{k}}})
\right).
\end{align}
The fluctuating part can be used to calculate leading-order finite-size correction to the band gap, where the static part leads to the next-to-leading-order correction.
\begin{align}
V_s \equiv \frac{1}{\Omega}\sum_{\bs{k}\neq\bs{0}} \frac{1}{2} v_k \braket{\rho_{\bs{k}}}\braket{\rho_{-\bs{k}}}.
\end{align}
The goal of this Appendix is to find the FSC formula for $V_s$. Assuming the finite-size error in the electron density can be fully recovered by twist-averaging, the infinite-system potential energy can be computed using the twist-averaged density $\overline{\braket{\rho_{\bs{k}}}}$, where twist average is denoted by overline
\begin{align}
V_s^{N\rightarrow\infty} = \frac{1}{\Omega}\sum_{\bs{k}\neq\bs{0}} \frac{1}{2} v_k \overline{\braket{\rho_{\bs{k}}}} \ \ \overline{\braket{\rho_{-\bs{k}}}}. \label{eq:vsinf}
\end{align}
Therefore, the FSC of $V_s$ is
\begin{align}
\delta V_s \equiv V_s^{N\rightarrow\infty} - V_s = \frac{1}{\Omega}\sum_{\bs{k}\neq\bs{0}} \frac{1}{2} v_k \left[ \overline{\braket{\rho_{\bs{k}}}} \ \ \overline{\braket{\rho_{-\bs{k}}}} -
\braket{\rho_{\bs{k}}}\braket{\rho_{-\bs{k}}}
\right].
\end{align}
$\braket{\rho_{\bs{k}}}$ differs from twist to twist, so there is one such correction for each twist. Define
\begin{align}
C_{\bs{k}} \equiv  \left[ \overline{\braket{\rho_{\bs{k}}}} \ \ \overline{\braket{\rho_{-\bs{k}}}} -
\braket{\rho_{\bs{k}}}\braket{\rho_{-\bs{k}}}
\right], \label{eq:ck}
\end{align}
then the FSC of the potential energy is%, with the summand plotted for Si in Fig.~\ref{fig:si-cbyk2}
\begin{align}
\delta V_s = \frac{2\pi}{\Omega} \sum_{\bs{k}\neq\bs{0}} C_{\bs{k}}/k^2.
\end{align}
In simulation, we assemble the $MN_e+1$ system with varying density at each twist
\begin{align} \label{eq:rkt}
\rho_{\bs{k}}^{MN_e\pm 1} = \sum\limits_{\bs{\theta}}
\left(\braket{\rho_{\bs{k}}}_{N_e\pm1, \bs{\phi}}-\braket{\rho_{\bs{k}}}_{N_e, \bs{\phi}}\right)\delta_{\bs{\theta}, \bs{\phi}}+
\braket{\rho_{\bs{k}}}_{N_e,\bs{\theta}}. 
\end{align}
The GCTA corrected charge density varies only at the target twist $\bs{\phi}$
\begin{align} \label{eq:rkm}
\overline{\rho_{\bs{k}}}^{MN_e\pm 1} = \sum\limits_{\bs{\theta}}
\left(\braket{\rho_{\bs{k}}}_{N_e\pm1, \bs{\phi}}-\braket{\rho_{\bs{k}}}_{N_e, \bs{\phi}}\right)\delta_{\bs{\theta}, \bs{\phi}} + \overline{\rho_{\bs{k}}}^{N_e}.
\end{align}
The \emph{neutral mean density} $\overline{\rho_{\bs{k}}}^{N_e}$ has no dependence on the twist $\bs{\theta}$
\begin{align}
\overline{\rho_{\bs{k}}}^{N_e} \equiv \frac{1}{M_{\bs{\theta}}} \braket{\rho}_{N_e, \bs{\theta}}.
\end{align}
For simplicity, define the charge density of the particle/hole as
\begin{align}
\Pi_{\bs{k}} \equiv \braket{\rho_{\bs{k}}}_{N_e\pm1, \bs{\phi}}-\braket{\rho_{\bs{k}}}_{N_e, \bs{\phi}}.
\end{align}
Now, the GCTA correction can be clearly seen as the replacement of \emph{neutral twist density} $\braket{\rho_{\bs{k}}}_{N_e,\bs{\theta}}$ with \emph{neutral mean density} $\overline{\rho_{\bs{k}}}^{N_e}$
\begin{align}
\left\{
\begin{array}{lrl}
\rho_{\bs{k}}^{MN_e\pm 1} \equiv& \sum\limits_{\bs{\theta}} \rho^{N_e\pm1}_{\bs{k}}(\bs{\theta})
=& \sum\limits_{\bs{\theta}}\left(
\Pi_{\bs{k}}\delta_{\bs{\theta}, \bs{\phi}}+\braket{\rho_{\bs{k}}}_{N_e,\bs{\theta}}
\right) \\
\overline{\rho_{\bs{k}}}^{MN_e\pm 1} \equiv&  \sum\limits_{\bs{\theta}} \overline{\rho_{\bs{k}}}^{N_e\pm1}(\bs{\theta})
=& \sum\limits_{\bs{\theta}}\left(
\Pi_{\bs{k}}\delta_{\bs{\theta}, \bs{\phi}} + \overline{\rho_{\bs{k}}}^{N_e}
\right)
\end{array}
\right..
\end{align}
The \emph{charged twist density} $\rho^{N_e\pm1}_{\bs{k}}(\bs{\theta})$ and the \emph{charged mean density} $\overline{\rho_{\bs{k}}}^{N_e\pm 1}(\bs{\theta})$ are defined by the summand on each line. A likely cause of confusion here is that the charged mean density depends on the twist $\bs{\theta}$. Further, it is \textcolor{red}{not} the mean of the charged twist density
\begin{align}
\overline{\rho_{\bs{k}}}^{N_e\pm 1}(\bs{\theta}') \textcolor{red}{\neq}
\frac{1}{M_{\bs{\theta}}}\sum\limits_{\bs{\theta}}\rho^{N_e\pm1}_{\bs{k}}(\bs{\theta}).
\end{align}
This correction does not affect properties linear in $\rho_{\bs{k}}$, but does change the potential energy, which is quadratic in $\rho_{\bs{k}}$. If the $MN_e$ system is constructed from $M$ independent simulations each containing $N_e$ electrons, then the charge density is different at each twist. The Hartree contribution to the total potential energy is
\begin{align}\label{eq:vsmnp1}
V_s^{MN_e+1} = \frac{1}{\Omega} \sum\limits_{\bs{\theta}} \sum\limits_{\bs{k}\in G}
\frac{1}{2}v_k~
\rho^{N_e+1}_{\bs{k}}(\bs{\theta}) \rho^{N_e+1}_{-\bs{k}}(\bs{\theta}). 
\end{align}
One can speed up the convergence of eq.~(\ref{eq:vsmnp1}) to the thermodynamic limit by replacing twist density with mean density
%correcting each twist to interact with the same mean potential generated from the mean charge density eq.~(\ref{eq:rkm}). Therefore, the finite-size correction to the total potential energy is
\begin{align}
\lim\limits_{N\rightarrow\infty} V_s^{MN_e+1} = \frac{1}{\Omega} \sum\limits_{\bs{\theta}} \sum\limits_{\bs{k}\in G}
\frac{1}{2}v_k~
\overline{\rho_{\bs{k}}}^{N_e+1}(\bs{\theta}) \overline{\rho_{-\bs{k}}}^{N_e+1}(\bs{\theta}).
\end{align}
Define the correction factor
\begin{align}\label{eq:gc-ckt}
C_{\bs{k}}^{N_e+1}(\bs{\theta}) \equiv& \overline{\rho}^{N_e+1}_{\bs{k}}(\bs{\theta})\overline{\rho}^{N_e+1}_{-\bs{k}}(\bs{\theta})
-
\rho^{N_e+1}_{\bs{k}}(\bs{\theta})\rho^{N_e+1}_{-\bs{k}}(\bs{\theta}),
\end{align}
then the GCTA finite-size correction to the Hartree contribution to the potential is
\begin{align}\label{eq:dvsmnp1}
\delta V_s^{MN_e+1} = \frac{1}{\Omega} \sum\limits_{\bs{k}\in G}\frac{1}{2}v_k~
\sum\limits_{\bs{\theta}}
C_{\bs{k}}^{N_e+1}(\bs{\theta}).
\end{align}
Eq.~(\ref{eq:gc-ckt}) can be much simplified when calculated relative to the neutral state
\begin{align}
C_{\bs{k}}^{N_e+1}(\bs{\theta})=&
(\Pi_{\bs{k}}\delta_{\bs{\theta}, \bs{\phi}} + \overline{\rho_{\bs{k}}}^{N_e})(\Pi_{\bs{k}}\delta_{\bs{\theta}, \bs{\phi}} + \overline{\rho_{\bs{k}}}^{N_e}) -
(\Pi_{\bs{k}}\delta_{\bs{\theta}, \bs{\phi}}+\braket{\rho_{\bs{k}}}_{N_e,\bs{\theta}})(\Pi_{\bs{k}}\delta_{\bs{\theta}, \bs{\phi}}+\braket{\rho_{\bs{k}}}_{N_e,\bs{\theta}}) \nonumber \\
=& \Pi_{\bs{k}}\delta_{\bs{\theta}, \bs{\phi}}(\overline{\rho_{-\bs{k}}}^{N_e} - \braket{\rho_{-\bs{k}}}_{N_e, \bs{\theta}}) +
(\overline{\rho_{\bs{k}}}^{N_e} - \braket{\rho_{\bs{k}}}_{N_e, \bs{\theta}})\Pi_{-\bs{k}}\delta_{\bs{\theta}, \bs{\phi}} \nonumber \\
&+
\left[
\overline{\rho_{\bs{k}}}^{N_e}\overline{\rho_{-\bs{k}}}^{N_e} - \braket{\rho_{\bs{k}}}_{N_e, \bs{\theta}}\braket{\rho_{\bs{k}}}_{N_e, \bs{\theta}}
\right] \nonumber \\
=& \Pi_{\bs{k}}\delta_{\bs{\theta}, \bs{\phi}}(\overline{\rho_{-\bs{k}}}^{N_e} - \braket{\rho_{-\bs{k}}}_{N_e, \bs{\theta}}) +
(\overline{\rho_{\bs{k}}}^{N_e} - \braket{\rho_{\bs{k}}}_{N_e, \bs{\theta}})\Pi_{-\bs{k}}\delta_{\bs{\theta}, \bs{\phi}} + C_{\bs{k}}^{N_e}(\bs{\theta}).
\end{align}
Therefore, the correction to electron addition energy is
\begin{align}
\delta\mu_s^+ =& \frac{1}{\Omega} \sum\limits_{\bs{k}\in G} \frac{1}{2}v_k \left[ \sum\limits_{\bs{\theta}} C_{\bs{k}}^{N_e+1}(\bs{\theta})-C_{\bs{k}}^{N_e}(\bs{\theta}) \right] \nonumber \\
=& \frac{1}{\Omega} \sum\limits_{\bs{k}\in G} \frac{1}{2}v_k \left\{
\left[
\overline{\rho_{\bs{k}}}^{N_e}-\braket{\rho_{\bs{k}}}_{N_e,\bs{\phi}}
\right]\Pi_{-\bs{k}} +\Pi_{\bs{k}}\left[
\overline{\rho_{-\bs{k}}}^{N_e}-\braket{\rho_{-\bs{k}}}_{N_e,\bs{\phi}}
\right]
\right\}.
\end{align}
